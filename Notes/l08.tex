\documentclass[11pt]{article}
\usepackage[utf8]{inputenc}
\usepackage[czech, english]{babel}
\usepackage[left=2cm,right=2cm,top=2cm,bottom=2cm]{geometry}
\usepackage{amssymb}
\usepackage{amsthm}
\usepackage{amsmath}
\usepackage{tikz}
\usetikzlibrary{shapes.geometric}
\usetikzlibrary{arrows}
\usepackage{hyperref}


% Itemize environment with small skips
\newenvironment{packeditemize}{
\begin{itemize}
  \setlength{\itemsep}{1pt}
  \setlength{\parskip}{0pt}
  \setlength{\parsep}{0pt}
}{\end{itemize}}


% Fancy footnote....
\usepackage{fancyhdr}
\pagestyle{fancy}
\usepackage{lastpage}
\rfoot{MATH 566 - 08, page \thepage/\pageref{LastPage}}
\cfoot{}
\rhead{}
\lhead{}
\renewcommand{\headrulewidth}{0pt}
\renewcommand{\footrulewidth}{0pt}


\newtheorem{theorem*}{Theorem} 

% Itemize environment with small skips
\newenvironment{packedenumerate}{
\begin{enumerate}
  \setlength{\itemsep}{1pt}
  \setlength{\parskip}{0pt}
  \setlength{\parsep}{0pt}
}{\end{enumerate}}


\newcounter{excounter}
\setcounter{excounter}{1}
\newcommand\question[2]{\vskip 1em  \noindent\textbf{\arabic{excounter}\addtocounter{excounter}{1}:} \emph{#1} \noindent#2}
\newcommand\solution[1]{\vskip 0.5em \noindent\textbf{Solution:} #1}
\newcommand\like{\par \noindent\emph{(This question is: good - bad - ugly? Difficulty: 0-10:\hskip 3em )}} 

\newcommand\lecturer[1]{\textbf{#1}}
\newcommand\hideforstudent[1]{#1}
\newcommand\skipforstudent[1]{#1}


%\renewcommand\lecturer[1]{{\color{white} \textbf{#1} }}
%\renewcommand\hideforstudent[1]{{\color{white} #1 }}
%\renewcommand\solution[1]{{\color{white} \vskip 0.5em \noindent\textbf{Solution:} #1 }}
%\renewcommand\skipforstudent[1]{}



\setlength{\parindent}{0cm}
\setlength{\parskip}{0.1cm}

\begin{document}

Fall  2016, MATH-566

\centerline{{\Large \textbf{Linear Programming Algorithms - Simplex method}}}

Source: Chapters 4,5 of \href{http://www.springer.com/us/book/9783540306979}{Matou\v{s}ek}

Assume linear program $(P)$ in \emph{equational} form:
\[
(P) \begin{cases} 
 \text{maximize} & \mathbf{c}^T\mathbf{x} \\
 \text{subject to} & A\mathbf{x} = \mathbf{b} \\
                           & \mathbf{x} \geq  \mathbf{0}
\end{cases}
\]
where $A \in \mathbb{R}^{m \times n}$, $\mathbf{b} \in \mathbb{R}^m$ and $\mathbf{c} \in \mathbb{R}^n$.

\question{Can we assume that rows of $A$ are linearly independent? Can we assume $n \geq m$?}
\solution{
Yes - if two rows are linearly dependent, then either there is no solution or one is useless.
If the rows are linearly independent, we get for free that $n \geq m$.
}

A solution $\mathbf{x}$ is called $\emph{basic feasible solution}$ if $n-m$ entries of $\mathbf{x}$ are zero
and the columns of $A$ corresponding to these remaining $m$ entries are linearly independent.

\question{Is it possible to find two different basic feasible solutions with the same
positions of $n-m$ zero entries?}
\solution{
No - if the solution has $m$ nonzeros. The system of equations $m \times m$ has a unique solution since the columns of $A$
are linearly independent.
Yes - if the actual optimal solution has more zeros, then some could be replaced by different columns.
}

\begin{theorem*}
If program $(P)$ has an optimal solution, it also has an optimal solution that is a basic feasible solution.
\end{theorem*}

Corollary: Optimal solution to $(P)$ can be found by examining all $\binom{n}{m}$  subsets of columns of $A$.


A set $B \subset \{1,\ldots,n\}$ is \emph{base} if columns of $A$ indexed by $B$ give a basic feasible solution (denoted by $A_B$).

Simplex method: Start with a base and alter the base by changing one entry at a time. 

Example of simplex method:
\[
(P) \begin{cases} 
\begin{array}{lrccccc} 
 \text{maximize}  &x_1  &+&  &x_2  &        &   \\
 \text{subject to}  & -x_1 &+&  &x_2   &\leq  &1 \\
                           & x_1  &   &  &         & \leq & 3 \\
                           &         &   &  & x_2   &  \leq & 2\\ 
\end{array}\\
\text{ \hskip 8em }
 x_1,x_2 \geq 0
 \end{cases}
\]

\question{}{
Rewrite the program to equational form. \emph{(Hint: use slack variables - that is add 3 more variables)}
}
\solution{
\[
(P) \begin{cases} 
\begin{array}{lrccccccccccc} 
 \text{maximize}  &x_1  &+&  &x_2  &        &   \\
 \text{subject to}  & -x_1 &+&  &x_2   & + & x_3 &    &         &    &         &=  &1 \\
                           & x_1  &   &  &         &    &        & + & x_4 &    &         & = & 3 \\
                           &         &   &  & x_2   &   &        &    &        & + & x_5 & = & 2\\ 
\end{array}\\
\text{ \hskip 15em }
 x_1,x_2,x_3,x_4,x_5 \geq 0
 \end{cases}
\]
}
\question{}{
Is there some obvious basic feasible (not necessarily optimal solution)?
}
\solution{$x_3 = 1$, $x_4 = 3$, $x_5 = 2$, the base is nice identity matrix.}

We create a thing called \emph{simplex tableau} for base $B=\{3,4,5\}$:
\[
\begin{array}{ccccccc} 
x_3  &   =  &  1  &+&   x_1 &-&   x_2 \\
x_4  &   =  &  3  &- &    x_1 & &          \\
x_5  &   =  &  2  &  &          &-&   x_2 \\
\hline 
z   &   =     &  0  &+ &    x_1 &+&   x_2 \\
\end{array}
\]
Features: 
$A_B$ is identity matrix, solution is obvious, if non-basis veriables are = 0, we keep only ``$A - A_B$''.
$z$ stands for the value of the objective function

\question{}{
Will $z$ increase if we increase $x_1$ or $x_2$? How much we can increase $x_2$ if $x_1$ is kept at zero?
}
\solution{
Both $x_1$ and $x_2$ have positive coefficient in the objective function. So increasing
any of them will result in increase of the objective function.

If we are increasing $x_2$, then $x_3$ and $x_5$ are decreasing. So we cannot make
$x_2 > 1$, otherwise $x_3 < 0$, which is not a feasible solution. Note that increasing $x_2$
to $1$ will make $x_3 = 0$.
}

\question{}{
Increase $x_2$ as much as you can put it in the base. Use steps like in Gauss elimination
to have $x_2$ instead if $x_3$ in the left top corner and nowhere else in the tableau.
Note that the base will change to $B=\{2,4,5\}$.
}
\solution{
\[
\begin{array}{ccccccc} 
x_3  &   =  &  1  &+&   x_1 &-&   x_2 \\
x_4  &   =  &  3  &- &    x_1 & &          \\
x_5  &   =  &  2  &  &          &-&   x_2 \\
\hline 
z   &   =     &  0  &+ &    x_1 &+&   x_2 \\
\end{array}
\sim
\begin{array}{ccccccc} 
x_3  &   =  &  1  &+&   x_1 &-&   x_2 \\
x_4  &   =  &  3  &- &    x_1 & &          \\
x_5-x_3  &   =  &  1  &- &x_1          & &    \\
\hline 
z+x_3   &   =     &  1  &+ &    2x_1 &   &    \\
\end{array}
\sim
\begin{array}{ccccccc} 
x_2  &   =  &  1  &+&   x_1 &-&   x_3 \\
x_4  &   =  &  3  &- &    x_1 & &          \\
x_5  &   =  &  1  &- &  x_1        &+&x_3    \\
\hline 
z   &   =     &  1  &+ &    2x_1 & -  &x_3    \\
\end{array}
\]
}
The process when one variable is entering the base and another is leaving is called the \textbf{pivot step}.
\question{}{
Do another pivot step.
Which of the variables in the objective function could be increased next? 
Increase it as much as possible and do a swap in the tableau 
as happened for $x_2$ and $x_3$.
}
\solution{
Since $x_3$ has negative coefficient, we should not increase it. But $x_2$ could be increased.
The equation that limits the increase of $x_5$ is the third one. Hence we take $x_1$ from there.
\[
\begin{array}{ccccccc} 
x_2  &   =  &  1  &+&   x_1 &-&   x_3 \\
x_4  &   =  &  3  &- &    x_1 & &          \\
x_5  &   =  &  1  &- &  x_1        &+&x_3    \\
\hline 
z   &   =     &  1  &+ &    2x_1 & -  &x_3    \\
\end{array}
\sim
\begin{array}{ccccccc} 
x_2+x_5  &   =  &  2  & &      & &    \\
x_4-x_5  &   =  &  2  &   &       & -&x_3          \\
x_5  &   =  &  1  &- &  x_1        &+&x_3    \\
\hline 
z+2x_5   &   =     &  3  &  &     & +  &2x_3    \\
\end{array}
\sim
\begin{array}{ccccccc} 
x_2  &   =  &  2  &- & x_5    & &    \\
x_4  &   =  &  2  &+ & x_5   & -&x_3          \\
x_1  &   =  &  1  &- &  x_5        &+&x_3    \\
\hline 
z   &   =     &  3  & - &2x_5     & +  &x_3    \\
\end{array}
\]
}
\question{}{
Can you do more pivot steps or is this the optimal solution? When is solution optimal?
}
\solution{
We can still increase $x_3$ and increase $z$. The bound is $x_4$.
\[
\begin{array}{ccccccc} 
x_2  &   =  &  2  &- & x_5    & &    \\
x_4  &   =  &  2  &+ & x_5   & -&x_3          \\
x_1  &   =  &  1  &- &  x_5        &+&x_3    \\
\hline 
z   &   =     &  3  & - &2x_5     & +  &x_3    \\
\end{array}
\sim
\begin{array}{ccccccc} 
x_2  &   =  &  2  &- & x_5    & &    \\
x_4  &   =  &  2  &+ & x_5   & -&x_3          \\
x_1+x_4  &   =  &  3  &   &          &    &    \\
\hline 
z+x_4 &   =     &  5  & -  & x_5       &     &     \\
\end{array}
\sim
\begin{array}{ccccccc} 
x_2  &   =  &  2  &- & x_5    & &    \\
x_3  &   =  &  2  &+ & x_5   & -&x_4          \\
x_1  &   =  &  3  &   &         & - &x_4    \\
\hline 
z &   =     &  5  & -  & x_5  & -    &x_4     \\
\end{array}
\]
Now we have the optimal solution, because increase in any of the variables will not result in increase
of the objective function.
}

\newpage

\question{}{
Draw the polytope of feasible solutions of program $(P)$ (the original program in 2 variables $x_1$ and $x_2$.
Mark points that correspond to the steps of the solutions using simplex method and draw
the direction of the objective function.
}
%\solution{
\begin{center}
\begin{tikzpicture}
\skipforstudent{
\fill[gray!20] (0,0)--(3,0)--(3,2)--(1,2)--(0,1)--cycle;
}
\draw[help lines] (-1,-1) grid (4,3); 
\draw[-left to,line width=0.7pt](-1,0) -- (4,0);
\draw[-right to,line width=0.7pt](0,-1) -- (0,3);
\skipforstudent{
\draw[-right to,line width=0.7pt](-1,2) -- (4,2);
\draw[-left to,line width=0.7pt](3,-1) -- (3,3);
\draw[-right to,line width=0.7pt](-1,0) -- (2,3);
\draw 
(0,0)node[label=below left:{$(0,0)$}]{}
(0,1)node[label=left:{$(0,1)$}]{}
(1,2)node[label=above left:{$(1,2)$}]{}
(3,2)node[label=above right:{$(3,2)$}]{}
;
\draw[line width=2pt,-latex] (0,0) -- (0,1);
\draw[line width=2pt,-latex] (0,1) -- (1,2);
\draw[line width=2pt,-latex] (1,2) -- (3,2);
\draw[line width=2pt,-latex,dashed] (0,0) -- (1,1);
}
\end{tikzpicture}
\end{center}
%}

\question{}{
Use simplex method on the following example:
\[
(P) \begin{cases} 
\begin{array}{lrccccc} 
 \text{maximize}  &         & &  &x_2  &        &   \\
 \text{subject to}  & -x_1 &+&  &x_2   &\leq  &0 \\
                           & x_1  &   &  &         & \leq & 2 \\
\end{array}\\
\text{ \hskip 8em }
 x_1,x_2 \geq 0
 \end{cases}
\]
That is, convert to the equational form and do iterations until optimum solution is reached.
}
\solution{
\[
\begin{array}{ccccccc} 
x_3  &   =  &  0  &   & x_1   &-& x_2   \\
x_4  &   =  &  2  &- & x_1   &  &         \\
\hline 
z   &   =     &  0  &   &        &   &x_2    \\
\end{array}
\sim
\begin{array}{ccccccc} 
x_2  &   =  &  0  &   & x_1   &-& x_3   \\
x_4  &   =  &  2  &- & x_1   &  &         \\
\hline 
z   &   =     &  0  &   & x_1  & -&x_3    \\
\end{array}
\sim
\begin{array}{ccccccc} 
x_1  &   =  &  2  &    &    &-&x_4         \\
x_2  &   =  &  2  &   -& x_3   &-& x_4   \\
\hline 
z   &   =     &  2  &   -& x_3  & -&x_4    \\
\end{array}
\]
\vskip 5em
}

\question{}{
What happens with the objective function in the first step?
}
\solution{
Nothing. It is staying zero. 
}

\begin{itemize}
\item Simplex tableau is usually written in a matrix form (more condensed).
\item There are versions - revised simplex method, dual simplex method for minimization, \ldots
\item It is possible to construct an example that simplex method will cycle and never find a solution, if the pivot is chosen badly.
\item Smart choice of pivot (Band's pivot rule - lexicographic rule) avoids cycling.
\item There are many choices of pivot rules.
\item polytopes may have many vertices (see cyclic polytope) but there is a chance of short
path between any two vertices (initial solution and optimal solution) - recall Hirsh's conjecture.
\item pivot rules can be tricked to walk through all vertices of cube (Klee-Minty cube)
\end{itemize}


\end{document}



