\documentclass[11pt]{article}
\usepackage[utf8]{inputenc}
\usepackage[czech, english]{babel}
\usepackage[left=2cm,right=2cm,top=2cm,bottom=2cm]{geometry}
\usepackage{amssymb}
\usepackage{amsthm}
\usepackage{amsmath}
\usepackage{tikz}
\usetikzlibrary{shapes.geometric}
\usepackage{hyperref}


% Itemize environment with small skips
\newenvironment{packeditemize}{
\begin{itemize}
  \setlength{\itemsep}{1pt}
  \setlength{\parskip}{0pt}
  \setlength{\parsep}{0pt}
}{\end{itemize}}


% Fancy footnote....
\usepackage{fancyhdr}
\pagestyle{fancy}
\usepackage{lastpage}
\rfoot{MATH 566 - 14, page \thepage/\pageref{LastPage}}
\cfoot{}
\rhead{}
\lhead{}
\renewcommand{\headrulewidth}{0pt}
\renewcommand{\footrulewidth}{0pt}


\newtheorem{theorem*}{Theorem} 

% Itemize environment with small skips
\newenvironment{packedenumerate}{
\begin{enumerate}
  \setlength{\itemsep}{1pt}
  \setlength{\parskip}{0pt}
  \setlength{\parsep}{0pt}
}{\end{enumerate}}


\newcounter{excounter}
\setcounter{excounter}{1}
\newcommand\question[2]{\vskip 1em  \noindent\textbf{\arabic{excounter}\addtocounter{excounter}{1}:} \emph{#1} \noindent#2}
\newcommand\solution[1]{\vskip 0.5em \noindent\textbf{Solution:} #1}
%\newcommand\like{\par \noindent\emph{(This question is: good - bad - ugly? Difficulty: 0-10:\hskip 3em )}} 

\newcommand\lecturer[1]{\textbf{#1}}
\newcommand\hideforstudent[1]{#1}


%\renewcommand\lecturer[1]{{\color{white} \textbf{#1} }}
%\renewcommand\hideforstudent[1]{{\color{white} #1 }}
%\renewcommand\solution[1]{{\color{white} \vskip 0.5em \noindent\textbf{Solution:} #1 }}



\setlength{\parindent}{0cm}
\setlength{\parskip}{0.1cm}

\begin{document}

Fall  2016, MATH-566

\centerline{{\Large \textbf{Network flows - first introduction}}}


\emph{Original application (product of cold war)}

Suppose country $X$ enters into a war. How quickly can $X$ move tank from storage $s$ 
to the target $t$ (battle ground)? The tanks are moved on a railroad. Every link gives the capacity
how many tank a day can be transported. 

\begin{center}
\tikzset{
insep/.style={inner sep=2pt, outer sep=0pt, circle, fill}, 
}
\begin{tikzpicture}[scale=1.3]
\draw [-latex] (0,0) node[insep,label=left:$s$](s){} -- ++(30:2) node[insep,label=above:$u$](u){} node[pos=0.5,above]{$30$} ;
\draw [-latex] (s) -- ++(-45:1) node[insep,label=below:$v$](v){} node[pos=0.5,above]{$20$} ;
\draw [-latex] (v) -- ++(0:2) node[insep,label=below:$w$](w){} node[pos=0.5,above]{$30$} ;
\draw [-latex] (u) -- ++(-30:2) node[insep,label=right:$t$](t){} node[pos=0.5,above]{$20$} ;
\draw [-latex] (u) -- (w) node[pos=0.5,right]{$20$} ;
\draw [-latex] (w) -- (t) node[pos=0.5,above]{$40$} ;
\end{tikzpicture}
\end{center}

\question{}{
How many tanks per day can be delivered to the battleground? Is the solution unique?
}

\solution{
50 tanks. Not unique.
}

\textbf{Problem:}
(Directed) graph $G$, source $s$, sink $t$, capacities $u: E(G) \rightarrow \mathbb{R}_+$.\\
\textbf{Network} is $(G,u,s,t)$.

Input:  Network $(G,u,s,t)$\\
Output: $s$-$t$-flow of maximum value
 
 \textbf{$s$-$t$-flow} $f$ is a function $f: E(G) \rightarrow \mathbb{R}_+$.\\
 \textbf{Value} of $f$ is $\sum_{su \in E} f(su) - \sum_{us \in E} f(us)$ i.e. leaving $-$ entering to $s$.
 
 \question{}{
How $f$ looks around one vertex of the network? (what must $f$ satisfy?)
 }
\solution{
\textbf{Flow conservation rule:} For all $v \in V\setminus\{s,t\}$:
\[
\sum_{uv \in E} f(uv) = \sum_{vu \in E} f(uv).
\]
and for $s,t$ it satisfies:
\[
\sum_{su \in E} f(su) - \sum_{us \in E} f(us) = \sum_{ut \in E} f(ut) - \sum_{tu \in E} f(tu).
\]
} 

\question{}{
How to improve these flows to be maximum? Description on edges are  values of $f,u$.
\begin{center}
\begin{tikzpicture}[scale=1.2]
\tikzset{insep/.style={inner sep=2pt, outer sep=0pt, circle, fill},}
\draw [-latex] (0,0) node[insep,label=left:$s$](s){} -- ++(45:1.5) node[insep,label=above:$u$](u){} node[pos=0.5,above]{$2,2$} ;
\draw [-latex] (s) -- ++(-45:1) node[insep,label=below:$v$](v){} node[pos=0.5,above]{$0,3$} ;
\draw [-latex] (v) -- ++(0:2) node[insep,label=below:$w$](w){} node[pos=0.5,above]{$0,2$} ;
\draw [-latex] (u) -- ++(-20:3) node[insep,label=right:$t$](t){} node[pos=0.5,above]{$2,2$} ;
\draw [-latex] (u) -- (w) node[pos=0.5,right]{$0,2$} ;
\draw [-latex] (w) -- (t) node[pos=0.5,above]{$0,3$} ;
\end{tikzpicture}
\begin{tikzpicture}[scale=1.2]
\tikzset{insep/.style={inner sep=2pt, outer sep=0pt, circle, fill},}
\draw [-latex] (0,0) node[insep,label=left:$s$](s){} -- ++(45:1.5) node[insep,label=above:$u$](u){} node[pos=0.5,above]{$2,2$} ;
\draw [-latex] (s) -- ++(-45:1) node[insep,label=below:$v$](v){} node[pos=0.5,above]{$0,3$} ;
\draw [-latex] (v) -- ++(0:2) node[insep,label=below:$w$](w){} node[pos=0.5,above]{$0,2$} ;
\draw [-latex] (u) -- ++(-20:3) node[insep,label=right:$t$](t){} node[pos=0.5,above]{$0,3$} ;
\draw [-latex] (u) -- (w) node[pos=0.5,right]{$2,2$} ;
\draw [-latex] (w) -- (t) node[pos=0.5,above]{$2,2$} ;
\end{tikzpicture}
\end{center}
}
\solution{
Path $s,v,w$ can be increased by 2. Path $s,v,w,u,t$ can be increased by two.
}

\question{}{
After the improvement, how do you argue that nobody can further improve?
}
\solution{
Consider set $A=\{s,v\}$. The capacity of edges going from $A$ is 4. Hence no flow
can route more than 4 through the network.
}
%\newpage

Let $A \subset V(G)$ such that $s \in A$ and $t \not\in A$. 
Use $\delta^+(A)$ to denote set of edges $uv$, where $u \in A$ and $v \not\in A$ (edges leaving $A$). 
Use $\delta^-(A)$ to denote set of edges $uv$, where $u \not\in A$ and $v \in A$ (edges entering $A$). \\
\textbf{Capacity} of $s$-$t$-cut $A$ is $\sum_{e \in \delta^+(A)} u(e)$.

\question{}{
Prove that for $A$ and any flow $f$ holds\\
(a) $\text{value}(f) =  \sum_{e \in \delta^+(A)} f(e) - \sum_{e \in \delta^-(A)} f(e) $\\
(b) $\text{value}(f) \leq  \sum_{e \in \delta^+(A)} u(e)$
}
\solution{
(a) Use conservation of flow at vertices in $A$.
\begin{align*}
\text{value}(f) &=  \sum_{e \in \delta^+(s)} f(e) - \sum_{e \in \delta^-(s)} f(e)  \\
                      &= \sum_{v \in A}\left(  \sum_{e \in \delta^+(v)} f(e) - \sum_{e \in \delta^-(v)} f(e)  \right) \\
                      &= \sum_{e \in \delta^+(A)} f(e) - \sum_{e \in \delta^-(A)} f(e)
\end{align*}
(b) clearly $\sum_{e \in \delta^+(A)} f(e) - \sum_{e \in \delta^-(A)} f(e) \leq\sum_{e \in \delta^+(A)} u(e)$
since $f(e) \leq u(e)$.
}

This proves the \emph{obvious} observation that maximum flow cannot exceed capacity of minimum cut.

\vskip 1em



Notice in 3. we were improving flow be reducing the flow on $uw$. We ``sent flow in the opposite direction''.

For a digraph $G$, define $\overleftrightarrow{G}$ by adding for every edge $e$ also its \textbf{reverse} $\overleftarrow{e}$.

For $f$ and $u$ define \textbf{residual capacities} $u_f: E(\overleftrightarrow{G}) \rightarrow \mathbb{R}_+ $
\begin{align*}
u_f(e) &=  u(e) - f(e)    &       u_f(\overleftarrow{e}) &= f(e)
\end{align*}
Residual capacities \ldots how much extra we can send in each direction.

\textbf{Residual graph} $G_f$ is  obtained from $\overleftrightarrow{G}$ by removing edges $e \in E(\overleftrightarrow{G})$ with $u_f(e) = 0$.

\textbf{Augmenting path} is an $s$-$t$ path in $G_f$.

\question{}{
Construct the residual graph for
\begin{center}
\begin{tikzpicture}[scale=1.2]
\tikzset{insep/.style={inner sep=2pt, outer sep=0pt, circle, fill},}
\draw [-latex] (0,0) node[insep,label=left:$s$](s){} -- ++(45:1.5) node[insep,label=above:$u$](u){} node[pos=0.5,above]{$2,2$} ;
\draw [-latex] (s) -- ++(-45:1) node[insep,label=below:$v$](v){} node[pos=0.5,above]{$1,3$} ;
\draw [-latex] (v) -- ++(0:2) node[insep,label=below:$w$](w){} node[pos=0.5,above]{$1,4$} ;
\draw [-latex] (u) -- ++(0:3) node[insep,label=right:$x$](x){} node[pos=0.5,above]{$1,2$} ;
\draw [-latex] (w) -- ++(45:1) node[insep,label=right:$t$](t){} node[pos=0.5,right]{$1,1$} ;
\draw [-latex] (u) -- (w) node[pos=0.5,right]{$1,2$} ;
\draw [-latex] (x) -- (t) node[pos=0.5,left]{$2,4$} ;
\draw [-latex] (w) to[bend right=40,looseness=2] node[pos=0.5,right]{$1,2$}  (x);
\end{tikzpicture}
\end{center}
and find an augmenting path and increase the flow using the augmenting path.
}

\vskip 3em

\question{}{
How to update (to \textbf{augment} the flow $f$ using augmenting path in $\overleftrightarrow{G}$?
}

\solution{
If flow is increasing on edge $e$ by $\gamma$, then decrease $u_f(e)$ by $\gamma$ and increase $u_f(\overleftarrow{e})$
by $\gamma$. }

\end{document}






