\documentclass[11pt]{article}
\usepackage[utf8]{inputenc}
\usepackage[czech, english]{babel}
\usepackage[left=2cm,right=2cm,top=2cm,bottom=2cm]{geometry}
\usepackage{amssymb}
\usepackage{amsthm}
\usepackage{amsmath}
\usepackage{tikz}
\usetikzlibrary{shapes.geometric}
\usetikzlibrary{shapes.multipart}
\usetikzlibrary{arrows}
\usetikzlibrary{graphs}
\usepackage{hyperref}


% Itemize environment with small skips
\newenvironment{packeditemize}{
\begin{itemize}
  \setlength{\itemsep}{1pt}
  \setlength{\parskip}{0pt}
  \setlength{\parsep}{0pt}
}{\end{itemize}}


% Fancy footnote....
\usepackage{fancyhdr}
\pagestyle{fancy}
\usepackage{lastpage}
\rfoot{MATH 566 - 22, page \thepage/\pageref{LastPage}}
\cfoot{}
\rhead{}
\lhead{}
\renewcommand{\headrulewidth}{0pt}
\renewcommand{\footrulewidth}{0pt}


\newtheorem{theorem*}{Theorem} 

% Itemize environment with small skips
\newenvironment{packedenumerate}{
\begin{enumerate}
  \setlength{\itemsep}{1pt}
  \setlength{\parskip}{0pt}
  \setlength{\parsep}{0pt}
}{\end{enumerate}}


\newcommand{\vc}[1]{\ensuremath{\vcenter{\hbox{#1}}}}

\newcounter{excounter}
\setcounter{excounter}{1}
\newcommand\question[2]{\vskip 1em  \noindent\textbf{\arabic{excounter}\addtocounter{excounter}{1}:} \emph{#1} \noindent#2}
\newcommand\solution[1]{\vskip 0.5em \noindent\textbf{Solution:} #1}
%\newcommand\like{\par \noindent\emph{(This question is: good - bad - ugly? Difficulty: 0-10:\hskip 3em )}} 

\newcommand\lecturer[1]{\textbf{#1}}
\newcommand\hideforstudent[1]{#1}


%\renewcommand\lecturer[1]{{\color{white} \textbf{#1} }}
%\renewcommand\hideforstudent[1]{{\color{white}  }}
%\renewcommand\solution[1]{{\color{white} \vskip 0.5em \noindent\textbf{Solution:} #1 }}



\setlength{\parindent}{0cm}
\setlength{\parskip}{0.1cm}

\begin{document}

Fall  2016, MATH-566

\centerline{{\Large \textbf{Integer Programming - Solution \emph{Methods} - Branch and Bound}}}

Source:  \url{http://co-at-work.zib.de/files/Gurobi_MIP.pdf}

Problem:
\[
(IP)
\begin{cases}
\text{maximize} & \mathbf{c}^T\mathbf{x} \\
\text{subject to} & A\mathbf{x} \leq \mathbf{b},
\end{cases}
\]
where $\mathbf{c} \in \mathbb{Z}^n, \mathbf{b} \in \mathbb{Z}^m, A \in \mathbb{Z}^{m \times n}$, and $\mathbf{x} \in \mathbb{Z}^n$.

Suppose we try to relax the problem and solve it as a linear programming problem.
The set of feasible solutions is $P$. Suppose that the optimum is $\mathbf{x}^\star=(4.6,3.3)$. We know $x_2$ cannot be $3.3$. So we create two new instances, where we add constraints $x_2 \geq \lceil 3.3 \rceil$ and  $x_2 \leq \lfloor 3.3 \rfloor$. Variable $x_2$ is a\emph{branch variable}. We solve both instances and better of the solutions is the solution to the original problem. 

\begin{center}
\tikzset{insep/.style={inner sep=1pt, outer sep=0pt, circle, fill},}
\vc{
\begin{tikzpicture}[scale=0.8]
\draw[fill=gray!20] (0.5,0.5) -- (4.7,1.5) -- (4.6,3.3) coordinate(x){} -- (0.5,4.7) -- cycle;
\foreach \x in {0,1,...,5}{
\foreach \y in {0,1,...,5}{
\draw (\x,\y) node[insep](x\x\y){};
}
\draw[-latex](0,0) -- (5.5,0) node[below]{$x_1$};
\draw[-latex](0,0) -- (0,5.5) node[left]{$x_2$};
\draw(2.4,1.4) node{$P$};
\draw[line width=1pt, color=red] (x) --++(-50:1)  (x) --++(-50:-2)node[right]{$\mathbf{c}^T\mathbf{x}$};
\draw (x) node[above right]{$\mathbf{x^\star}=(4.6,3.3)$};
}
\end{tikzpicture}
}
\hskip 1em
$\Rightarrow$
\hskip 1em
\vc{
\begin{tikzpicture}[scale=0.8]
\begin{scope}
\clip (0,4) rectangle (5,6);
\draw[fill=gray!20] (0.5,0.5) -- (4.7,1.5) -- (4.6,3.3) coordinate(x){} -- (0.5,4.7) -- cycle;
\end{scope}
\begin{scope}
\clip (0,0) rectangle (5,3);
\draw[fill=gray!20] (0.5,0.5) -- (4.7,1.5) -- (4.6,3.3) coordinate(x){} -- (0.5,4.7) -- cycle;
\end{scope}
\foreach \x in {0,1,...,5}{
\foreach \y in {0,1,...,5}{
\draw (\x,\y) node[insep](x\x\y){};
}
\draw[-latex](0,0) -- (5.5,0) node[below]{$x_1$};
\draw[-latex](0,0) -- (0,5.5) node[left]{$x_2$};
\draw(2.4,1.4) node{$P_1$};
\draw(0.8,4.4) node{$P_2$};
\draw[line width=1pt, color=blue, dotted] (0,4) -- (5,4)node[right]{$x_2 \geq \lceil 3.3 \rceil$}  (0,3) -- (5,3) node[right]{$x_2 \leq \lfloor 3.3 \rfloor$};
}
\end{tikzpicture}
}
\end{center}

The same process repeats with $P_1$ and $P_2$. Result is a \emph{big} branch and bound tree $T$.
\begin{center}
\begin{tikzpicture}
\draw (0,0) node[ellipse,draw](P){$P$, $(4.6,3.3)$}
(P) ++(-45:2) node[circle,draw](P2){$P_2$} 
(P)  ++(45:-2) node[circle,draw](P1){$P_1$}
(P) -- node[right]{$x_2 \geq \lceil 3.3 \rceil$}  (P2)
(P) -- node[left]{$x_2 \leq \lfloor 3.3 \rfloor$} (P1)
;
\draw[dotted,line width=1pt]
(P1) -- ++(-45:1)
(P1) -- ++(45:-1)
(P2) -- ++(-45:1)
(P2) -- ++(45:-1)
;
\end{tikzpicture}
\end{center}

\textbf{Branch and (no Bound) outline}
\begin{enumerate}
\item Let $P = \{ \mathbf{x}: A\mathbf{x} \leq \mathbf{b}\}$
\item Build tree $T$ with one node $P$ (and mark it unexplored)
\item while $T$ has unexplored node $X$
\item \hskip 4em $\mathbf{x}^\star :=  $ optimum for LP relaxation of $X$; mark $X$ explored 
\item \hskip 4em If  $\mathbf{x}^\star_i  \not\in\mathbb{Z}$ for some $i$
\item \hskip 8em       $X_1 := X \cap \{ \mathbf{x}: \mathbf{x}_i \leq \lfloor  \mathbf{x}^\star_i \rfloor  \}$
\item \hskip 8em       $X_2 := X \cap \{ \mathbf{x}: \mathbf{x}_i \leq \lceil  \mathbf{x}^\star_i \rceil  \}$
\item \hskip 8em       Add $X_1$ and $X_2$ to $T$ as unexplored nodes
\item Return maximum of 	integer solutions in $T$.
\end{enumerate}


\question{}{
Consider problem 
\[
(IP)
\begin{cases} \text{maximize} & 100x_2 +  x_1\\
\text{subject to} &  (x_1,x_2) \in P,
\end{cases}
\]
where $P$ is depicted below.
Solve $(IP)$ using Brand and Bound. Create branch and bound tree $T$.

\begin{center}
\tikzset{insep/.style={inner sep=1pt, outer sep=0pt, circle, fill},}
\vc{
\begin{tikzpicture}[scale=1]
\draw[fill=gray!20] (0.6,0.5) -- (4.7,1.5) -- (4.6,2.9) -- (2.7,4) -- cycle;
\foreach \x in {0,1,...,5}{
\foreach \y in {0,1,...,4}{
\draw (\x,\y) node[insep](x\x\y){};
}
\draw (0,0) node[below left]{0,0};
\draw[-latex](0,0) -- (5.5,0) node[below]{$x_1$};
\draw[-latex](0,0) -- (0,4.5) node[left]{$x_2$};
\draw(2.4,1.4) node{$P$};
%\draw[line width=1pt, color=red] (x) --++(-50:1)  (x) --++(-50:-2)node[right]{$\mathbf{c}^T\mathbf{x}$};
%\draw (x) node[above right]{$\mathbf{x^\star}=(4.6,3.3)$};
}
\end{tikzpicture}
}
\end{center}
}

\hideforstudent{
\solution{
Here is the sequence of cuttings.
\begin{center}
\tikzset{insep/.style={inner sep=1pt, outer sep=0pt, circle, fill},}
\vc{
\begin{tikzpicture}[scale=1]
\draw[fill=gray!20] (0.6,0.5) -- (4.7,1.5) -- (4.6,2.9) -- (2.7,4) coordinate(x) -- cycle;
\foreach \x in {0,1,...,5}{
\foreach \y in {0,1,...,4}{
\draw (\x,\y) node[insep](x\x\y){};
}
\draw (0,0) node[below left]{0,0};
\draw[-latex](0,0) -- (5.5,0) node[below]{$x_1$};
\draw[-latex](0,0) -- (0,4.5) node[left]{$x_2$};
\draw(2.4,1.4) node{$P$};
\draw[line width=1pt, color=red] (x) --++(-1.8:2)  (x) --++(-1.8:-2)node[above]{$\mathbf{c}^T\mathbf{x}$};
\draw[line width=1pt, color=blue, dotted] (2,-0.3) -- (2,4.3) node[above]{$\mathbf{x}_1 \leq 2$};
\draw[line width=1pt, color=blue, dotted] (3,-0.3) node[below]{$\mathbf{x}_1 \geq 3$} -- (3,4.3);
\draw[color = blue]
(1.5,1.3) node {$P_1$}
(3.5,2.5) node {$P_2$}
;
%\draw (x) node[above right]{$\mathbf{x^\star}=(4.6,3.3)$};
}
\end{tikzpicture}
\hskip 4em
\begin{tikzpicture}[scale=1]
\begin{scope}
\clip (0,0) rectangle (2,4);
\draw[fill=blue!10] (0.6,0.5) -- (4.7,1.5) -- (4.6,2.9) -- (2.7,4)  -- cycle;
\end{scope}
\begin{scope}
\clip (3,0) rectangle (6,4);
\draw[fill=blue!10]  (0.6,0.5) -- (4.7,1.5) -- (4.6,2.9) -- (2.7,4)  -- cycle;
\end{scope}
\foreach \x in {0,1,...,5}{
\foreach \y in {0,1,...,4}{
\draw (\x,\y) node[insep](x\x\y){};
}
\draw (0,0) node[below left]{0,0};
\draw[-latex](0,0) -- (5.5,0) node[below]{$x_1$};
\draw[-latex](0,0) -- (0,4.5) node[left]{$x_2$};
\draw(2.4,1.4) node{$P$};
\draw (3,3.8) coordinate(x);
\draw[line width=1pt, color=red] (x) --++(-1.8:1)  (x) --++(-1.8:-1);
\draw (2,2.8) coordinate(x);
\draw[line width=1pt, color=red] (x) --++(-1.8:1)  (x) --++(-1.8:-1);
\draw[line width=1pt, color=blue, dotted] (2,-0.3) -- (2,4.3) node[above]{$\mathbf{x}_1 \leq 2$};
\draw[line width=1pt, color=blue, dotted] (3,-0.3) node[below]{$\mathbf{x}_1 \geq 3$} -- (3,4.6);
\draw[line width=1pt,dotted,color = orange]
(-0.3,2) node[left]{$\mathbf{x}_2 \leq 2$}  -- (2,2) 
(-0.3,3) node[left]{$\mathbf{x}_2 \geq 3$}  -- (2,3) 
(1.5,1.3) node {$P_{1,1}$}
(1,3.5) node {$P_{1,2}=\emptyset$}
;
\draw[line width=1pt, color=green!70!black, dotted] 
(3,3) -- (5.3,3) node[right]{$\mathbf{x}_2 \leq 3$}
(3,4) -- (5.3,4) node[right]{$\mathbf{x}_2 \geq 4$}
(3.5,2.5) node {$P_{2,1}$}
(4,4.5) node {$P_{2,2} = \emptyset$}
;
}
\end{tikzpicture}
}
\vskip 2em
\vc{
\begin{tikzpicture}[scale=1]
\begin{scope}
\clip (0,0) rectangle (2,2);
\draw[fill=orange!10]  (0.6,0.5) -- (4.7,1.5) -- (4.6,2.9) -- (2.7,4)  -- cycle;
\end{scope}
\begin{scope}
\clip (3,0) rectangle (6,3);
\draw[fill=green!10]  (0.6,0.5) -- (4.7,1.5) -- (4.6,2.9) -- (2.7,4)  -- cycle;
\end{scope}
\foreach \x in {0,1,...,5}{
\foreach \y in {0,1,...,4}{
\draw (\x,\y) node[insep](x\x\y){};
}
\draw (0,0) node[below left]{0,0};
\draw[-latex](0,0) -- (5.5,0) node[right]{$x_1$};
\draw[-latex](0,0) -- (0,4.5) node[left]{$x_2$};
\draw(2.4,1.4) node{$P$};
\draw (4.5,3) coordinate(x);
\draw[line width=1pt, color=red] (x) --++(-1.8:1)  (x) --++(-1.8:-1);
\draw (2,2) coordinate(x);
\draw[line width=1pt, color=red] (x) --++(-1.8:1)  (x) --++(-1.8:-1);
\draw[line width=1pt, color=blue, dotted] (2,-0.3) -- (2,4.3) node[above]{$\mathbf{x}_1 \leq 2$};
\draw[line width=1pt, color=blue, dotted] (3,-0.3) node[below]{$\mathbf{x}_1 \geq 3$} -- (3,4.6);
\draw[line width=1pt,dotted,color = orange]
(-0.3,2) node[left]{$\mathbf{x}_2 \leq 2$}  -- (2,2) 
(1.5,1.3) node {$P_{1,1}$}
;
\draw[line width=1pt, color=green!70!black, dotted] 
(3,3) -- (5.5,3) node[right]{$\mathbf{x}_2 \leq 3$}
(4.3,1.7) node {$P_{2,1}$}
;
\draw[line width=1pt, color=pink!60!black, dotted] 
(4,0) -- (4,4.2) node[above]{$\mathbf{x}_1 \leq 4$}
(5,0) node[below]{$\mathbf{x}_1 \geq 5$} -- (5,4.2) 
(3.5,2.5) node {$P_{2,1,1}$}
(5.8,2.5) node {$P_{2,1,2}=\emptyset$}
;
}
\end{tikzpicture}
\hskip3em
\begin{tikzpicture}[scale=1]
\begin{scope}
\clip (0,0) rectangle (2,2);
\draw[fill=orange!10]  (0.6,0.5) -- (4.7,1.5) -- (4.6,2.9) -- (2.7,4)  -- cycle;
\end{scope}
\begin{scope}
\clip (3,0) rectangle (4,3);
\draw[fill=pink!20]  (0.6,0.5) -- (4.7,1.5) -- (4.6,2.9) -- (2.7,4)  -- cycle;
\end{scope}

\foreach \x in {0,1,...,5}{
\foreach \y in {0,1,...,4}{
\draw (\x,\y) node[insep](x\x\y){};
}
\draw (0,0) node[below left]{0,0};
\draw[-latex](0,0) -- (5.5,0) node[right]{$x_1$};
\draw[-latex](0,0) -- (0,4.5) node[left]{$x_2$};
\draw(2.4,1.4) node{$P$};
\draw (4,3) coordinate(x);
\draw[line width=1pt, color=red] (x) --++(-1.8:1)  (x) --++(-1.8:-1);
\draw (2,2) coordinate(x);
\draw[line width=1pt, color=red] (x) --++(-1.8:1)  (x) --++(-1.8:-1);
\draw[line width=1pt, color=blue, dotted] (2,-0.3) -- (2,4.3) node[above]{$\mathbf{x}_1 \leq 2$};
\draw[line width=1pt, color=blue, dotted] (3,-0.3) node[below]{$\mathbf{x}_1 \geq 3$} -- (3,4.6);
\draw[line width=1pt,dotted,color = orange]
(-0.3,2) node[left]{$\mathbf{x}_2 \leq 2$}  -- (2,2) 
(1.5,1.3) node {$P_{1,1}$}
;
\draw[line width=1pt, color=green!70!black, dotted] 
(3,3) -- (5.5,3) node[right]{$\mathbf{x}_2 \leq 3$}
;
\draw[line width=1pt, color=pink!60!black, dotted] 
(4,0) -- (4,4.2) node[above]{$\mathbf{x}_1 \leq 4$}
(3.5,2.5) node {$P_{2,1,1}$}
;
}
\draw
(2,2) node[rectangle,fill=yellow,inner sep=2pt,draw]{}
(4,3) node[rectangle,fill=yellow,inner sep=2pt,draw]{}
;
\end{tikzpicture}
}
\end{center}
And this is the resulting tree $T$.
\begin{center}
\begin{tikzpicture} [
every node/.style={circle},
level distance=25mm,
level 1/.style={sibling distance=60mm}, 
level 2/.style={sibling distance=30mm}, 
level 3/.style={sibling distance=30mm}]
\draw
  node[draw,circle split,fill=gray!20] {$P$\nodepart{lower} $(3.7,4)$ }
     child[label=left:$a$] {node[circle split,draw,fill=blue!20] {$P_1$\nodepart{lower} $(2,2.9)$}
          child {node[draw,circle split,fill=orange!20] {$P_{1,1}$\nodepart{lower} $(2,2)$ }  edge from parent node[left] {$x_2 \leq 2$}   }
          child { node[draw,circle split] {$P_{1,2}$\nodepart{lower} $\emptyset$}  edge from parent node[right] {$x_2 \geq 3$}  }
          edge from parent node[above left] {$x_1 \leq 2$}
     }
     child {node[draw,fill=blue!20,circle split] {$P_2$\nodepart{lower} $(3,3.8)$}
          child {node[draw,fill=green!20,circle split] {$P_{2,1}$\nodepart{lower} $(4.5,3)$}  
               child {node[draw,fill=pink!20,circle split] {$P_{2,1,1}$\nodepart{lower} $(4,3)$}   edge from parent node[left] {$x_1 \leq 5$} }
               child {node[draw,circle split] {$P_{2,1,2}$\nodepart{lower} $\emptyset$}             edge from parent node[right] {$x_1 \geq 5$} }               
               edge from parent node[left] {$x_2 \leq 3$} }
          child {node[draw,circle split] {$P_{2,2}$\nodepart{lower} $\emptyset$}  edge from parent node[right] {$x_2 \geq 4$}  }
          edge from parent node[above right] {$x_1 \geq 3$}
     }
;
\end{tikzpicture}
\end{center}
Notice that the leaves have either integer solution or are empty.
Also notice that there is more than just one branching on  $x_1$. 
And the optimum solution is $(4,3)$, value 304.
}
}

\vfill
\vfill


\question{}{
Will branch and bound ALWAYS find an optimal solution if one exists?
}
\solution{
Yes, this EVENTUALLY gets the right answer.
}

\question{}{
Is there a \emph{good} bound on the size of the tree?
}
\solution{
No - the tree may explode. It may have exponential size.
}

\question{}{
Is it possible to identify nodes in $T$ that will not contain the optimal solution?
}
\solution{
Sometimes. See the example above. Consider we computed node $P_{2,1,1}$ and get an integer solution
of value $304$. This tells us that the optimum integral solution has value at least 304.
Now we look at node $P_1$ - it gives solution with value $292$. In the whole subtree
under $P_1$, all integer solutions in the subtree rooted at $P_1$ will have value at most $292$.
Hence no need to solve under $P_1$. \textbf{That is why the method is branch and bound}
Note: good idea to try to round and get some integers solutions - helps cut the tree.
This is the bound part of the name.
}

\question{}{
What are (dis)advantages of processing nodes deep in the search tree vs nodes close to the root?
}
\solution{
Deep is more likely to give integer solution. But more likely to be eliminated later by some better solution. No clear winner.
}


\question{}{
Which if a solution in a node has more non-integer coordinates, which variable to branch on first?
}
\solution{
Depends on problem - branch on important first. Example - decide if building factory at all before deciding how many production lines it should have.
}

\vfill

\emph{Next time: Cutting Planes for Integer Programming.}

\end{document}






