\documentclass[11pt]{article}
\usepackage[utf8]{inputenc}
\usepackage[czech, english]{babel}
\usepackage[left=2cm,right=2cm,top=2cm,bottom=2cm]{geometry}
\usepackage{amssymb}
\usepackage{amsthm}
\usepackage{amsmath}
\usepackage{tikz}


% Itemize environment with small skips
\newenvironment{packeditemize}{
\begin{itemize}
  \setlength{\itemsep}{1pt}
  \setlength{\parskip}{0pt}
  \setlength{\parsep}{0pt}
}{\end{itemize}}


% Fancy footnote....
\usepackage{fancyhdr}
\pagestyle{fancy}
\usepackage{lastpage}
\rfoot{MATH 566 - 07, page \thepage/\pageref{LastPage}}
\cfoot{}
\rhead{}
\lhead{}
\renewcommand{\headrulewidth}{0pt}
\renewcommand{\footrulewidth}{0pt}


\newtheorem{theorem*}{Theorem} 

% Itemize environment with small skips
\newenvironment{packedenumerate}{
\begin{enumerate}
  \setlength{\itemsep}{1pt}
  \setlength{\parskip}{0pt}
  \setlength{\parsep}{0pt}
}{\end{enumerate}}


\newcounter{excounter}
\setcounter{excounter}{1}
\newcommand\question[2]{\vskip 1em  \noindent\textbf{\arabic{excounter}\addtocounter{excounter}{1}:} \emph{#1} \noindent#2}
\newcommand\solution[1]{\vskip 0.5em \noindent\textbf{Solution:} #1}
\newcommand\like{\par \noindent\emph{(This question is: good - bad - ugly? Difficulty: 0-10:\hskip 3em )}} 

\newcommand\lecturer[1]{\textbf{#1}}
\newcommand\hideforstudent[1]{#1}


\renewcommand\lecturer[1]{{\color{white} \textbf{#1} }}
%\renewcommand\hideforstudent[1]{{\color{white} #1 }}
%\renewcommand\solution[1]{{\color{white} #1 }}



\setlength{\parindent}{0cm}
\setlength{\parskip}{0.1cm}

\begin{document}

Fall  2016, MATH-566

\centerline{{\Large \textbf{Consequences of duality}}}

\textbf{Complementary slackness:}
Let
\[ (P)  \text{ min } \mathbf{c}^T\mathbf{x} \text{ s.t. } A\mathbf{x} = \mathbf{b}, \mathbf{x} \geq 0 \]
\[ (D)  \text{ max } \mathbf{b}^T\mathbf{y} \text{ s.t. } A^T\mathbf{y} \leq \mathbf{c} \]

\textbf{Theorem}
Let $\mathbf{x}$ and $\mathbf{y}$ be feasible for $(P)$ and $(D)$. Vectors $\mathbf{x}$ and $\mathbf{y}$
are optimal solutions iff
\begin{packeditemize}
\item $x_i > 0 \Rightarrow \mathbf{y}^T\mathbf{a}_i = c_i$
\item $x_i = 0 \Leftarrow \mathbf{y}^T\mathbf{a}_i < c_i$
\end{packeditemize}
where $\mathbf{a}_i$ is $i$th column of $A$.
\lecturer{Draw the arrow between (P) and (D).}

\question{}{Prove the theorem. Consider $(\mathbf{y}^TA - \mathbf{c}^T)\mathbf{x} = ?$}
\solution{
conditions $\Rightarrow$ optimality:
\begin{align*}
\left(\mathbf{y}^TA - \mathbf{c}^T\right)\mathbf{x} &= 0 \\
\mathbf{y}^TA\mathbf{x} - \mathbf{c}^T\mathbf{x} &= 0 \\
\mathbf{y}^T\mathbf{b} - \mathbf{c}^T\mathbf{x} &= 0 \\
\mathbf{y}^T\mathbf{b} &= \mathbf{c}^T\mathbf{x}
\end{align*}
conditions $\Leftarrow$ optimality:
Reverse the computation and observe that $\left(\mathbf{y}^TA - \mathbf{c}^T\right) \leq 0$
and $\mathbf{x} \geq 0$. Hence every coordinate must be $0$, which are the constraints.
}


Example: Diet problem.
Suppose in optimal solution $\mathbf{a_i}\mathbf{x} >  b_i$. The $y_j = 0$. In optimal solution we get more than we need, so the cost in the dual is zero.




\textbf{Geometric interpretation} and solutions with complementary slackness

\begin{align*}
(P) \begin{cases}
\text{min} & 18x_1 + 12x_2 + 2x_3 + 6x_4\\
\text{s.t}   &   3x_1 + x_2     - 2x_3 + x_4 = 2 \\
                &     x_1 + 3x_2 + 0 x_3 - x_4  = 2  \\
                & x_1,x_2,x_3,x_4 \geq 0
\end{cases} 
\end{align*}
\question{}{
Write the dual and solve it using graphical method (draw half-spaces).
The reconstruct the solution of $(P)$.
}
\solution{
\begin{align*}
(D) \begin{cases}
\text{max} & 2y_1 + 2y_2 \\
\text{s.t}   &   3y_1 + y_2 \leq 18 \\
                &    y_1 + 3y_2   \leq 12  \\
                & -2y_1 \leq 2 \\
                & y_1 - y_2 \leq 6 
\end{cases} 
\end{align*}

%Model dual
%  Variables
%    y[1:2] = 0
%  End Variables
%  Equations
%    3*y[1] + y[2] <= 18
%    y[1] + 3*y[2] <= 12
%  -2*y[1] <= 2
%   y[1] - y[2] <= 6
%     maximize 2*y[1] + 2*y[2]
%  End Equations
%End Model

\begin{center}
\begin{tikzpicture}[scale=0.7]
\clip (-2,-1) rectangle (8,5);
\draw[-latex] (-2,0)  -- (8,0);
\draw[-latex] (0,-2) -- (0,5);
\draw[line width=1pt, -latex] (0,0) -- (2,2);
\draw[domain=-2:7,smooth,variable=\x] plot ({\x},{-3*\x+18});
\draw[domain=-2:8,smooth,variable=\x] plot ({\x},{-\x/3+4});
\draw[domain=-2:8,smooth,variable=\x] plot ({\x},{\x-6});
\draw[domain=-5:8,smooth,variable=\x] plot ({-1},{\x});
\fill (5.25,2.25) circle (4pt) ++(30:1.5) node{$[5.25,2.25]$};
\end{tikzpicture}
\end{center}
Solution will be intersection of $ 3y_1 + y_2 = 18$ and $ y_1 + 3y_2 = 12$, which
gives $y_1 = 5.25, y_2 = 2.25$ and objective value is $15$.

Complementary slackness implies that $x_3 = x_4 = 0$.
This gives us
\begin{align*}
(P') \begin{cases}
\text{min} & 18x_1 + 12x_2\\
\text{s.t}   &   3x_1 + x_2     = 2 \\
                &     x_1 + 3x_2   = 2  \\
                & x_1,x_2  \geq 0
\end{cases} 
\end{align*}
Notice that original matrix 
\[
A = 
\begin{pmatrix}
  3  & 1 & -2 & 1 \\
   1 & 3 &  0  &  -1
\end{pmatrix}
\]
can be written as $A = (B|\text{ trash})$ and the solution can be computed by solving $B^{-1}\mathbf{b}$.
The solution is $x_1 = x_2 = \frac{1}{2}$. The objective value is also 15.
}

\textbf{Sensitivity}
Let
\[ (P)  \text{ min } \mathbf{c}^T\mathbf{x} \text{ s.t. } A\mathbf{x} = \mathbf{b}, \mathbf{x} \geq 0 \]
\[ (D)  \text{ max } \mathbf{b}^T\mathbf{y} \text{ s.t. } A^T\mathbf{y} \leq \mathbf{c} \]

How does the solution change if $\mathbf{b}$ changes? 

Which of the constraints are important and which are not?

Consider optimal solution $\mathbf{x}^\star = (\mathbf{x}_B,0)$. Then
$A = (B|\text{trash})$.
Submatrix $B$ is called the \emph{base} of the solution. 
Note $\mathbf{x}_B = {B}^{-1}\mathbf{b}$.

\[
\mathbf{c}^T\mathbf{x}^\star = \mathbf{c}^T_B\mathbf{x}_B = \mathbf{c}^T_BB^{-1}\mathbf{b}
\]
Hence $(\mathbf{y}^\star)^T = \mathbf{c}^T_BB^{-1}$.

Suppose $\mathbf{b} \rightarrow (\mathbf{b} + \Delta \mathbf{b})$.
If $\Delta\mathbf{b}$ small, base $B$ is still the same. (see example)
Then the new optimal solution is
\[
B^{-1}(\mathbf{b} + \Delta \mathbf{b}) = \mathbf{x}_B + \Delta \mathbf{x}_B
\]

\question{}{
What will be the change of the value of the objective function? (denoted by $\Delta z$) 
}
\solution{
\[
\Delta z = \mathbf{c}^T_B \cdot \Delta \mathbf{x} =  \mathbf{c}^T_B \cdot  B^{-1} (\Delta \mathbf{b}) 
=  (\mathbf{y}^\star)^T (\Delta \mathbf{b}) 
\]
So if $\mathbf{b}$ is changed by $\Delta \mathbf{b}$, then the value of objective function 
is changed by $(\mathbf{y}^\star)^T (\Delta \mathbf{b})$.
}

$\mathbf{y}^\star$ gives sensitivity of the solution.

\newpage

Let 
\[ (P)  \text{ min } \mathbf{c}^T\mathbf{x} \text{ s.t. } A\mathbf{x} \geq \mathbf{b}, \mathbf{x} \geq 0 \]
\[ (D)  \text{ max } \mathbf{b}^T\mathbf{y} \text{ s.t. } A^T\mathbf{y} \leq \mathbf{c}, \mathbf{y} \geq 0 \]
Then complementary slackness gives
\begin{itemize}
\item $x_i > 0 \Rightarrow \mathbf{y}^T\mathbf{a}_i = c_i$ 
\item $x_i = 0 \Leftarrow \mathbf{y}^T\mathbf{a}_i < c_i$ 
\item $y_i > 0 \Rightarrow \mathbf{a}^i\mathbf{x} = b_i$
\item $y_i = 0 \Leftarrow \mathbf{a}^i\mathbf{x} > b_i$,
\end{itemize}
where $\mathbf`{a}^i$ is $i$th row of $A$ and where $\mathbf{a}_i$ is $i$th column of $A$.



\end{document}



