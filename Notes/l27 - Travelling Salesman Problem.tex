\documentclass[11pt]{article}
\usepackage[utf8]{inputenc}
\usepackage[czech, english]{babel}
\usepackage[left=2cm,right=2cm,top=2cm,bottom=2cm]{geometry}
\usepackage{amssymb}
\usepackage{amsthm}
\usepackage{amsmath}
\usepackage{mathtools}
\usepackage{tikz}
\usetikzlibrary{shapes.geometric}
\usetikzlibrary{shapes.multipart}
\usetikzlibrary{arrows}
\usetikzlibrary{graphs}
\usepackage{hyperref}


% Itemize environment with small skips
\newenvironment{packeditemize}{
\begin{itemize}
  \setlength{\itemsep}{1pt}
  \setlength{\parskip}{0pt}
  \setlength{\parsep}{0pt}
}{\end{itemize}}


% Fancy footnote....
\usepackage{fancyhdr}
\pagestyle{fancy}
\usepackage{lastpage}
\rfoot{MATH 566 - 27, page \thepage/\pageref{LastPage}}
\cfoot{}
\rhead{}
\lhead{}
\renewcommand{\headrulewidth}{0pt}
\renewcommand{\footrulewidth}{0pt}


\newtheorem{theorem*}{Theorem} 

% Itemize environment with small skips
\newenvironment{packedenumerate}{
\begin{enumerate}
  \setlength{\itemsep}{1pt}
  \setlength{\parskip}{0pt}
  \setlength{\parsep}{0pt}
}{\end{enumerate}}


\newcommand{\vc}[1]{\ensuremath{\vcenter{\hbox{#1}}}}

\newcounter{excounter}
\setcounter{excounter}{1}
\newcommand\question[2]{\vskip 1em  \noindent\textbf{\arabic{excounter}\addtocounter{excounter}{1}:} \emph{#1} \noindent#2}
\newcommand\solution[1]{\vskip 0.5em \noindent\textbf{Solution:} #1}
%\newcommand\like{\par \noindent\emph{(This question is: good - bad - ugly? Difficulty: 0-10:\hskip 3em )}} 

\newcommand\lecturer[1]{\textbf{#1}}
\newcommand\hideforstudent[1]{#1}


\renewcommand\lecturer[1]{{\color{white} \textbf{#1} }}
\renewcommand\hideforstudent[1]{{\color{white}  }}
\renewcommand\solution[1]{{\color{white} \vskip 0.5em \noindent\textbf{Solution:} #1 }}



\setlength{\parindent}{0cm}
\setlength{\parskip}{0.1cm}

\begin{document}

Fall  2016, MATH-566

\centerline{{\Large \textbf{Traveling Salesman Problem}}}


Let $G=(V,E)$ be a complete graph on $n$ vertices. Let $c: E \rightarrow \mathbb{R}^+$.
Find a closed cycle/circuit $C$ through all vertices of minimum cost.
\[
\min\left\{\sum_{e\in C} c(e): C\text{ is a circuit of all vertices on } G  \right\}
\]
The problem is NP-complete. We will try for vertices on the plane (triangle inequality satisfied and distance is positive)

Heuristics
\begin{itemize}
\item \emph{Nearest neighbor:} Build a path, always include the nearest neighbor.
On test data gives 1.26 of optimum. 

\question{}{
Show that the nearest neighbor algorithm can do arbitrarily bad it no triangle-inequality
}
\solution{
Four vertices are enough. One could lead the shortest path to a trap.
}

Worst case if triangle-inequality is satisfied $\frac{1}{3}(\log_2(n+1)+\frac{4}{9})$ times optimum.

\item \emph{Cheapest insertion:} Start with an edge and keep adding vertices one by one that are cheapest to insert. At most $2$ times optimum.

\item \emph{Furthest insertion:} Start with longest edge and keep adding vertices one by one that are furthest away.  At most $\log_2 n+1$ times optimum. Better in experiments than previous.

note: No instance is known, where insertion method would do worse than 4 times the optimum.

\item \emph{Christofides Heuristics:} Start with Minimum Spanning Tree. Add Minimum Matching to vertices of odd degree. Vertices of degree at least 4 can split off. Does at most $\frac{3}{2}$ of optimum.

\question{}{
Show that the upper bound of the algorithm is at most $\frac{3}{2}$ of optimum.
}
\solution{
Tree has cost $\leq$ cost of minimum spanning tree. 
Matching can also have cost less than half of the TSP if we walk along the TSP.
Splitoff does not increase the cost.
}

\end{itemize}

Tour improvements
\begin{itemize}
\item \emph{2-optimal switch:} Replace 2 edges by different 2 edges. (more generally, $k$-switch)
\item \emph{Lin-Kernighan:} Better way of doing 2-switches.
\end{itemize}

Lower Bounds
\begin{itemize}
\item \emph{Held-Karp:} Find a vertex  $v$ and minimum spanning tree $T$ in $G-v$, then add $v$ to $T$ by using to smallest cost edges adjacent to $v$. Modify cost of edges/vertices and rerun. Try to make costs such that every vertex is in exactly two edges.

\begin{center}
\begin{tikzpicture}[scale=1.2]
\tikzset{insep/.style={inner sep=2pt, outer sep=0pt, circle, fill},}
\draw  
(0,0) node[insep](a){}   -- node[pos=0.5,label=left:$0$]{ }  (0,2) node[insep](b){}
(1,1) node[insep](c){}   -- node[pos=0.5,label=below:$ $]{0}
(2,1) node[insep](d){}   -- node[pos=0.5,label=below:$ $]{1}
(4,1) node[insep](e){}  
(a) -- node[pos=0.5,label=below:$ $]{0}  (c)
(b) -- node[pos=0.5,label=below:$$]{0}  (c)
(a) -- node[pos=0.5,label=below:$0$]{}  (e)
(b) -- node[pos=0.5,label=above:$0$]{}  (e)
(a) -- node[pos=0.5,label=below:$ $]{10}  (d)
(b) -- node[pos=0.5,label=above:$ $]{10}  (d)
;
\end{tikzpicture}
\end{center}

\item \emph{Linear-programming:} Can be used to provide a relaxation of integer programing version.\\
(can be modified to match Held-Karp)
\end{itemize}

\end{document}






