\documentclass[11pt]{article}
\usepackage[utf8]{inputenc}
\usepackage[czech, english]{babel}
\usepackage[left=2cm,right=2cm,top=2cm,bottom=2cm]{geometry}
\usepackage{amssymb}
\usepackage{amsthm}
\usepackage{amsmath}
\usepackage{tikz}
\usetikzlibrary{shapes.geometric}
\usepackage{hyperref}


% Itemize environment with small skips
\newenvironment{packeditemize}{
\begin{itemize}
  \setlength{\itemsep}{1pt}
  \setlength{\parskip}{0pt}
  \setlength{\parsep}{0pt}
}{\end{itemize}}


% Fancy footnote....
\usepackage{fancyhdr}
\pagestyle{fancy}
\usepackage{lastpage}
\rfoot{MATH 566 - 15, page \thepage/\pageref{LastPage}}
\cfoot{}
\rhead{}
\lhead{}
\renewcommand{\headrulewidth}{0pt}
\renewcommand{\footrulewidth}{0pt}


\newtheorem{theorem*}{Theorem} 

% Itemize environment with small skips
\newenvironment{packedenumerate}{
\begin{enumerate}
  \setlength{\itemsep}{1pt}
  \setlength{\parskip}{0pt}
  \setlength{\parsep}{0pt}
}{\end{enumerate}}


\newcounter{excounter}
\setcounter{excounter}{1}
\newcommand\question[2]{\vskip 1em  \noindent\textbf{\arabic{excounter}\addtocounter{excounter}{1}:} \emph{#1} \noindent#2}
\newcommand\solution[1]{\vskip 0.5em \noindent\textbf{Solution:} #1}
%\newcommand\like{\par \noindent\emph{(This question is: good - bad - ugly? Difficulty: 0-10:\hskip 3em )}} 

\newcommand\lecturer[1]{\textbf{#1}}
\newcommand\hideforstudent[1]{#1}


%\renewcommand\lecturer[1]{{\color{white} \textbf{#1} }}
%\renewcommand\hideforstudent[1]{{\color{white} #1 }}
%\renewcommand\solution[1]{{\color{white} \vskip 0.5em \noindent\textbf{Solution:} #1 }}



\setlength{\parindent}{0cm}
\setlength{\parskip}{0.1cm}

\begin{document}

Fall  2016, MATH-566

\centerline{{\Large \textbf{Network flows - First Algorithm}}}


\textbf{Ford-Fulkerson Algorithm}\\
\emph{Input:} Network $(G,u,s,t)$.\\
\emph{Output:} and $s$-$t$-flow $f$ of maximum value
\begin{enumerate}
\item $f(e) = 0$ for all $e \in E(G)$
\item while $f$-augmenting path $P$ exists:
\item \hskip 4em compute $\gamma := \min_{e \in E(P)} u_f(e)$
\item \hskip 4em augment $f$ along $P$ by $\gamma$ (as much as possible)
\end{enumerate}

\question{}{
How many iterations (at most) the algorithm needs at the following network:
\begin{center}
\begin{tikzpicture}[scale=1.2]
\tikzset{insep/.style={inner sep=2pt, outer sep=0pt, circle, fill},}
\draw [-latex] (0,0) node[insep,label=left:$s$](s){} -- ++(40:1.5) node[insep,label=above:$u$](u){} node[pos=0.5,above]{$N$} ;
\draw [-latex] (s) -- ++(-40:1.5) node[insep,label=below:$v$](v){} node[pos=0.5,below]{$N$} ;
\draw [-latex] (v) -- ++(40:1.5) node[insep,label=right:$t$](t){} node[pos=0.5,below]{$N$} ;
\draw [-latex] (u) -- (v) node[pos=0.5,right]{$1$} ;
\draw [-latex] (u) -- (t) node[pos=0.5,above]{$N$} ;
\end{tikzpicture}
\end{center}
($N$ is a big integer. Try to trick the algorithm to do many steps by picking unlucky $P$.)
}
\solution{
Always use the edge $uv$ with capacity 1. The augmentation will be always by one. So $2N$ iterations.
The running time depends on $u$.
}

\question{}{
Show that $s$-$t$-flow $f$ is maximum if and only if there is no $f$-augmenting path. (That is, Ford-Fulkerson algorithm is correct.)
}
\solution{
If there is an augmenting path, then the flow can increase. If there is no augmenting path, then from $s$ reachable vertices in reduced graph from  a cut. Recall 
\[
\text{value}(f) =  \sum_{e \in \delta^+(A)} f(e) - \sum_{e \in \delta^-(A)} f(e).
\]
Since there is no augmenting path, 
\begin{align*}
\sum_{e \in \delta^-(A)} f(e) &= 0   &   \sum_{e \in \delta^+(A)} f(e) &= \sum_{e \in \delta^+(A)} f(e).
\end{align*}
This proves the maximality of the flow.
}

The question gives proof to\\
\textbf{Theorem} (Ford Fulkerson)
Maximum value of an $s$-$t$-flow equals minimum capacity of an $s$-$t$-cut.

\question{}{
If $c:E \rightarrow \mathbb{Z}$, is it true that the flow produced from Ford-Fulekrson is integral and that the algorithm finishes in a finite time?
}

\solution{
Yes, the augmenting is always by an integer. Every augmentation raises the value of the flow by at least one. This gives finiteness.
}

This proves\\
\textbf{Theorem} Dantzig Fulkerson: If the capacities are integral, then there exists an integral maximum flow.

\question{}{
If capacities are integral, is it true that every maximum flow is integral?
}
\solution{
No - build network that has small cut and 2 paths to/or from it.
}


\textbf{Theorem} Gallai; Ford and Fulerson\\
Every flow $f$ can be decomposed into $s$-$t$-paths $\mathcal{P}$ and circuits $\mathcal{C}$
with weight function $w: \mathcal{P} \cup \mathcal{C} \rightarrow \mathbb{R}_+$ such that
\begin{packeditemize}
\item $f(e) = \sum_{e\in P\in\mathcal{P}} w(P) + \sum_{e\in C\in\mathcal{C}} w(C)$
\item $value(f) = \sum_{P\in\mathcal{P}} w(P)$.
\item $|\mathcal{P}+\mathcal{C}| \leq |E(G)|$.
\end{packeditemize}
\question{}{
Prove the theorem. Hint, use induction on number of edges $e$, where $f(e) > 0$.
}
\solution{
Take the longest circuit or $s$-$t$-path  $W$. 
Assign weight $w$ to  $W$ such that $f(h) = w(h)$ for some edge $h$ and for
all other edges $f(h) \geq w(h)$. Put $W$ to the collection and apply induction.

}

\centerline{{\Large \textbf{Network flows - Menger's theorem}}}

\textbf{Theorem} Menger:
Let $G$ be a graph (directed or undirected), let $s,t \in V(G)$, and $k \in \mathbb{N}$. 
Then there are $k$ edge-disjoint $s$-$t$-paths iff after deleting any $k-1$ edges $t$ is still reachable from $s$.
\question{}{Use flows to prove the theorem in directed case.}

\solution{
Assign capacities one to all edges, find maximum flow and use the previous decomposition theorem. Also notice that the maximum flow is integral.
\vskip 4em
}


\question{}{Use the directed case to prove undirected case.  Replacing an edge by a pair of opposite
directed edges does not work. (Why?) Replace edge $uv$ by a suitable orientation of the following gadget
}
\begin{center}
\tikzset{insep/.style={inner sep=2pt, outer sep=0pt, circle, fill},}
\begin{tikzpicture}
\draw (0,0) node[insep,label=left:$u$](u){}  -- ++(0:1.5) node[insep,label=right:$v$](v){} 
(4,0) node[insep,label=left:$u$](u){}  -- ++(45:1) node[insep](x){} (u) -- ++(-45:1) node[insep](y){} 
(x)--(y) (x) -- ++(-45:1) node[insep,label=right:$v$](v){} -- (y);
\end{tikzpicture}
\end{center}
\solution{
\begin{center}
\tikzset{insep/.style={inner sep=2pt, outer sep=0pt, circle, fill},}
\begin{tikzpicture}
\draw (4,0) node[insep,label=left:$u$](u){}  -- ++(45:1) node[insep](x){} (u) -- ++(-45:1) node[insep](y){} 
(x)--(y) (x) -- ++(-45:1) node[insep,label=right:$v$](v){} -- (y);
\draw[-latex] (u)--(x); \draw[-latex] (v)--(x);
\draw[-latex] (y)--(u); \draw[-latex] (y)--(v);
\draw[-latex] (x)--(y);
\end{tikzpicture}
\end{center}
}

Paths $P_1$ and $P_2$ are called \emph{internally disjoint} if they do not share more than the endvertices.

\textbf{Theorem} Menger:  Let $G$ be a graph (directed or undirected), let $s$ and $t$ be two non-adjacent vertices, and $k\in\mathbb{N}$. Then there are $k$ pairwise internally disjoint $s$-$t$-paths iff after deleting any $k-1$ vertices (distinct from $s$ and $t$) $t$ is still reachable from $s$.

\question{}{
Prove the directed case. Use the edge case.
}

\end{document}


\end{document}






