\documentclass[11pt]{article}
\usepackage[utf8]{inputenc}
\usepackage[czech, english]{babel}
\usepackage[left=2cm,right=2cm,top=2cm,bottom=2cm]{geometry}
\usepackage{amssymb}
\usepackage{amsthm}
\usepackage{amsmath}
\usepackage{mathtools}
\usepackage{tikz}
\usetikzlibrary{shapes.geometric}
\usetikzlibrary{shapes.multipart}
\usetikzlibrary{arrows}
\usetikzlibrary{graphs}
\usepackage{hyperref}


% Itemize environment with small skips
\newenvironment{packeditemize}{
\begin{itemize}
  \setlength{\itemsep}{1pt}
  \setlength{\parskip}{0pt}
  \setlength{\parsep}{0pt}
}{\end{itemize}}


% Fancy footnote....
\usepackage{fancyhdr}
\pagestyle{fancy}
\usepackage{lastpage}
\rfoot{MATH 566 - 23, page \thepage/\pageref{LastPage}}
\cfoot{}
\rhead{}
\lhead{}
\renewcommand{\headrulewidth}{0pt}
\renewcommand{\footrulewidth}{0pt}


\newtheorem{theorem*}{Theorem} 

% Itemize environment with small skips
\newenvironment{packedenumerate}{
\begin{enumerate}
  \setlength{\itemsep}{1pt}
  \setlength{\parskip}{0pt}
  \setlength{\parsep}{0pt}
}{\end{enumerate}}


\newcommand{\vc}[1]{\ensuremath{\vcenter{\hbox{#1}}}}

\newcounter{excounter}
\setcounter{excounter}{1}
\newcommand\question[2]{\vskip 1em  \noindent\textbf{\arabic{excounter}\addtocounter{excounter}{1}:} \emph{#1} \noindent#2}
\newcommand\solution[1]{\vskip 0.5em \noindent\textbf{Solution:} #1}
%\newcommand\like{\par \noindent\emph{(This question is: good - bad - ugly? Difficulty: 0-10:\hskip 3em )}} 

\newcommand\lecturer[1]{\textbf{#1}}
\newcommand\hideforstudent[1]{#1}


%\renewcommand\lecturer[1]{{\color{white} \textbf{#1} }}
%\renewcommand\hideforstudent[1]{{\color{white}  }}
%\renewcommand\solution[1]{{\color{white} \vskip 0.5em \noindent\textbf{Solution:} #1 }}



\setlength{\parindent}{0cm}
\setlength{\parskip}{0.1cm}

\begin{document}

Fall  2016, MATH-566

\centerline{{\Large \textbf{Integer Programming - Solution \emph{Methods} - Cutting Planes}}}

Source: Bill,Bill,Bill; \\
\url{http://cgm.cs.mcgill.ca/~avis/courses/567/notes/cutplane_ex.pdf} \\
\url{https://www.youtube.com/watch?v=YIeSwj4W6YI}\\

Problem:
\[
(IP)
\begin{cases}
\text{maximize} & \mathbf{c}^T\mathbf{x} \\
\text{subject to} & A\mathbf{x} \leq \mathbf{b},
\end{cases}
\]
where $\mathbf{c} \in \mathbb{Z}^n, \mathbf{b} \in \mathbb{Z}^m, A \in \mathbb{Z}^{m \times n}$, and $\mathbf{x} \in \mathbb{Z}^n$.
Notation:
\begin{align*}
P &= \{ \mathbf{x}\in\mathbb{R}^n: A\mathbf{x} \leq \mathbf{b}  \}   &  
P_I = conv(\{\mathbf{x} \in \mathbb{Z}^n: A\mathbf{x} \leq \mathbf{b} \})
\end{align*}

Idea: Get NEW inequalities that better describe $P_I$ (cut piece of $P$ away). Main tool is $\lfloor\ \rfloor$.

Example:
\[
4x_1 + 2x_5 \leq 5 \ \Rightarrow\  2x_1 + x_5 \leq \frac{5}{2}\ \Rightarrow\  2x_1 + x_5 \leq \left\lfloor\frac{5}{2}\right\rfloor = 2.
\]



In general, for every $\mathbf{u} \geq 0$:
\[
P_I \subseteq P' =  \{ \mathbf{x}\in\mathbb{R}^n:  \mathbf{u}^TA\mathbf{x} \leq \lfloor\mathbf{u}^T\mathbf{b}\rfloor \text{ for all } \mathbf{u} \geq 0 \text{ with } \mathbf{u}^TA  \text{ integral} \}   \subseteq P
\]

\textbf{Theorem:} It is sufficient to consider $0 \leq \mathbf{u} \leq 1$.
\question{}{
Find $P'$ for the following set of inequalities:
\[
\begin{cases}
2x_1 + x_2 \leq 6  \\
-2x_1+x_2  \leq 0  \\
-x_2 \leq -1
\end{cases}
\vc{
\tikzset{insep/.style={inner sep=1pt, outer sep=0pt, circle, fill},}
\begin{tikzpicture}[scale=1]
\draw[fill=gray!20] (0.5,1) -- (2.5,1) -- (1.5,3)  -- cycle;
\foreach \x in {0,1,...,3}{
\foreach \y in {0,1,...,3}{
\draw (\x,\y) node[insep](x\x\y){};
}
\draw[-latex](0,0) -- (3.5,0) node[below]{$x_1$};
\draw[-latex](0,0) -- (0,3.5) node[left]{$x_2$};
\draw(1.5,1.5) node{$P$};
}
\end{tikzpicture}
}
\]
}
\solution{
We can generate the following equations and plot $P'$.
\begin{align*}
 0.5\cdot( 2x_1 + x_2 \leq 6) + 0.5\cdot(-x_2  \leq -1)  &\Rightarrow x_1 \leq 2.5 \Rightarrow  x_1 \leq 2 \\
 0.5\cdot( -2x_1 + x_2 \leq 0) + 0.5\cdot(-x_2  \leq -1)  &\Rightarrow -x_1 \leq -0.5 \Rightarrow  -x_1 \leq -1 \\
 \frac{1}{4} \cdot( 2x_1 + x_2 \leq 6) + \frac{3}{4} \cdot ( -2x_1 + x_2 \leq 0) &\Rightarrow -x_1 + x_2 \leq \frac{3}{2} \Rightarrow   -x_1 + x_2 \leq 1  \\
 \frac{3}{4} \cdot( 2x_1 + x_2 \leq 6) + \frac{1}{4} \cdot ( -2x_1 + x_2 \leq 0) &\Rightarrow x_1 + x_2 \leq \frac{9}{2} \Rightarrow     x_1 + x_2 \leq 4 
\end{align*}


\begin{center}
\tikzset{insep/.style={inner sep=1pt, outer sep=0pt, circle, fill},}
\begin{tikzpicture}[scale=1,color=black]
\draw (0.5,1) -- (2.5,1) -- (1.5,3)  -- cycle;
\draw[fill=gray!20] (1,1) -- (2,1) --  (2,2) -- (1.5,2.5) -- (1,2)  -- cycle;
\foreach \x in {0,1,...,3}{
\foreach \y in {0,1,...,3}{
\draw (\x,\y) node[insep](x\x\y){};
}
\draw[-latex](0,0) -- (3.5,0) node[below]{$x_1$};
\draw[-latex](0,0) -- (0,3.5) node[left]{$x_2$};
\draw(1.5,1.5) node{$P'$};
}
\end{tikzpicture}
\end{center}
}

\question{}{
Try to do the same operation on $P'$ and obtain $P''$. Recall $P'$ is given by:
\begin{align*}
2x_1 + x_2 &\leq 6   &  -2x_1+x_2  &\leq 0    & -x_2 &\leq -1 \\
x_1 &\leq 2 & -x_1 &\leq -1  &  -x_1 + x_2 &\leq 1 & x_1+x_2 &\leq 4 
\end{align*}
}
\solution{
\[
\frac{1}{2}(-x_1 + x_2 \leq 1)  + \frac{1}{2}(x_1+x_2 \leq 4) \Rightarrow x_2 \leq 2.5 \Rightarrow x_2 \leq 2 
\]
\begin{center}
\tikzset{insep/.style={inner sep=1pt, outer sep=0pt, circle, fill},}
\begin{tikzpicture}[scale=1,color=black]
\draw (0.5,1) -- (2.5,1) -- (1.5,3)  -- cycle;
\draw (1,1) -- (2,1) --  (2,2) -- (1.5,2.5) -- (1,2)  -- cycle;
\draw[fill=gray!20] (1,1) -- (2,1) --  (2,2) -- (1,2)  -- cycle;
\foreach \x in {0,1,...,3}{
\foreach \y in {0,1,...,3}{
\draw (\x,\y) node[insep](x\x\y){};
}
\draw[-latex](0,0) -- (3.5,0) node[below]{$x_1$};
\draw[-latex](0,0) -- (0,3.5) node[left]{$x_2$};
\draw(1.5,1.5) node{$P''$};
}
\end{tikzpicture}
\end{center}
}

Notice $P'' = P_I$.

Make a sequence $P = P^{(0)} \supseteq  P' = P^{(1)}  \supseteq  P'' = P^{(2)}   \supseteq  \cdots \supseteq P_I$.

\textbf{Theorem} If $P$ is a rational polytope, then there exists $k$ such that $P^{(k)}=P_I$.\\

The smallest $k$ such that $P^{(k)}=P_I$ is called \emph{Chv\'atal's rank}.

\textbf{How to generate cutting planes?} Run simplex algorithm and get cuts for things that are not integral.


Assume $x_1,\ldots, x_n \geq 0$ and integral. Constructing \emph{Gomory Cut} for
\begin{align}
a_1x_1+\cdots+a_nx_n = b \label{eq:first}
\end{align}
where $a_j, b \in \mathbb{R}$ (not necessarily integral). 
Note that \eqref{eq:first}  can be written as 
\[
(\lfloor a_1\rfloor + (\underbracket{a_1 - \lfloor a_1\rfloor}_{f_1})) x_1+ \cdots +(\lfloor a_n\rfloor + (\underbracket{a_n - \lfloor a_n\rfloor}_{f_n}))  x_n  = \lfloor b\rfloor+  (\underbracket{b -  \lfloor b\rfloor}_{f}),
\]

\begin{align}
(\lfloor a_1\rfloor + f_1) x_1+ \cdots +(\lfloor a_n\rfloor + f_n)  x_n  &= \lfloor b\rfloor+  f \label{A}
\\
\lfloor a_1\rfloor  x_1+ \cdots +\lfloor a_n\rfloor  x_n  &\leq \lfloor b\rfloor+  f 
\label{B}
\\
\lfloor a_1\rfloor x_1+ \cdots +\lfloor a_n\rfloor  x_n  &\leq \lfloor b\rfloor 
\label{C}
\\
-\lfloor a_1\rfloor x_1- \cdots -\lfloor a_n\rfloor  x_n  &\geq -\lfloor b\rfloor 
\label{D}
\\
 f_1 x_1+ \cdots + f_n x_n &\geq f.
\label{E}
\end{align}

Notice \eqref{B} is obtained from \eqref{A} by removing non-integral parts on the lefthand side. 
Since the lefthand side of \eqref{B} is an integer, we can make the righthand side
an integer and obtain \eqref{C}. By multiplying \eqref{C} by $-1$ we obtain \eqref{D}. 
Finally, \eqref{E} is obtained by adding \eqref{A} and \eqref{D}.

%\[
%f_1 x_1+ \cdots + f_n x_n  -f  = \lfloor b\rfloor  - \lfloor a_1\rfloor x_1 - 
%\cdots  - \lfloor a_n\rfloor x_n
%\]

%Observe: Right hand side $\in \mathbb{Z}$ and $0 \leq f < 1$. Hence 
%\begin{align*}
%f_1 x_1+ \cdots + f_n x_n  -f  &\geq 0   &    f_1 x_1+ \cdots + f_n x_n &\geq f.
%\end{align*}
This can be used in Simplex method if it gives a solution that is not integral.

 
Example:

\[
(IP)
\begin{cases}
\text{maximize} & x_2 \\
\text{subject to} & 3x_1 + 2x_2 \leq 6 \\
                        & -3x_1 + 2x_2 \leq 0 \\
                        &     x_1,x_2 \geq 0
\end{cases}
\hskip 5em
\vc{
\tikzset{insep/.style={inner sep=1pt, outer sep=0pt, circle, fill},}
\begin{tikzpicture}[scale=1]
\draw[fill=gray!20, line width=1pt] (0,0) -- (2,0) -- (1,1.5)  -- cycle;
\foreach \x in {0,1,...,2}{
\foreach \y in {0,1,...,2}{
\draw (\x,\y) node[insep](x\x\y){};
}
\draw[-latex](0,0) -- (2.5,0) node[below]{$x_1$};
\draw[-latex](0,0) -- (0,2.5) node[left]{$x_2$};
\draw[dotted, line width=1pt]
(0,0) -- ++(56.3:3)
(2,0) -- ++(-56.3:-3)
;
}
\end{tikzpicture}
}
\]

Solve LP relaxation using simplex method.

\[
\begin{array}{ccccccc} 
x_3  &   =  &  6  &-&   3x_1 &-&   2x_2 \\
x_4  &   =  &  0  &+ & 3x_1 &-&   2x_2  \\
%\hline 
z   &   =     &  0  &  &          &+&   x_2 
\end{array}
\sim
\begin{array}{ccccccc} 
x_1  &   =  &  1  &-&   \frac{1}{6}x_3 &+&  \frac{1}{6}x_4 \\
x_2  &   =  &  \frac{3}{2}  &-& \frac14 x_3 &-&   \frac14 x_4  \\
%\hline 
z   &   =     &  \frac{3}{2}  &-  & \frac14 x_3 &-&   \frac14 x_4 
\end{array}
\]

\question{}{
Find a cutting plane using $x_2  =   \frac{3}{2}  - \frac14 x_3 - \frac14 x_4$.
Then substitute for $x_3$ and $x_4$ and get an inequality for the original problem.
}
\solution{
Small rewriting gets:
$x_2 +  \frac14 x_3 + \frac14 x_4  =   \frac{3}{2}$.
The cutting plane is
\[
 \frac14 x_3 + \frac14 x_4 \geq \frac{1}{2}
\]
Notice the cutting plane is not satisfied by solution $(1,\frac{3}{2},0,0)$, which was the result of simplex method.
By substituting 
$
\begin{array}{ccccccc} 
x_3  &   =  &  6  &-&   3x_1 &-&   2x_2\\
x_4  &   =  &  0  &+ & 3x_1 &-&   2x_2  
\end{array}
$
we get $x_2 \leq 1$.

\begin{center}
\tikzset{insep/.style={inner sep=1pt, outer sep=0pt, circle, fill},}
\begin{tikzpicture}[scale=1,color=black]
\draw[dotted,line width = 1pt]
(0,0) -- ++(56.3:3)
(2,0) -- ++(-56.3:-3)
(0,1) -- (2.5,1)
;
\draw[fill=gray!20,line width=1pt] (0,0) -- (2,0) -- (1.3333,1) -- (0.66666,1)  -- cycle;
\foreach \x in {0,1,...,2}{
\foreach \y in {0,1,...,2}{
\draw (\x,\y) node[insep](x\x\y){};
}
\draw[-latex](0,0) -- (2.5,0) node[below]{$x_1$};
\draw[-latex](0,0) -- (0,2.5) node[left]{$x_2$};
}
\end{tikzpicture}
\end{center}
}
Notice we got additional inequality. It is possible to add new slack variable $x_5$
and add the following equation 
\[
 \frac14 x_3 + \frac14 x_4 \geq \frac{1}{2}   \Rightarrow    \frac14 x_3 + \frac14 x_4 - x_5 = \frac{1}{2}
\]
to the simplex table:

\[
\begin{array}{ccccccc} 
x_1  &   =  &  1  &-&   \frac{1}{6}x_3 &+&  \frac{1}{6}x_4 \\
x_2  &   =  &  \frac{3}{2}  &-& \frac14 x_3 &-&   \frac14 x_4  \\
%\hline 
z   &   =     &  \frac{3}{2}  &-  & \frac14 x_3 &-&   \frac14 x_4 
\end{array}
\Rightarrow
\begin{array}{ccccccc} 
x_1  &   =  &  1  &-&   \frac{1}{6}x_3 &+&  \frac{1}{6}x_4 \\
x_2  &   =  &  \frac{3}{2}  &-& \frac14 x_3 &-&   \frac14 x_4  \\
x_5  &   =  &  -\frac{1}{2}  &+& \frac14 x_3 &+&   \frac14 x_4  \\
%\hline 
z   &   =     &  \frac{3}{2}  &-  & \frac14 x_3 &-&   \frac14 x_4 
\end{array}
\]
Notice that the table is illegal since it assigns $x_5 = -\frac12$. Notice we can reoptimize
by changing $x_3$ for $x_5$. We should actually use something called \emph{Dual Simplex Method}.
We get
\[
\begin{array}{ccccccc} 
x_1  &   =  &  \frac{2}{3}  &-& \frac{2}{3}x_5 &+&  \frac{1}{3}x_4 \\
x_2  &   =  &  1  &-& x_5 & &    \\
x_3  &   =  &  2  &+& 4 x_5 &-&  x_4  \\
%\hline 
z   &   =     &  1  &-  & x_5 & & 
\end{array}
\]
\question{}{
Find another Gomory Cut.
}
\solution{
The equation used for cut is
\[
x_1  =  \frac{2}{3}  - \frac{2}{3}x_5 + \frac{1}{3}x_4
\ \Rightarrow\ 
x_1  + \frac{2}{3}x_5 + \frac{-1}{3}x_4  =  \frac{2}{3} 
\]
The resulting cutting plane is
\[
 \left(\frac{2}{3} - \left\lfloor\frac23\right\rfloor\right) x_5  + \left(\frac{-1}{3}-\left\lfloor\frac{-1}{3}\right\rfloor\right)x_4  \geq  \frac{2}{3} - \left\lfloor \frac{2}{3} \right\rfloor
\ \Rightarrow\ 
\frac{2}{3}x_5  + \frac{2}{3}x_4  \geq  \frac{2}{3}  
\]
Using substitution we obtain equation
\[
x_1 - x_2 \geq 0 
\]

\begin{center}
\tikzset{insep/.style={inner sep=1pt, outer sep=0pt, circle, fill},}
\begin{tikzpicture}[scale=1,color=black]
\draw[dotted,line width = 1pt]
(0,0) -- ++(56.3:3)
(2,0) -- ++(-56.3:-3)
(0,1) -- (2.5,1)
(0,0) -- (45:3.3)
;
\draw[fill=gray!20,line width=1pt] (0,0) -- (2,0) -- (1.3333,1) -- (1,1)  -- cycle;
\foreach \x in {0,1,...,2}{
\foreach \y in {0,1,...,2}{
\draw (\x,\y) node[insep](x\x\y){};
}
\draw[-latex](0,0) -- (2.5,0) node[below]{$x_1$};
\draw[-latex](0,0) -- (0,2.5) node[left]{$x_2$};
}
\end{tikzpicture}
\end{center}
}

Last simplex table is
\[
\begin{array}{ccccccc} 
x_1  &   =  &  \frac{2}{3}  &-& \frac{2}{3}x_5 &+&  \frac{1}{3}x_4 \\
x_2  &   =  &  1  &-& x_5 & &    \\
x_3  &   =  &  2  &+& 4 x_5 &-&  x_4  \\
x_6  &  =   & -\frac{2}{3}  &+& \frac23 x_5 &+& \frac{2}3 x_4  \\
%\hline 
z   &   =     &  1  &-  & x_5 & & 
\end{array}
\sim
\begin{array}{ccccccc} 
x_1  &   =  &  1  &-& x_5 &+&  \frac{1}{2}x_6 \\
x_2  &   =  &  1  &-& x_5 & &    \\
x_3  &   =  &  1  &+& 5 x_5 &-&  \frac32x_6  \\
x_4  &  =   & 1  &-&  x_5 &+& \frac{2}3 x_6  \\
%\hline 
z   &   =     &  1  &-  & x_5 & & 
\end{array}
\]
Now the solution is integral.

\begin{itemize}
\item May need many cuts (but terminates if something like Bland's rule used)
\item Used together with Branch and Bound
\end{itemize}

%\vfill

%\emph{Next time: Matchings (or Mixed Integer Cuts).}

\end{document}






