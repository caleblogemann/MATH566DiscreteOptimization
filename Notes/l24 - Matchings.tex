\documentclass[11pt]{article}
\usepackage[utf8]{inputenc}
\usepackage[czech, english]{babel}
\usepackage[left=2cm,right=2cm,top=2cm,bottom=2cm]{geometry}
\usepackage{amssymb}
\usepackage{amsthm}
\usepackage{amsmath}
\usepackage{mathtools}
\usepackage{tikz}
\usetikzlibrary{shapes.geometric}
\usetikzlibrary{shapes.multipart}
\usetikzlibrary{arrows}
\usetikzlibrary{graphs}
\usepackage{hyperref}


% Itemize environment with small skips
\newenvironment{packeditemize}{
\begin{itemize}
  \setlength{\itemsep}{1pt}
  \setlength{\parskip}{0pt}
  \setlength{\parsep}{0pt}
}{\end{itemize}}


% Fancy footnote....
\usepackage{fancyhdr}
\pagestyle{fancy}
\usepackage{lastpage}
\rfoot{MATH 566 - 24, page \thepage/\pageref{LastPage}}
\cfoot{}
\rhead{}
\lhead{}
\renewcommand{\headrulewidth}{0pt}
\renewcommand{\footrulewidth}{0pt}


\newtheorem{theorem*}{Theorem} 

% Itemize environment with small skips
\newenvironment{packedenumerate}{
\begin{enumerate}
  \setlength{\itemsep}{1pt}
  \setlength{\parskip}{0pt}
  \setlength{\parsep}{0pt}
}{\end{enumerate}}


\newcommand{\vc}[1]{\ensuremath{\vcenter{\hbox{#1}}}}

\newcounter{excounter}
\setcounter{excounter}{1}
\newcommand\question[2]{\vskip 1em  \noindent\textbf{\arabic{excounter}\addtocounter{excounter}{1}:} \emph{#1} \noindent#2}
\newcommand\solution[1]{\vskip 0.5em \noindent\textbf{Solution:} #1}
%\newcommand\like{\par \noindent\emph{(This question is: good - bad - ugly? Difficulty: 0-10:\hskip 3em )}} 

\newcommand\lecturer[1]{\textbf{#1}}
\newcommand\hideforstudent[1]{#1}


%\renewcommand\lecturer[1]{{\color{white} \textbf{#1} }}
%\renewcommand\hideforstudent[1]{{\color{white}  }}
%\renewcommand\solution[1]{{\color{white} \vskip 0.5em \noindent\textbf{Solution:} #1 }}



\setlength{\parindent}{0cm}
\setlength{\parskip}{0.1cm}

\begin{document}

Fall  2016, MATH-566

\centerline{{\Large \textbf{Maximum Matchings in Graphs (Edmonds' Blossom Algorithm)}}}

Source: Chapter 10.1, 10.5

Let $G=(V,E)$ be a graph. \textbf{Matching} $M$ a subset of $E$, such
that graph $(M,E)$ has maximum degree one.

Problem: For a graph $G$, find maximum matching $M$.


\question{}{
Formulate the maximum matching problem as an integer programming $(IP)$.
}
\solution{
To every edge $e \in E$ assign variable $x_e \in \{0,1\}$.
For every vertex, sum of edges is $\leq 1$. 
Maximize sum over all edges.
\[
(IP)
\begin{cases}
\text{maximize}  &  \sum_{e \in E} x_e \\
\text{subject to} & A\mathbf{x} \leq \mathbf{1}\\
                          &  \mathbf{x} \in \{0,1\}^{|E|},
\end{cases}
\]
where $A$ is the incidence matrix of $G$.
}

\question{}{
Is there a condition on $G$ that guarantees that a relaxation of $(IP)$ to linear program always has an integral optimal solution?
Notice that the matrix $A$ is an incidence matrix of $G$. 
}
\solution{
If $G$ is a bipartite graph, then the incidence matrix is totally unimodular. That guarantees the relaxation to LP will give an integral solution.
}



\question{}{
Show that finding maximum matching in a bipartite graph $G$ can be done using maximum
flow algorithm. Is there a \emph{certificate} that a matching is maximum?
\begin{center}
\tikzset{insep/.style={inner sep=1.5pt, outer sep=0pt, circle, fill},}
\begin{tikzpicture}[scale=1]
\foreach \x in {0,1,...,3}{
\draw (0,\x) node[insep](x0\x){};
\draw (2,\x) node[insep](x1\x){};
}
\foreach \x/\y in {0/2,1/1,3/2,2/2,2/0,1/3}{
\draw (x0\x) -- (x1\y);
}
\end{tikzpicture}
\end{center}
}
\solution{
Let $G=(V,E)$, where $V=V_1\cup V_2$. Add $s$ adjacent to all vertices in $V_1$
and $t$ adjacent to all vertices in $V_2$. Orient all edges from $s$ to $t$ and add 
capacities 1 everywhere.
\begin{center}
\tikzset{insep/.style={inner sep=1.5pt, outer sep=0pt, circle, fill},}
\begin{tikzpicture}[scale=1]
\draw
(-2,1.5) node[insep,label=left:$s$](s){} 
(4,1.5) node[insep,label=right:$t$](t){} 
;
\foreach \x in {0,1,...,3}{
\draw[-latex] (s) -- (0,\x) node[insep](x0\x){};
\draw[-latex] (2,\x) node[insep](x1\x){} -- (t);
}
\foreach \x/\y in {0/2,1/1,3/2,2/2,2/0,1/3}{
\draw[-latex] (x0\x) -- (x1\y);
}
\end{tikzpicture}
\end{center}
Notice that maximality can be shown using minimum cut.
}

\textbf{Definition:}
Let $M$ be a matching in graph $G=(V,E)$.
A vertex $v\in V$ is \textbf{covered} if exists $e \in M$ such that $v \in e$,
otherwise $v$ is \textbf{exposed}.
A matching is called a \textbf{perfect matching} if all vertices are covered.

Perfect matchings are covered in the book but we are skipping them.

Notice: maximizing $M$ is the same as minimizing exposed vertices.

\newpage

Inspiration by flow: use \emph{augmenting paths}.

A path $P$ is \textbf{$M$-alternating} if $E(P)\setminus M$ is a matching.
An \textbf{$M$-alternating} path $P$ is \textbf{$M$-augmenting}  if $P$ has
positive length and its endpoints are exposed in $M$. 
Augmented $M' = M \Delta E(P)$.


\question{}{
Assume there is a matching $M$ (thick lines). Find augmenting path(s) and augment $M$.
\begin{center}
\tikzset{insep/.style={inner sep=1.5pt, outer sep=0pt, circle, fill},}
\begin{tikzpicture}[scale=1]
\foreach \x in {1,2,...,6}{
\draw  (\x,0) node[insep,label=below:$v_\x$](x\x){};
}
\draw(x1) -- (x6);
\draw[line width=1.6pt] (x2)--(x3) (x4)--(x5);
\foreach \x in {2,...,5}{ \draw  (\x,1) node[insep,label=above:$u_\x$](u\x){}; } 
\draw (x1)--(u2)--(u5) (x2)--(u3) (u3)--(x4);
\draw[line width=1.6pt] (u2)--(u3) (u4)--(u5);  
\end{tikzpicture}
\hskip 4em
\begin{tikzpicture}[scale=1]
\draw  
(0,0) node[insep,label=below:$v_1$](x1){}
--
(1,0) node[insep,label=below:$v_2$](x2){}
--
(1,1) node[insep,label=left:$v_3$](x3){}
--
++(1,0) node[insep,label=above right:$v_4$](x4){}
--
++(1,0) node[insep,label=right:$v_5$](x5){}
--
++(0,-1) node[insep,label=below:$v_6$](x6){}
--
++(-1,0) node[insep,label=below:$v_7$](x7){}
(x4) -- ++(0,1) node[insep,label=right:$v_8$](x8){}
;
\draw(x2) -- (x7) (x3)--(x7)--(x5) (x6)--(x4) ;
\draw[line width=1.6pt] (x2)--(x3) (x4)--(x5) (x6)--(x7) ;
\end{tikzpicture}
\end{center}
}



\solution{
Augmenting path is a path $P$ with endpoints exposed, inner points covered and 
the edges of the matching are alternating on $P$. 
\begin{center}
\tikzset{insep/.style={inner sep=1.5pt, outer sep=0pt, circle, fill},}
\begin{tikzpicture}[scale=1]
\foreach \x in {1,2,...,6}{
\draw  (\x,0) node[insep,label=below:$v_\x$](x\x){};
}
\draw(x1) -- (x6);
\draw[line width=1.6pt] (x1)--(x2) (x3)--(x4) (x5)--(x6) ;
\foreach \x in {2,...,5}{ \draw  (\x,1) node[insep,label=above:$u_\x$](u\x){}; } 
\draw (x1)--(u2)--(u5) (x2)--(u3) (u3)--(x4);
\draw[line width=1.6pt] (u2)--(u3) (u4)--(u5);  
\end{tikzpicture}
\hskip 4em
\begin{tikzpicture}[scale=1]
\draw  
(0,0) node[insep,label=below:$v_1$](x1){}
--
(1,0) node[insep,label=below:$v_2$](x2){}
--
(1,1) node[insep,label=left:$v_3$](x3){}
--
++(1,0) node[insep,label=above right:$v_4$](x4){}
--
++(1,0) node[insep,label=right:$v_5$](x5){}
--
++(0,-1) node[insep,label=below:$v_6$](x6){}
--
++(-1,0) node[insep,label=below:$v_7$](x7){}
(x4) -- ++(0,1) node[insep,label=right:$v_8$](x8){}
;
\draw(x2) -- (x7) (x3)--(x7)--(x5) (x6)--(x4) ;
\draw[line width=1.6pt] (x1)--(x2) (x3)--(x7) (x6)--(x5) (x4)-- (x8) ;
\end{tikzpicture}
\end{center}
}


\textbf{Theorem 10.7}
Let $G$ be a graph with matching $M$. Then $M$ is maximum iff there is no $M$-augmenting path.

\question{}{Prove Theorem 10.7}
\solution{
$\Rightarrow$: Augmenting path increases the size of the matching\\
$\Leftarrow$. Consider $M'$ being a matching with more edges than $M$.
Take the symmetric difference of $M$ and $M'$. It is a graph of maximum
degree two, which gives set of even cycles and paths. Implies one of the paths
must be $M$-augmenting.
}

\question{}{
Can we use augmenting walks instead of  paths? It particular, examine walk\\
$v_1, v_2, v_3, v_4, v_5, v_6, v_7, v_5, v_4, v_8$ in the graph on the right-hand side.
}
\solution{
We cannot augment on it. Both $v_4$ and $v_5$ would have two matching edges.
}


Idea: Start from exposed vertices as roots and build \textbf{alternating forest} $F$. 
Alternate non-matching edges and matching edges. 
This gives layers of matching edges and non-matching edges.

\begin{center}
\tikzset{insep/.style={inner sep=1.5pt, outer sep=0pt, circle, fill},}
\tikzset{inwhite/.style={inner sep=1.5pt, outer sep=0pt, circle, fill=white,draw},}
\begin{tikzpicture}[scale=1]
\draw  
(0,0) node[inwhite](x1){}
(-0.5,1) node[insep](x2){}
 (0.5,1) node[insep](x3){}
(-0.5,2) node[inwhite](x4){}
 (0.5,2) node[inwhite](x5){}
(-1,3) node[insep](x6){}
 (0,3) node[insep](x7){}
(-1,4) node[inwhite](x8){}
 (0,4) node[inwhite](x9){}
(1,3) node[insep](x10){}
(1,4) node[inwhite](x11){}
 
 (3,0) node[inwhite](y1){}
 (2,1) node[insep](y2){}
 (3,1) node[insep](y3){}
 (4,1) node[insep](y4){}
 (2,2) node[inwhite](y5){}
 (3,2) node[inwhite](y6){}
 (4,2) node[inwhite](y7){}
 (4,3) node[insep](y8){}
 (4,4) node[inwhite](y9){}
;
\draw(x1)--(x2) (x1)--(x3) (x4)--(x6) (x4)--(x7) (x5)--(x10) (y1)--(y2) (y1)--(y3) (y1)--(y4) (y7)--(y8);
\draw[line width=1.6pt] (x2)--(x4)  (x3)--(x5) (x6)--(x8) (x7)--(x9) (x10)--(x11) (y2)--(y5)
(y3)--(y6)(y4)--(y7) (y8)--(y9);
\end{tikzpicture}
\end{center}
Notice that not all edges of $G$ are present in the forest $F$! Call vertex \textbf{outer} vertex if it is in even distance from the root (white ones).

Assume building of the alternating forest by picking an edge adjacent to outer vertex and trying to extend the forest edge by edge.

\question{}{
What happens when we try to add any of the dotted edges?

\begin{center}
\tikzset{insep/.style={inner sep=1.5pt, outer sep=0pt, circle, fill},}
\tikzset{inwhite/.style={inner sep=1.5pt, outer sep=0pt, circle, fill=white,draw},}
\begin{tikzpicture}[scale=1]
\draw  
(0,0) node[inwhite](x1){}
(-0.5,1) node[insep](x2){}
 (0.5,1) node[insep](x3){}
(-0.5,2) node[inwhite](x4){}
 (0.5,2) node[inwhite](x5){}
 
 (3,0) node[inwhite](y1){}
 (2,1) node[insep](y2){}
 (3,1) node[insep](y3){}
 (4,1) node[insep](y4){}
 (2,2) node[inwhite](y5){}
 (3,2) node[inwhite](y6){}
 (4,2) node[inwhite](y7){}
 (5,2) node[inwhite](y8){}
 (6,2) node[inwhite](y9){}

;
\draw(x1)--(x2) (x1)--(x3) (y1)--(y2) (y1)--(y3) (y1)--(y4) ;
\draw[line width=1.6pt] (x2)--(x4)  (x3)--(x5)  (y2)--(y5) (y8)--(y9)
(y3)--(y6)(y4)--(y7);
\draw[line width=2pt,dotted,color=red] (x5) --node[above]{$e_1$} (y5) (y7)--node[above]{$e_2$}(y8)  (y6)--node[above]{$e_3$}(y4)
(x3) --node[below]{$e_4$} (y5)
;
\end{tikzpicture}
\end{center}
}
\solution{
$e_1$ gives an augmenting path, $e_2$ allows the forest to grow, 
$e_3,e_4$ can be dropped - if there is an augmenting path using them, there is one avoiding them.
}

\question{}{
What happens with $e_5$?
\begin{center}
\tikzset{insep/.style={inner sep=1.5pt, outer sep=0pt, circle, fill},}
\tikzset{inwhite/.style={inner sep=1.5pt, outer sep=0pt, circle, fill=white,draw},}
\begin{tikzpicture}[scale=1]
\draw  
(0,0) node[inwhite](x1){}
(-0.5,1) node[insep](x2){}
 (0.5,1) node[insep](x3){}
(-0.5,2) node[inwhite](x4){}
 (0.5,2) node[inwhite](x5){}
 
 (3,0) node[inwhite](y1){}
 (2,1) node[insep](y2){}
 (3,1) node[insep](y3){}
 (4,1) node[insep](y4){}
 (2,2) node[inwhite](y5){}
 (3,2) node[inwhite](y6){}
 (4,2) node[inwhite](y7){}
;
\draw(x1)--(x2) (x1)--(x3) (y1)--(y2) (y1)--(y3) (y1)--(y4) ;
\draw[line width=1.6pt] (x2)--(x4)  (x3)--(x5)  (y2)--(y5) (y3)--(y6)(y4)--(y7);
\draw[line width=2pt,dotted,color=red] (x4) --node[above]{$e_5$} (x5)  
%(x3) --node[below]{$e_4$} (y5)
;
\end{tikzpicture}
\end{center}
}
\solution{
Edge $e_5$ is important. Consider edge $e_4$ from the previous. Show  the augmenting paths is using $e_5$ and it cannot be avoided.
\begin{center}
\tikzset{insep/.style={inner sep=1.5pt, outer sep=0pt, circle, fill},}
\tikzset{inwhite/.style={inner sep=1.5pt, outer sep=0pt, circle, fill=white,draw},}
\begin{tikzpicture}[scale=1]
\draw  
(0,0) node[inwhite](x1){}
(-0.5,1) node[insep](x2){}
 (0.5,1) node[insep](x3){}
(-0.5,2) node[inwhite](x4){}
 (0.5,2) node[inwhite](x5){}
 
 (3,0) node[inwhite](y1){}
 (2,1) node[insep](y2){}
 (3,1) node[insep](y3){}
 (4,1) node[insep](y4){}
 (2,2) node[inwhite](y5){}
 (3,2) node[inwhite](y6){}
 (4,2) node[inwhite](y7){}
;
\draw(x1)--(x2) (x1)--(x3) (y1)--(y2) (y1)--(y3) (y1)--(y4) ;
\draw[line width=1.6pt] (x2)--(x4)  (x3)--(x5)  (y2)--(y5) (y3)--(y6)(y4)--(y7);
\draw[line width=2pt,dotted] (x4) --node[above]{$e_5$} (x5)  
(x3) --node[below]{$e_4$} (y5)
;
\draw[line width=2pt,dotted]  (x1)--(x2)--(x4)--(x5)--  
(x3) -- (y5) -- (y2)--(y1)
;
\end{tikzpicture}
\end{center}
}

\textbf{Blossom} is an odd cycle on $k$ vertices containing $\frac{k-1}{2}$ edges from $M$.
\question{}{
Let $C$ be a blossom, where $v$ is not matched with other vertex in $C$. Show that alternating path entering $C$ using a matching edge $e$ containing $v$ can leave $C$ using unmatched edge from any vertex of $C$.

\begin{center}
\tikzset{insep/.style={inner sep=1.5pt, outer sep=0pt, circle, fill},}
\tikzset{inwhite/.style={inner sep=1.5pt, outer sep=0pt, circle, fill=white,draw},}
\begin{tikzpicture}[scale=1]
\draw  
(0,0) node[inwhite,label=left:$v$](x1){}
(-0.5,1) node[inwhite](x2){}
 (0.5,1) node[inwhite](x3){}
(-0.5,2) node[inwhite](x4){}
 (0.5,2) node[inwhite](x5){}
 (0,-1) node[inwhite](y1){}
 (0,-2) node[inwhite](y2){}

;
\draw (y2)--(y1) (x1)--(x2) (x1)--(x3) (x4)--(x5);
\draw[line width=1.6pt] (x2)--(x4)  (x3)--(x5) (y1)-- node[right]{$e$}(x1);
;
\draw
(x1) -- ++(0:1) node[inwhite,label=right:$z_1$](z1){}
(x2) -- ++(180:1) node[inwhite,label=left:$z_2$](z2){}
(x3) -- ++(0:1) node[inwhite,label=right:$z_1$](z3){}
(x4) -- ++(130:1) node[inwhite,label=left:$z_1$](z4){}
(x5) -- ++(50:1) node[inwhite,label=right:$z_1$](z5){}
;
\end{tikzpicture}
\end{center}
}


\newpage

Since blossom acts like a vertex that can be matched to anything, we contract the blossom.

\begin{center}
\tikzset{insep/.style={inner sep=1.5pt, outer sep=0pt, circle, fill},}
\tikzset{inwhite/.style={inner sep=1.5pt, outer sep=0pt, circle, fill=white,draw},}
\vc{
\begin{tikzpicture}[scale=1]
\draw  
(0,0) node[inwhite,label=left:$v$](x1){}
(-0.5,1) node[inwhite](x2){}
 (0.5,1) node[inwhite](x3){}
(-0.5,2) node[inwhite](x4){}
 (0.5,2) node[inwhite](x5){}
 (0,-1) node[inwhite](y1){}
 (0,-2) node[inwhite](y2){}

;
\draw (y2)--(y1) (x1)--(x2) (x1)--(x3) (x4)--(x5);
\draw[line width=1.6pt] (x2)--(x4)  (x3)--(x5) (y1)--(x1);
;
\draw
(x1) -- ++(0:1) node[inwhite,label=right:$z_1$](z1){}
(x2) -- ++(180:1) node[inwhite,label=left:$z_2$](z2){}
(x3) -- ++(0:1) node[inwhite,label=right:$z_1$](z3){}
(x4) -- ++(130:1) node[inwhite,label=left:$z_1$](z4){}
(x5) -- ++(50:1) node[inwhite,label=right:$z_1$](z5){}
;
\end{tikzpicture}
}
\hskip 3em
$\Rightarrow$
\hskip 3em
\vc{
\begin{tikzpicture}[scale=1]
\draw  
(0,0) node[inwhite,label=left:$v$](x1){}
(-0.5,1) coordinate (x2)
 (0.5,1) coordinate(x3){}
(-0.5,2) coordinate(x4){}
 (0.5,2) coordinate(x5){}
 (0,-1) node[insep](y1){}
 (0,-2) node[inwhite](y2){}

;
\draw (y2)--(y1) ;
\draw[line width=1.6pt]  (y1)--(x1);
;
\draw
(x1)  ++(0:1) node[inwhite,label=right:$z_1$](z1){} -- (x1)
(x2)  ++(180:1) node[inwhite,label=left:$z_2$](z2){}  -- (x1)
(x3)  ++(0:1) node[inwhite,label=right:$z_1$](z3){}  -- (x1)
(x4)  ++(130:1) node[inwhite,label=left:$z_1$](z4){}  -- (x1)
(x5)  ++(50:1) node[inwhite,label=right:$z_1$](z5){}  -- (x1)
;
\end{tikzpicture}
}
\end{center}

\textbf{Edmonds Blossom Algorithm} (sketch)
\begin{enumerate}
\item $M = \emptyset$
\item\hskip 0em  $F = $ uncovered vertices, all edges unexplored
\item\hskip 0em while exists unexplored edge $e$ adjacent to outer vertex of $F$
\item\hskip 4em  if $e$ connects two outer vertices from different components of $F$, 
\item \hskip 6em get $M$-alternating path and augment $M$, go to 2.
\item\hskip 4em  if $e$ connects two outer vertices from the same components of $F$:
\item \hskip 6em  find a blossom and contract it, unexplore edges going out from no-outer vertices of the blossom..
\item\hskip 4em  if $e$ connects and outer vertex to unexplored node $x$,
\item \hskip 6em  add $x$ and its neighbor in $M$ to $F$.
\end{enumerate}

If implemented carefully, runs in $O(\sqrt{n}m)$, where $n=|V|$ and $m=|E|$.

\question{}{
Try to run the algorithm on Petersen's graph.
}
\solution{
A good choice of order of explored edges can show all features of the algorithm.
}



\end{document}






