\documentclass[11pt]{article}
\usepackage[utf8]{inputenc}
\usepackage[czech, english]{babel}
\usepackage[left=2cm,right=2cm,top=2cm,bottom=2cm]{geometry}
\usepackage{amssymb}
\usepackage{amsthm}
\usepackage{amsmath}
\usepackage{tikz}
\usetikzlibrary{shapes.geometric}
\usepackage{hyperref}


% Itemize environment with small skips
\newenvironment{packeditemize}{
\begin{itemize}
  \setlength{\itemsep}{1pt}
  \setlength{\parskip}{0pt}
  \setlength{\parsep}{0pt}
}{\end{itemize}}


% Fancy footnote....
\usepackage{fancyhdr}
\pagestyle{fancy}
\usepackage{lastpage}
\rfoot{MATH 566 - 17, page \thepage/\pageref{LastPage}}
\cfoot{}
\rhead{}
\lhead{}
\renewcommand{\headrulewidth}{0pt}
\renewcommand{\footrulewidth}{0pt}


\newtheorem{theorem*}{Theorem} 

% Itemize environment with small skips
\newenvironment{packedenumerate}{
\begin{enumerate}
  \setlength{\itemsep}{1pt}
  \setlength{\parskip}{0pt}
  \setlength{\parsep}{0pt}
}{\end{enumerate}}


\newcounter{excounter}
\setcounter{excounter}{1}
\newcommand\question[2]{\vskip 1em  \noindent\textbf{\arabic{excounter}\addtocounter{excounter}{1}:} \emph{#1} \noindent#2}
\newcommand\solution[1]{\vskip 0.5em \noindent\textbf{Solution:} #1}
%\newcommand\like{\par \noindent\emph{(This question is: good - bad - ugly? Difficulty: 0-10:\hskip 3em )}} 

\newcommand\lecturer[1]{\textbf{#1}}
\newcommand\hideforstudent[1]{#1}


%\renewcommand\lecturer[1]{{\color{white} \textbf{#1} }}
%\renewcommand\hideforstudent[1]{{\color{white} #1 }}
%\renewcommand\solution[1]{{\color{white} \vskip 0.5em \noindent\textbf{Solution:} #1 }}



\setlength{\parindent}{0cm}
\setlength{\parskip}{0.1cm}

\begin{document}

Fall  2016, MATH-566

\centerline{{\Large \textbf{Minimum Cuts in Unidrected Graphs}}}

Recall: $\delta(S)$ is the set of edges with exactly one endpoint in $S$.

\textbf{Problem:}\\
\emph{Input:} Graph $G=(V,E)$ and cost function $u: E \rightarrow \mathbb{R}^+$.\\
\emph{Output:} Global Minimum Cut, that is $S \subset V$ such minimizing $u(\delta(S))$

\question{}{
Find a minimum cut in the following graph:
\begin{center}
\tikzset{insep/.style={inner sep=2pt, outer sep=0pt, circle, fill},}
\begin{tikzpicture}[scale=1.3]
\draw
(0,0) node[insep,label=below:$g$](g){}
(2,0) node[insep,label=below:$f$](f){}
(2,1) node[insep,label=right:$e$](e){}
(2,2) node[insep,label=right:$a$](a){}
(1,3) node[insep,label=above:$b$](b){}
(0,1) node[insep,label=below left:$d$](d){}
(0,2) node[insep,label=above:$c$](c){}
(-1,1) node[insep,label=left:$h$](h){}
;
\draw
\foreach \x/\y/\t in {g/f/6, d/e/3, c/a/2, c/b/3, b/a/5,h/c/2, h/d/1}{
(\x)--node[pos=0.5,label=above:$\t$]{} (\y)
}
(h)--node[pos=0.5,label=below:$5$]{} (g)
\foreach \x/\y/\t in {g/d/2, d/c/4, f/e/2, e/a/3}{
(\x)--node[pos=0.5,label=right:$\t$]{} (\y)
}
;
\end{tikzpicture}
\end{center}
}


Notation: 

$\lambda(G)$ is the capacity of minimum cut of  $G$.

$\lambda(G; v, w)$ is the capacity of minimum $(v,w)$-cut of $G$.


\question{}{Find an algorithm using network flows.}
\solution{
Pick a vertex and try to find a network flow to every other vertex.
}


\textbf{Node Identification Algorithm:}

Let $G_{uv}$ be a graph obtained from $G$ by identification of $u$ and $v$ (delete loops, keep parallel edges).

Main idea: 
\[
\lambda(G) \leq \min( \lambda(G_{vw}), \lambda(G;v,w))
\]

How to pick $v,w$?

\textbf{A legal ordering} starting at $v_1$ is $v_1,v_2,\ldots,v_n$
if for all $i$, $v_i$ has the largest capacity of edges joining it to $v_1,\ldots,v_{i-1}$.

\question{}{
Find a legal ordering starting with vertex $a$.
}
\solution{
$a,b,c,d,e,h,g,f$
}

Main theorem: If $v_1,\ldots,v_n$ is a legal ordering of $G$, then $\delta(v_n)$ is a minimum $v_n,v_{n-1}$ cut of $G$.

\begin{enumerate}
\item $M := \infty$ and $A$ undefined
\item while $G$ has more than 1 vertex
\item \hskip 4em Find a legal ordering $v_1,v_2,\ldots,v_n$ of $G$
\item \hskip 4em If $u(\delta(v_n)) < M$
\item \hskip 8em      $M := u(\delta(v_n))$ and $A := \delta(v_n)$
\item \hskip 4em $G := G_{v_nv_{n-1}}$
\item return $A$
\end{enumerate}

\question{}{
Run node identification algorithm.
}
\vskip 7em


\textbf{Random Contraction Algorithm:}

\begin{enumerate}
\item while $G$ has more than 2 vertices
\item \hskip 4em Choose and edge $e$ of $G$ with probability $u(e)/u(E)$
\item \hskip 4em $G := G_{vw}$, where $e = vw$
\item return the unique cut in $G$.
\end{enumerate}

\question{}{
Let $A$ be a minimum cut of $G$. 
Show that the random contraction algorithm returns $A$ with probability at least $2/(n(n-1))$.
}
\solution{
Let $u(A) = \sum_{e \in A}u(A)$.  Then
\[
P(\text{edge of } A \text{ is picked for contraction}) = \frac{u(A)}{u(E)}
\]
Notice that $A$ is the minimum cut in $G$. 
Hence $u(A) \leq u(C)$ for any other cut. 
In particular, we consider cuts around each vertex.
A cut around vertex $v$ has capacity $\sum_{e \in \delta(v)}u(e)$.
The average cost of a cut around one vertex is
\[
\frac{\sum_{e \in  \delta(v)}u(e)}{n} = 
\frac{2\sum_{e \in E}u(e)}{n} = \frac{2u(E)}{n}.
\]
Then picking an edge from $A$ has lower probability than picking an edge from an average cut around a vertex
\[
\frac{u(A)}{u(E)} \leq \frac{2u(E)}{n \cdot u(E)} = \frac{2}{n}.
\]
After $i$ rounds of the algorithm, $G$ has $n-i$ edges and we get
\[
\frac{u(A)}{u(E)} \leq  \frac{2}{n-i}.
\]
Now the probability that no edge of $A$ was choses is at least 
\[
1-\frac{2}{n-i} = \frac{n-i-2}{n-i}
\]
The algorithm is running for rounds with $i = 0,\ldots,n-2$ and we get
that the probability no edge of $A$ is ever chosen is at least
\[
\frac{n-2}{n} \cdot \frac{n-3}{n-1} \cdot \frac{n-4}{n-2} \cdots \frac{3}{5} \cdot \frac{2}{4} \cdot \frac{1}{3} = \frac{2}{n(n-1)}.
\]
}

\question{}{Let $k \in \mathbb{N}$. Show that the probability that the random contraction algorithm does not return $A$ is one
of $kn^2$ runs is at most $e^{-2k}$.}

\solution{
We use the estimate from previous round $kn^2$ times.
\[
\left(1 - \frac{2}{n(n-1)}\right)^{kn^2} \leq \left( 1 - \frac{2}{n^2}\right)^{kn^2} \leq  \left(e^{-\frac{2}{n^2}} \right)^{kn^2} = e^{-2k}.
\]
}

\end{document}






