\documentclass[11pt]{article}
\usepackage[utf8]{inputenc}
\usepackage[czech, english]{babel}
\usepackage[left=2cm,right=2cm,top=2cm,bottom=2cm]{geometry}
\usepackage{amssymb}
\usepackage{amsthm}
\usepackage{amsmath}
\usepackage{tikz}
\usetikzlibrary{shapes.geometric}
\usepackage{hyperref}


% Itemize environment with small skips
\newenvironment{packeditemize}{
\begin{itemize}
  \setlength{\itemsep}{1pt}
  \setlength{\parskip}{0pt}
  \setlength{\parsep}{0pt}
}{\end{itemize}}


% Fancy footnote....
\usepackage{fancyhdr}
\pagestyle{fancy}
\usepackage{lastpage}
\rfoot{MATH 566 - 09, page \thepage/\pageref{LastPage}}
\cfoot{}
\rhead{}
\lhead{}
\renewcommand{\headrulewidth}{0pt}
\renewcommand{\footrulewidth}{0pt}


\newtheorem{theorem*}{Theorem} 

% Itemize environment with small skips
\newenvironment{packedenumerate}{
\begin{enumerate}
  \setlength{\itemsep}{1pt}
  \setlength{\parskip}{0pt}
  \setlength{\parsep}{0pt}
}{\end{enumerate}}


\newcounter{excounter}
\setcounter{excounter}{1}
\newcommand\question[2]{\vskip 1em  \noindent\textbf{\arabic{excounter}\addtocounter{excounter}{1}:} \emph{#1} \noindent#2}
\newcommand\solution[1]{\vskip 0.5em \noindent\textbf{Solution:} #1}
\newcommand\like{\par \noindent\emph{(This question is: good - bad - ugly? Difficulty: 0-10:\hskip 3em )}} 

\newcommand\lecturer[1]{\textbf{#1}}
\newcommand\hideforstudent[1]{#1}


%\renewcommand\lecturer[1]{{\color{white} \textbf{#1} }}
%\renewcommand\hideforstudent[1]{{\color{white} #1 }}
%\renewcommand\solution[1]{{\color{white} \vskip 0.5em \noindent\textbf{Solution:} #1 }}



\setlength{\parindent}{0cm}
\setlength{\parskip}{0.1cm}

\begin{document}

Fall  2016, MATH-566

\centerline{{\Large \textbf{Simplex method is not polynomial time - Klee-Minty Cube}}}


Source: Chapter 5 of \href{http://www.amazon.com/Nonlinear-Programming-International-Operations-Management/dp/0387745025/ref=sr_1_2?s=books&ie=UTF8&qid=1442802899&sr=1-2&keywords=linear+and+nonlinear+programming}{Linear and Nonlinear Programming, Luenberger and Ye}

Simplex method outline: Convert problem to $A\mathbf{x} = \mathbf{b}$. Find a basic feasible solution. Perform pivot steps until no variable can increase.

Geometry of simplex method - start at vertex of the polytope of feasible  solutions and find a path on edges of the polytope 
to the optimal vertex.

Consider greedy way - always go in the direction that maximizes the slope in the objective function.

% TODO: Make a smaller example just for $x_1$ $x_2$ and then discuss this big ugly one...

\[
(P) \begin{cases}
\text{maximize}  & 100x_1 + 10x_2 + x_3  \\
\text{subject to}  & 
\begin{array}{rcrcrclr} 
      x_1 &   &          &      &         & \leq & 1  &       (A)\\
  20x_1 & + & x_2   &     &         & \leq & 100  &    (B)\\
200x_1 & + &20x_2 & + & x_3  & \leq & 10000  &  (C)\\
      x_1 &     &           &      &         & \geq & 0  &  (D)\\
             &    &      x_2 &     &         & \geq & 0  &  (E)\\
             &   &  &              & x_3  & \geq & 0  &  (F)\\
\end{array}    
\end{cases}
\]

Notice that a vertex is given by intersection of three of the halfspaces. That is, pick three of the equations to be satisfied
with equality and it gives a vertex.

Steps in simplex method:
\begin{center}
\begin{tabular}{cccccc}
step &  $x_1$  &  $x_2$  &  $x_3$  & value of objective &  equalities        \\
0     &      0      &      0        &     0       &         0                    &  $(D),(E),(F)$  \\
1    &       1      &      0        &     0       &        100                  &  $(A),(E),(F)$   \\
2    &       1      &      80        &     0       &       900                  &  $(A),(B),(F)$   \\
3    &       0      &    100        &     0       &      1000                  &  $(D),(B),(F)$   \\
4    &       0      &    100        &   8000    &    9000                  &  $(D),(B),(C)$   \\
5    &       1      &    \hideforstudent{80}  &  \hideforstudent{8200}    &    \hideforstudent{9100}     &  \hideforstudent{$(A),(B),(C)$}   \\
6    &       \hideforstudent{1}      &    \hideforstudent{0}  &  \hideforstudent{9800}    &    \hideforstudent{9900}     &  \hideforstudent{$(A),(E),(C)$}   \\
7    &       \hideforstudent{0}      &    \hideforstudent{0}  &  \hideforstudent{10000}    &    \hideforstudent{10000}     &  \hideforstudent{$(D),(E),(C)$}   \\
\end{tabular}
\end{center}

Corresponds to a travel in cube
\begin{center}
\begin{tikzpicture}[scale=3]
\draw 
(0,0,0) coordinate (a)
(0.5,0.15,1) coordinate (b)
(0.75,0.21,1) coordinate (c)
(          1,0.3,0) coordinate (d)
(          1,0.7,0) coordinate (e)
(0.75,0.79,1) coordinate (f)
(0.5,0.85,1) coordinate (g)
(0,1,0) coordinate (h)
;
\draw[fill=gray] (a) -- (b) -- (g) -- (h) -- cycle;
\draw[fill=gray!50] (b)--(c)--(f)--(g) -- cycle;
\draw[fill=gray!30] (c)--(d)--(e)--(f) -- cycle;
\draw[fill=gray!20] (e)--(f) -- (g) -- (h) --  cycle;
\foreach \x/\y in {a/b,b/c,c/d,d/e,e/f,f/g,g/h}{
\draw[line width=2pt,-latex](\x)--(\y);
}
\draw[dotted]
(0,0,0) -- (1,0.3,0)
;
\draw[line width=2pt, color=white]
(1,0,1) -- (1,1,1) -- (1,1,0) (1,1,1) -- (0,1,1);
;
\draw[dashed]
(0,0,0) -- (1,0,0) -- (1,1,0) -- (1,1,1) -- (1,0,1) -- (0,0,1) -- (0,0,0)
-- (0,1,0)  -- (1,1,0) (1,1,1)--(0,1,1) -- (0,1,0) (0,1,1) -- (0,0,1) (1,0,0) -- (1,0,1);
\end{tikzpicture}
\end{center}

How many vertices will be in $n$ dimensional cube?

\question{If $n=50$ and computer examines one million points in one second, how long will it take
to finish the computation?}

\solution{
$n=50$ gives about $10^{15}$ vertices. It gives 33 years.
}

Klee-Minty cubes are known for different rules too. But the simplex algorithm works great in practice.

\newpage

\centerline{{\Large \textbf{The Ellipsoid Method}}}

Problem: Let $P = \{ \mathbf{x} \in \mathbb{R}^n: A\mathbf{x} \leq \mathbf{b}$\}. Find a point in $P$.
(given a polytope, find one point in it)

Extra assumptions:
\begin{packeditemize}
\item $\exists R \in \mathbb{R}, P \subseteq B(\mathbf{0},R)$
\item $\exists r \in \mathbb{R}, \exists \mathbf{c} \in \mathbb{R}^n,  B(\mathbf{c},r) \subseteq P$
\end{packeditemize}
In other words, $P$ is in a big ball with radius $R$ and contains a small ball of radius $r$.\\
The $R$ and $r$ are part of the running time.

Algorithm:
\begin{packedenumerate}
\item $E_1 := B(0,R)$, $i:=1$
\item if center $\mathbf{y}_i$ of $E_i$ in $P$, point found
\item if  $\mathbf{y}_i \not\in P$, there is a separating hyperplane cutting out half of  $E_i$
\item Pick $E_{i+1}$ to be the smallest ellipsoid containing the half of $E_i$ that contains $P$
\item $i := i +1$ and goto 2.
\end{packedenumerate}

\begin{center}
\begin{tikzpicture}[scale=0.8]
\draw(0,0) circle (2cm);
\draw(-2,-2) node {$E_i$};
\draw(1,0.3)node[regular polygon,regular polygon sides= 5,draw]{P};
\fill(0,0) circle (3pt) node[label=left:{$\mathbf{y_i}$}]{};
\end{tikzpicture}
\hskip 2em
\begin{tikzpicture}[scale=0.8]
\draw(0,0) circle (2cm);
\draw(-2,-2) node {$E_i$};
\draw[fill=gray!20] 
(0,0)--(250:2) arc (250:430:2) -- (0,0);
\fill(0,0) circle (3pt) node[label=left:{$\mathbf{y_i}$}]{};
\draw(0,0)--(70:2.3)(0,0)--(70:-2.3);
\draw(1,0.3)node[regular polygon,regular polygon sides= 5,draw]{P};
\end{tikzpicture}
\hskip 2em
\begin{tikzpicture}[scale=0.8]
\draw(0,0) circle (2cm);
\draw(-2,-2) node {$E_i$};
\draw[fill=gray!20] 
(0,0)--(250:2) arc (250:430:2) -- (0,0);
\fill(0,0) circle (3pt) node[label=left:{$\mathbf{y_i}$}]{};
\draw(0,0)--(70:2.3)(0,0)--(70:-2.3);
\draw(1,0.3)node[regular polygon,regular polygon sides= 5,draw]{P};
\draw (-20:0.8) circle [x radius=2.6cm, y radius=1.2cm, rotate=70];
\draw (2.7,1) node{$E_{i+1}$};
\end{tikzpicture}
\end{center}

\textbf{Claim:}
If $E_i \in \mathbb{R}^n$ and $E_{i+1}$ is the smallest ellipsoid contain $\frac{1}{2}$ of $E_i$, then
\[
\frac{volume(E_{i+1})}{volume(E_i)}  < e^{\frac{-1}{2(n+1)}}  < 1.
 \]

\question{
Compute an upper bound on
\[
\frac{volume(E_{i+2(n+1)})}{volume(E_i)} 
\]
}
\solution{
\[
\frac{volume(E_{i+2(n+1)})}{volume(E_i)} < \left(e^{\frac{-1}{2(n+1)}}\right)^{2(n+1)} = e^{-1}.
\]
}
\question{
How many iterations of the algorithm are needed? (Use that $B(\mathbf{c},r)\subset P$.)
}
\solution
{
If volume of $volume(E_i) < volume(B(\mathbf{c},r))$, we would get a contradiction
since $B(\mathbf{c},r) \subset  P \subset E_i$. 
We need $O(n)$ steps to reduce volume by half and we do it $\log\left( \frac{volume(B,R)}{volume(B,r)}\right)$
times.
\[
O\left(n \cdot \log\left( \frac{R^n}{r^n}  \right) \right) = O\left(n^2 \log\left(\frac{R}{r}\right)\right)
\]
}
One iteration takes $O(n^2)$ operations and volume of balls is at most exponential (in size of
input numbers). \\
Not a practical algorithm in speed.

\end{document}



