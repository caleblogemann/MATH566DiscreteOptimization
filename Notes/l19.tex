\documentclass[11pt]{article}
\usepackage[utf8]{inputenc}
\usepackage[czech, english]{babel}
\usepackage[left=2cm,right=2cm,top=2cm,bottom=2cm]{geometry}
\usepackage{amssymb}
\usepackage{amsthm}
\usepackage{amsmath}
\usepackage{tikz}
\usetikzlibrary{shapes.geometric}
\usetikzlibrary{shapes.multipart}
\usetikzlibrary{arrows}
\usepackage{hyperref}


% Itemize environment with small skips
\newenvironment{packeditemize}{
\begin{itemize}
  \setlength{\itemsep}{1pt}
  \setlength{\parskip}{0pt}
  \setlength{\parsep}{0pt}
}{\end{itemize}}


% Fancy footnote....
\usepackage{fancyhdr}
\pagestyle{fancy}
\usepackage{lastpage}
\rfoot{MATH 566 - 19, page \thepage/\pageref{LastPage}}
\cfoot{}
\rhead{}
\lhead{}
\renewcommand{\headrulewidth}{0pt}
\renewcommand{\footrulewidth}{0pt}


\newtheorem{theorem*}{Theorem} 

% Itemize environment with small skips
\newenvironment{packedenumerate}{
\begin{enumerate}
  \setlength{\itemsep}{1pt}
  \setlength{\parskip}{0pt}
  \setlength{\parsep}{0pt}
}{\end{enumerate}}


\newcommand{\vc}[1]{\ensuremath{\vcenter{\hbox{#1}}}}

\newcounter{excounter}
\setcounter{excounter}{1}
\newcommand\question[2]{\vskip 1em  \noindent\textbf{\arabic{excounter}\addtocounter{excounter}{1}:} \emph{#1} \noindent#2}
\newcommand\solution[1]{\vskip 0.5em \noindent\textbf{Solution:} #1}
%\newcommand\like{\par \noindent\emph{(This question is: good - bad - ugly? Difficulty: 0-10:\hskip 3em )}} 

\newcommand\lecturer[1]{\textbf{#1}}
\newcommand\hideforstudent[1]{#1}


%\renewcommand\lecturer[1]{{\color{white} \textbf{#1} }}
%\renewcommand\hideforstudent[1]{{\color{white} #1 }}
%\renewcommand\solution[1]{{\color{white} \vskip 0.5em \noindent\textbf{Solution:} #1 }}



\setlength{\parindent}{0cm}
\setlength{\parskip}{0.1cm}

\begin{document}

Fall  2016, MATH-566

\centerline{{\Large \textbf{Minimum Cost Flow}}}

Problem: There are $n$ coal mines and $m$ power plants.
Power plants have demands, coal mines supply coal. How to
transport coal in order to satisfy the demands and minimize 
cost of transportation.



Let $G=(V,E)$ be a directed graph,
$u: E \rightarrow \mathbb{R}^+$ be capacities on edges
and $c: E \rightarrow \mathbb{R}$ be costs for every edge.

Let $b: V \rightarrow \mathbb{R}$ with $\sum_vb(v)=0$ be a \emph{supply demand} function. Called \emph{boundary}.

\textbf{b-flow} is $f: E \rightarrow \mathbb{R}^+$ such that $f(e) \leq u(e)$
and  $\sum_{e\in\delta^+(v)}f(e)  - \sum_{e\in\delta^-(v)}f(e) = b(v)$.

\question{}{
Find a $b$-flow (that minimizes $\sum_{e}c(e)f(e)$) in the following network:
($b$ is in every vertex, edges have $c,u$).
\begin{center}
\begin{tikzpicture}[xscale=2]
\draw 
(0,0) node[draw,circle](x){$-4$}
(4,0) node[draw,circle](y){$2$}
(0,4) node[draw,circle](z){$1$}
(4,4) node[draw,circle](w){$-4$}
(2,1) node[draw,circle](a){$5$}
(1,2) node[draw,circle](b){$-4$}
(2,3) node[draw,circle](c){$4$}
(3,2) node[draw,circle](d){$0$}
;
\foreach \x/\y/\l in {y/x/{5,4} , z/b/{-2,2}, d/w/{2,3},b/c/{2,4},c/d/{4,6},a/b/{-3,2},a/d/{3,2},
c/z/{6,3}, w/c/{5,2}  }{
\draw[-triangle 45](\x)--node[below]{\l}(\y);
}
\foreach \x/\y/\l in {z/w/{-2,4}, a/x/{0,3}, a/y/{2,3}  }{
\draw[-triangle 45](\x)--node[above]{\l}(\y);
}
\foreach \x/\y/\l in {x/z/{6,2} }{
\draw[-triangle 45](\x)--node[left]{\l}(\y);
}
\foreach \x/\y/\l in {y/w/{0,4} }{
\draw[-triangle 45](\x)--node[right]{\l}(\y);
}
\end{tikzpicture}
\end{center}
}

If $b(v) > 0$, then $b$ is \emph{supply}, if $b(v) < 0$, then $b$ is \emph{demand}.
Like flows but multiple sources and sinks.

\textbf{Minimum Cost Flow Problem:}
find a $b$-flow $f$ that  minimizes $c(f) = \sum_{e}c(e)f(e)$.


\question{}{
Show that $b$-flow $f$ exists iff 
\[
\sum_{e \in \delta^+(X)} u(e) \geq \sum_{v\in X} b(v) \text{ for all } X \subseteq V(G).
\]
(That is, there is always enough capacity to take excessive flow out of $X$.)
}
\solution{
$\Rightarrow$ If there is a flow, the condition is easily satisfied, since $f(e) \leq u(e)$.

$\Leftarrow$ Suppose there is no $b$-flow $f$. Add new vertices $s$ and $t$, where
$sv$ is edge for every $v\in V$ with $b(v) > 0$. Assign $u(sv) = b(v)$. Do analogous operation with $t$. No  $b$-flow implies no flow in the new graph. Hence there is a cut.
The cut gives $X$ that violates the big condition.
}

Consequence: It is possible to detect no solution case.

\emph{Circulation} is a flow in a network where $b(v)=0$ for every vertex.


\question{}{
Let $f$ and $f'$ be two $b$-flows. Consider their \emph{difference} $f-f'$
and show that it is a circulation.\\
Try on example first: Edge labels are $c,\frac{f}{f'}$. Compute $c(f)$, $c(f')$, find what is the difference.
\begin{center}
\begin{tikzpicture}[xscale=1]
\draw 
(0,0) node[draw,circle](s){$4$}
(2,0) node[draw,circle](y){$0$}
(4,1) node[draw,circle](a){$1$}
(4,-1) node[draw,circle](b){$-1$}
(6,0) node[draw,circle](t){$-4$}
;
\foreach \x/\y/\l in {y/b/{5,$\frac{1}{3}$},  b/t/{4,$\frac{0}{2}$}   }{
\draw[-triangle 45](\x)--node[below]{\l}(\y);
}
\foreach \x/\y/\l in {s/y/{$-2,\frac{4}{4}$}, y/a/{0,$\frac{3}{1}$}, a/t/{$2,\frac{4}{2}$}  }{
\draw[-triangle 45](\x)--node[above]{\l}(\y);
}
\end{tikzpicture}
\end{center}
}
\solution{
$c(f) = 5$, $c(f')= 19$. Cost difference $c(f) - c(f') = -14$.
In picture, we assign the difference of the flows, if the difference
was negative, flip the edge and count negative cost.
The picture gives a circulation with cost $-14$.
\begin{center}
\begin{tikzpicture}[xscale=1]
\draw 
(0,0) node[draw,circle](s){$0$}
(2,0) node[draw,circle](y){$0$}
(4,1) node[draw,circle](a){$0$}
(4,-1) node[draw,circle](b){$0$}
(6,0) node[draw,circle](t){$0$}
;
\foreach \x/\y/\l in {b/y/{-5,$2$},  t/b/{-4,$2$}   }{
\draw[-triangle 45](\x)--node[below]{\l}(\y);
}
\foreach \x/\y/\l in {s/y/{$-2,0$}, y/a/{0,$2$}, a/t/{$2,2$}  }{
\draw[-triangle 45](\x)--node[above]{\l}(\y);
}
\end{tikzpicture}
\end{center}
General: Difference between two flows is a circulation,
its cost is the difference in the costs of $f$ and $f'$.
Since $f$ and $f'$ are $b$-flows, we get
$\sum_{e\in\delta^+(v)}(f(e)-f'(e))  - \sum_{e\in\delta^-(v)}(f(e)-f'(e)) = b(v) - b(v) = 0,$
Hence the difference is a circulation.
Summing the difference gives the difference in costs.
}




\textbf{Algorithm Minimum Cost Flow}:
\begin{enumerate}
\item $f$ be any $b$-flow
\item while exists negative cost cycle $C$ in residual graph
\item \hskip 4em pick $C$ of minimum mean cost $= \frac{\sum_{e\in C} c(e)}{|C|}$.
\item \hskip 4em augment on $C$
\end{enumerate}
Minimum mean cost cycle gives polynomial time $O(m^2n^2\log n)$ (picking any cycle - same problem as Ford-Fulkerson).



\question{}{ Run the algorithm on
\begin{center}
\begin{tikzpicture}[xscale=2]
\draw 
(0,0) node[draw,circle](x){$-4$}
(4,0) node[draw,circle](y){$2$}
(0,4) node[draw,circle](z){$1$}
(4,4) node[draw,circle](w){$-4$}
(2,1) node[draw,circle](a){$5$}
(1,2) node[draw,circle](b){$-4$}
(2,3) node[draw,circle](c){$4$}
(3,2) node[draw,circle](d){$0$}
;
\foreach \x/\y/\l in {y/x/{5,4} , z/b/{-2,2}, d/w/{2,3},b/c/{2,4},c/d/{4,6},a/b/{-3,2},a/d/{3,2},
c/z/{6,3}, w/c/{5,2}  }{
\draw[-triangle 45](\x)--node[below]{\l}(\y);
}
\foreach \x/\y/\l in {z/w/{-2,4}, a/x/{0,3}, a/y/{2,3}  }{
\draw[-triangle 45](\x)--node[above]{\l}(\y);
}
\foreach \x/\y/\l in {x/z/{6,2} }{
\draw[-triangle 45](\x)--node[left]{\l}(\y);
}
\foreach \x/\y/\l in {y/w/{0,4} }{
\draw[-triangle 45](\x)--node[right]{\l}(\y);
}
\end{tikzpicture}
\end{center}
}



\question{}{
Show that the algorithm is correct when it finishes.
That is, $f$ is an optimal $b$-flow iff it has no negative cycle.
}
\solution{
$\Rightarrow$
If there is a negative cycle, augmenting on it, it will
decrease the cost of $f$, contradiction with optimality of $f$.

$\Leftarrow$ If there is a better flow $f'$, the difference between
flows is a circulation. Since the sum of the circulation is negative,
it must contain a negative circuit.
}

\question{}{
How to find minimum mean cycle?
}


%\vfill

%\emph{Next time: Minimum Mean Cycle and  Integer Programming}

\end{document}






