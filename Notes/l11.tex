\documentclass[11pt]{article}
\usepackage[utf8]{inputenc}
\usepackage[czech, english]{babel}
\usepackage[left=2cm,right=2cm,top=2cm,bottom=2cm]{geometry}
\usepackage{amssymb}
\usepackage{amsthm}
\usepackage{amsmath}
\usepackage{tikz}
\usetikzlibrary{shapes.geometric}
\usepackage{hyperref}


% Itemize environment with small skips
\newenvironment{packeditemize}{
\begin{itemize}
  \setlength{\itemsep}{1pt}
  \setlength{\parskip}{0pt}
  \setlength{\parsep}{0pt}
}{\end{itemize}}


% Fancy footnote....
\usepackage{fancyhdr}
\pagestyle{fancy}
\usepackage{lastpage}
\rfoot{MATH 566 - 11, page \thepage/\pageref{LastPage}}
\cfoot{}
\rhead{}
\lhead{}
\renewcommand{\headrulewidth}{0pt}
\renewcommand{\footrulewidth}{0pt}


\newtheorem{theorem*}{Theorem} 

% Itemize environment with small skips
\newenvironment{packedenumerate}{
\begin{enumerate}
  \setlength{\itemsep}{1pt}
  \setlength{\parskip}{0pt}
  \setlength{\parsep}{0pt}
}{\end{enumerate}}


\newcounter{excounter}
\setcounter{excounter}{1}
\newcommand\question[2]{\vskip 1em  \noindent\textbf{\arabic{excounter}\addtocounter{excounter}{1}:} \emph{#1} \noindent#2}
\newcommand\solution[1]{\vskip 0.5em \noindent\textbf{Solution:} #1}
\newcommand\like{\par \noindent\emph{(This question is: good - bad - ugly? Difficulty: 0-10:\hskip 3em )}} 

\newcommand\lecturer[1]{\textbf{#1}}
\newcommand\hideforstudent[1]{#1}


%\renewcommand\lecturer[1]{{\color{white} \textbf{#1} }}
%\renewcommand\hideforstudent[1]{{\color{white} #1 }}
%\renewcommand\solution[1]{{\color{white} \vskip 0.5em \noindent\textbf{Solution:} #1 }}



\setlength{\parindent}{0cm}
\setlength{\parskip}{0.1cm}

\begin{document}

Fall  2016, MATH-566

\centerline{{\Large \textbf{Spanning Trees}}}


Source: Chapter 2.1 (Bills), Chapter 6.1 of Combinatorial Optimization (Korte)

Problem: Connect cities $V$ with optic cable. For every pair of cities, it is known if the cable can be built and the cost of building it: $c: V^2 \rightarrow \mathbb{R}$. 
Which connections to build to minimize the building cost and make the network connected? (no isolated city)

\question{}{
Solve the cities and cable problem for the following diagram of cities:
\begin{center}
\begin{tikzpicture}[scale=2]
\draw (0,0) grid (3,2);
\fill
\foreach \x in {0,1,2,3}{
\foreach \y in {0,1,2}{ 
(\x,\y) circle (2pt)
}
}
;
\draw 
(0.5,0) node[below]{1}
(1.5,0) node[below]{2}
(2.5,0) node[below]{5}
(0.5,1) node[below]{6}
(1.5,1) node[below]{3}
(2.5,1) node[below]{9}
(0.5,2) node[below]{11}
(1.5,2) node[below]{7}
(2.5,2) node[below]{19}
(0,0.5) node[left]{13}
(0,1.5) node[left]{15}
(1,0.5) node[left]{22}
(1,1.5) node[left]{10}
(2,0.5) node[left]{14}
(2,1.5) node[left]{17}
(3,0.5) node[left]{12}
(3,1.5) node[left]{4}
;
\end{tikzpicture}
\end{center}
}


A \emph{graph} $G=(V,E)$ is a pair of vertices $V$ and edges $E$, where $E$ consists of pairs of vertices.

Recall definitions of \emph{circuit/cycle}, \emph{tree}, \emph{forest}, \emph{spanning tree}, \emph{connected components}, \emph{path}

\textbf{Cut} for $X\subset V$ is the set of edges with exactly one endpoint in $X$.

Formal definition of our problem:
\textbf{Minimum spanning tree problem}\\
\emph{Input:}     Graph $G=(V,E)$ and costs $c: E \rightarrow \mathbb{R}$.\\
\emph{Output:}  Spanning tree $T$ of minimum cost.

\question{}{
If $T$ is a spanning, then the following are equivalent:
\begin{enumerate}
\item $T$ is minimum spanning tree
\item For every $e=\{x,y\} \in E(G) \setminus E(T)$, no edge on the $x$-$y$-path in $T$ has higher cost than $e$.
\item For every $e\in E(T)$, $e$ is a minimum cost edge of the \emph{cut} between the connected components in $T-e$
\item We can order $E(T) = \{e_1,\ldots,e_{n-1}\}$ such that for each $i$ there exists a set $X_i\subset V(G)$ such that $e_i$
is the min cost edge of cut $X_i$ and no previous edge is in the cut $X_i$.
\end{enumerate}
Show $1 \rightarrow 2 \rightarrow 3 \rightarrow 4$.
Try $4 \rightarrow 1$ by taking $T$ that satisfies 4 and $T^\star$ satisfying (1) and check how they
can differ.
}
\solution{
See the book(s) for detailed solution.\\
(1) $\rightarrow$ (2): If 2 violated, $T$ was not optimal by replacing an edge \\
(2) $\rightarrow$ (3): if 3 violated, so is (2)\\
(3) $\rightarrow$ (4): take any ordering from (3), it gives (4).
(4) $\rightarrow$ (1): $T$ from (4) and $T^\star$ optimum. Let $e_i$ be the first
$e_i \in T$, that is missing in $T^\star$.  Let the corresponding cut for $e_i$ be $X_i$.
Add $e_i$ to $T^\star$, it contains a circuit, one other edge of the cut $X_i$ that is in $T^\star$ can be removed
and cost of $T^\star$ decreases. 
}

\newpage

\textbf{Kruskal's (greedy) algorithm [1956]}
\begin{enumerate}
\item sort edges of $G$ such that $c(e_1) \leq c(e_2) \leq \cdots \leq c(m)$
\item set $T=(V,\emptyset)$
\item for $i$ in $1$ to $m$:\\ if $T+e_i$ does not contain a circuit, then $T := T+e$.
\end{enumerate}
\question{}{
Do steps of the algorithm on the graph with cities (note that the algorithm has 11 iterations where edge is added since the tree has 11 edges).
Denote the order of edges as they enter the spanning three.
\begin{center}
\begin{tikzpicture}[scale=1.5]
\draw (0,0) grid (3,2);
\fill
\foreach \x in {0,1,2,3}{
\foreach \y in {0,1,2}{ 
(\x,\y) circle (2pt)
}
}
;
\end{tikzpicture}
\end{center}
}
\question{}{
Why is the output of the algorithm correct?
}
\solution{
Satisfies condition 2.
}


\textbf{Jarn\'{i}k's [1930] and Prims [1957] algorithm}
\begin{enumerate}
\item choose any $v \in V$ and $T=(\{v\},\emptyset)$
\item while $T$ does not contain all vertices: \\
         pick $e$ of minimum cost that has exactly one endpoint in $T$ and $T := T+e$
\end{enumerate}
\question{}{
Do steps of the algorithm on the graph with cities (note that the algorithm has 11 iterations since the tree has 11 edges).
Denote the order of edges as they enter the spanning three. Start with $v$ being the left top vertex.
\begin{center}
\begin{tikzpicture}[scale=1.5]
\draw (0,0) grid (3,2);
\fill
\foreach \x in {0,1,2,3}{
\foreach \y in {0,1,2}{ 
(\x,\y) circle (2pt)
}
}
;
\end{tikzpicture}
\end{center}
}
\question{}{
Why is the output of the algorithm correct?
}
\solution{
Satisfies condition 4.
}

\textbf{Bor\r{u}vka's  [1928] algorithm}
\begin{enumerate}
\item Let $T=(V,\emptyset)$
\item while $T$ has more than one connected component: \\
         in parallel, for every connected component $C$ in $T$, pick $e$ of minimum cost that has exactly one endpoint in $C$ and do $T := T+e$
\end{enumerate}
\question{}{
Do steps of the algorithm on the graph with cities. Note that one iteration always gives several edges in. The number of iterations is not clear at the beginning.
Denote the order of edges as they enter the spanning three.
\begin{center}
\begin{tikzpicture}[scale=1.5]
\draw (0,0) grid (3,2);
\fill
\foreach \x in {0,1,2,3}{
\foreach \y in {0,1,2}{ 
(\x,\y) circle (2pt)
}
}
;
\end{tikzpicture}
\end{center}
}
\question{}{
Why is the output of the algorithm correct?
}
\solution{
Suppose it creates a cycle. The we get  a contradiction with the choice of the edges.
But we run into troubles if edges have the same weights.
}

Algorithmic note: Which algorithm is fastest?

Complexity of algorithm counted in number of operations of CPU.

Count how many times every vertex and edge is used, constant does not matter, we use $O(.)$ notation.

Simple Kruskal's complexity for graph with $m$ edges and $n$ vertices:
\begin{packeditemize}
\item sorting takes $O(m \log(m))$
\item for cycles is done $m$ times
\item test for circuit can be done in $O(n)$ time
\end{packeditemize}
Total time: $O(m \log(m) + mn)$. Better implementation can do $O(m \log(n))$.

Jarn\'{\i}k's algorithm can be implemented in $O(m + n \log(n) )$.

More effective algorithms exists if weights are integers, graph is \emph{planar}, \ldots.
 

\question{}{
Let $G=(V,E)$ be a a graph and $c: E \rightarrow \mathbb{R}$ cost function on the edges.
Formulate the minimum spanning tree problem using linear programming. \\
\emph{(Don't be afraid that there are many constraints. Try to make constraint that graph has no cycles.)}
}
\solution{
We let every edge $e_i$ be a variable $x_i$. 
The objective is to minimize  $c_ix_i$. For every  $x_i$ we add $0 \leq x_i \leq 1$.
We also can add $\sum x_i = n-1$. But mainly we need to avoid cycles.
Description of the set of feasible solutions is
\[
F=\{
x \in [0,1]^E: \sum_{e \in E(G[X])} x_e \leq |X|-1 \text{ for } \emptyset \neq X \subseteq V
\},
\]
where $G[X]$ is the graph induced by vertices $X$ - contains all edges of $E$ that have both vertices in $X$.
}

\textbf{Theorem} [Edmonds 1970]
The set of feasible solutions is \emph{integral} polytope - i.e. - every vertex of the polytope has all coordinates integers. The polytope is called $\textbf{spanning tree polytope}$.

\question{}{
Draw the spanning tree polytope for $K_3$, where $K_3$ is the complete graph on $3$ vertices.
}
\solution{
Convex hull of points $\{1,1,0\}, \{0,1,1\}, \{1,0,1\}$. Notice it is two dimensional.
}



% - there are 947 cities in Iowa. The smallest city has 15 residents.)




\end{document}






