\documentclass[11pt]{article}
\usepackage[utf8]{inputenc}
\usepackage[czech, english]{babel}
\usepackage[left=2cm,right=2cm,top=2cm,bottom=2cm]{geometry}
\usepackage{amssymb}
\usepackage{amsthm}
\usepackage{amsmath}
\usepackage{mathtools}
\usepackage{tikz}
\usetikzlibrary{shapes.geometric}
\usetikzlibrary{shapes.multipart}
\usetikzlibrary{arrows}
\usetikzlibrary{graphs}
\usepackage{hyperref}


% Itemize environment with small skips
\newenvironment{packeditemize}{
\begin{itemize}
  \setlength{\itemsep}{1pt}
  \setlength{\parskip}{0pt}
  \setlength{\parsep}{0pt}
}{\end{itemize}}


% Fancy footnote....
\usepackage{fancyhdr}
\pagestyle{fancy}
\usepackage{lastpage}
\rfoot{MATH 566 - 26, page \thepage/\pageref{LastPage}}
\cfoot{}
\rhead{}
\lhead{}
\renewcommand{\headrulewidth}{0pt}
\renewcommand{\footrulewidth}{0pt}


\newtheorem{theorem*}{Theorem} 

% Itemize environment with small skips
\newenvironment{packedenumerate}{
\begin{enumerate}
  \setlength{\itemsep}{1pt}
  \setlength{\parskip}{0pt}
  \setlength{\parsep}{0pt}
}{\end{enumerate}}


\newcommand{\vc}[1]{\ensuremath{\vcenter{\hbox{#1}}}}

\newcounter{excounter}
\setcounter{excounter}{1}
\newcommand\question[2]{\vskip 1em  \noindent\textbf{\arabic{excounter}\addtocounter{excounter}{1}:} \emph{#1} \noindent#2}
\newcommand\solution[1]{\vskip 0.5em \noindent\textbf{Solution:} #1}
%\newcommand\like{\par \noindent\emph{(This question is: good - bad - ugly? Difficulty: 0-10:\hskip 3em )}} 

\newcommand\lecturer[1]{\textbf{#1}}
\newcommand\hideforstudent[1]{#1}


\renewcommand\lecturer[1]{{\color{white} \textbf{#1} }}
\renewcommand\hideforstudent[1]{{\color{white}  }}
\renewcommand\solution[1]{{\color{white} \vskip 0.5em \noindent\textbf{Solution:} #1 }}



\setlength{\parindent}{0cm}
\setlength{\parskip}{0.1cm}

\begin{document}

Fall  2016, MATH-566

\centerline{{\Large \textbf{Minimum-Weight Perfect Matching for Point in Plane}}}

Source: Bill,Bill,Bill

Goal: Find a minimum-weight perfect matching algorithm that works for points in the plane. 
Plan is to find a relaxation of the integer program and guide us to build an algorithm by cleverly
interpreting the dual solution.

First we write a good linear program for general matching problem.

Recall that Minimum-weight perfect matching problem can be formulated as an integer programming problem
in the following way
\[
(IP)
\begin{cases}
\text{minimize}  &  \sum_{e \in E} c(e) x_e \\
\text{subject to} & \sum_{e \in \delta(v)} x_e =  1 \text{ for all } v \in V\\
                          &  \mathbf{x} \in \{0,1\}^{|E|},
\end{cases}
\]

\question{}{
Show that a relaxation of $(IP)$ to linear program may result in optimal solution
that is not realizable by a perfect matching. You need to cleverly assign weights!

\begin{center}
\begin{tikzpicture}[scale=1.2]
\tikzset{insep/.style={inner sep=2pt, outer sep=0pt, circle, fill},}
\draw  (0,0) node[insep](a){}  -- +(45:1) node[insep]{}  -- +(-45:1) node[insep]{} -- (a);
\draw  (-2,0) node[insep](b){}  -- +(45:-1) node[insep]{}  -- +(-45:-1) node[insep]{} -- (b);
\draw (a)--(b);
\end{tikzpicture}
\end{center}
}



\question{}{
Write a better program $(P)$ that prevents issue from the previous figure and
its dual $(D)$. 
(Hint: Let $\mathcal C$ be the set of all odd cuts as set of edges and use $\mathcal{C}$. 
}
\solution{
We include constraint that for every odd cut, the edges going across sum to at least one.
A cut is \emph{odd} if it contains odd number of vertices on each side.
Let $\mathcal{C}$ be the set of all odd cuts, where we consider as a cut the set of edges.
\begin{align*}
(P)
\begin{cases}
\text{minimize}  &  \sum_{e \in E} c(e) x_e \\
\text{subject to} & \sum_{e \in \delta(v)} x_e =  1 \text{ for all } v \in V\\
                         & \sum_{e \in D} x_e \geq 1 \text{ for all } D \in \mathcal{C}\\
                          & x_e \geq 0 \text{ for all } e \in E
\end{cases}
&&
(D)
\begin{cases}
\text{maximize}  &  \sum_{v \in V} y_v \\
\text{subject to} & y_{u}  + y_v + \sum_{uv \in D \in \mathcal C}Y_D \leq  c(uv) \text{ for all } uv \in E\\
                          & y_v \in \mathbb{R} \text{ for all } v \in V\\
                          & Y_D \geq 0 \text{ for all } D \in \mathcal{C}
\end{cases}
\end{align*}
Notice that the bad solution from the previous relaxation is gone.
}


\textbf{Theorem} Edmonds: $G$ has a perfect matching iff $(P)$ has a feasible solution.
Moreover, the minimum weight of the perfect matching is equal to the value of optimal solution to $(P)$.


\question{}{
Write complementary slackness conditions for $(P)$.
}

\solution{\\
 $x_e > 0$ implies $y_e  + y_e + \sum_{e \in D \in \mathcal C}Y_D =  c(e)$\\
 $Y_D > 0$ implies $\sum_{e \in D} x_e = 1 \text{ for all } D$
}

\newpage

A family of sets $\mathcal{A}$ is \emph{nested} if for any $A,B \in \mathcal{A}$ exactly one of $A \cap B = \emptyset$, $ A \subseteq B$, and $B \subseteq A$ holds.

A solution to  $(D)$ is \emph{nested} if the family of $D$s corresponding to $Y_D > 0$ is nested.

\textbf{Theorem 5.17}
If an optimal solution to $(D)$ exists, then there exists a nested one.


If $c$ satisfies triangle-inequality, then the dual has a nice solution.

\textbf{Theorem 5.20}
Let $G$ be a complete graph having even number of nodes and $c \geq 0$ satisfy the triangle inequality,
then there exists an optimal solution to the dual with $y \geq 0$.

\vskip 2em
Finally, we are ready for some geometric ideas behind the algorithm that is on the cover of the textbook!
\vskip 2em

Let $V$ be points in the plane and let $c:uv \rightarrow \mathbb{R}^+$ the Euclidean distance of $u$ and $v$.
Find a perfect matching $M$ that is minimizing the sum of costs of the edges in the matching.
Let $E$ be the set of all pairs of vertices.

Notice this problem satisfies the triangle inequality.


\question{}{
Suppose $|V|=n$ and every point has  a disk of radius 1. Assume these disks are disjoint. Can you give a lower bound on the cost of a perfect matching?
\begin{center}
\begin{tikzpicture}[scale=1.2]
\tikzset{insep/.style={inner sep=2pt, outer sep=0pt, circle, fill},}
\draw  (0,0) node[insep](a){} circle(0.5 cm);
\draw  (1,0) node[insep](a){} circle(0.5 cm);
\draw  (2,0) node[insep](a){} circle(0.5 cm);
\draw  (4,0) node[insep](a){} circle(0.5 cm);
\end{tikzpicture}
\end{center}
}
\solution{
The lower bound is $n$ as for every edge in the matching, it must go trough a radius of two circles.
}
\question{}{
Consider the following extension, where big white disk has radius 2 and the gray area is actually union of three disk of  radius 2 without white disk of radius one. Can you provide a lower bound on the cost of a perfect matching?
\begin{center}
\begin{tikzpicture}[scale=1.2]
\tikzset{insep/.style={inner sep=2pt, outer sep=0pt, circle, fill=black},}
\draw[fill=white]  (0,0) node[insep](a){} circle(0.5 cm);
\foreach \x in {0,1,2}{
\fill[fill=gray!80]  (\x,0) circle(1 cm);
}
\foreach \x in {0,1,2}{
\draw[fill=white]  (\x,0) node[insep](a){} circle(0.5 cm);
}
\draw  (4,0) node[insep](a){} circle(1 cm);
\end{tikzpicture}
\end{center}
}
\solution{
Lower bound would be $6$. This comes from $1+1+1+2$ for the radius of the disks.
Then additional $+1$ comes from the fact that they gray area encloses an odd number of points. So at least one edge of the perfect matching must pass through the gray area.
}

We call the white disks around vertices \emph{control zones}.
A pair of compact sets  $(N,I)$ is a \emph{moat} if 
\[
I \subset N, |I \cap V| \text{ is odd, and } N \setminus \text{ interior}(I) \text{ contains no points in } V.
\]
Example of a moat is the pair of gray set as $N$ and the three white disks inside as $I$.

Notice we could interpret $y_v$ as a radius of the control zone and $Y_D$ as a width of a moat.
This is a possible interpretation of a  dual solution to $D$ - try example, where you find optimal matching and corresponding control zones and moates.

\begin{center}
\begin{tikzpicture}[scale=1]
\tikzset{insep/.style={inner sep=2pt, outer sep=0pt, circle, fill},}
\draw  (0,0) node[insep](a){}   +(45:1) node[insep]{}  --  +(-45:1) node[insep]{}  (a);
\draw  (-2,0) node[insep](b){}   +(45:-1) node[insep]{}  -- +(-45:-1) node[insep]{}  (b);
\draw (a)--(b);
\end{tikzpicture}
\end{center}


\newpage
\textbf{Goal:} Algorithm that finds a perfect matching of cost at most twice of the optimum solution.\\

Algorithm outline:\\
1) Obtain a forest $F$, where every component has even number of vertices. So called \emph{even forest}.\\
2) Transform the forest $F$ into a perfect matching $M$ such that the cost of used
     edges does not increase.

In order to provide a bound on the cost of the resulting matching $M$, we build a forest $F^\star$ whose
cost would be at most twice the lower bound  obtained from control zones and moats.

\textbf{Definition}
Let $C$ be a set. Define
\[
parity(C) = \begin{cases} 0 & \text{ if } |C| \text{ even} \\  1 & \text{ if } |C| \text{ odd.}  \end{cases}
\]

Let $\overline{c}_e = c(uv) -  y_{u}  + y_v + \sum_{uv \in D \in \mathcal C}Y_D$.
(Slack in the constraints in $D$.)
\vskip 1em

\textbf{Goemans-Williamson algorithm} (sketch)\\
Goal: Find an even forest $F$ and a feasible solution to dual program $(D)$.
\begin{enumerate}
\item\hskip 0em  $\mathcal{C} = \{ \{v\}: v \in V\}$;  $F=\emptyset$; $y = 0$; $Y = 0$
\item\hskip 0em while exists $C \in \mathcal{C}$ with $|C|$ odd 
\item\hskip 4em  Find an edge $e = uv$, with $u \in C_i$ and $v \in C_j$, $C_i \neq C_j$, that minimizes $\varepsilon = \overline{c}_e/(parity(C_i)+parity(C_j))$. 
Notice at least one of $|C_i|$ and $|C_j|$ is odd.
\item \hskip 4em For all $C\in \mathcal{C}$ where $C=\{x\}$, add $\varepsilon$ to $y_x$.
\item \hskip 4em For all $C\in \mathcal{C}$ where $|C| > 1$ and $|C|$ is odd, add $\varepsilon$ to $Y_C$.
\item \hskip 4em  add $e$  to $F$ and replace $C_i$ and $C_j$ by $C_i \cup C_j$.
\end{enumerate}


Example:

\begin{center}
\def\nodes{
\draw 
(0,0) node[insep](a){}
(a) ++(135:1.5) node[insep](b){}
(b) ++(220:1) node[insep](c){}
(a) ++(20:2.5) node[insep](d){}
;
}
\begin{tikzpicture}[scale=1]
\tikzset{insep/.style={inner sep=2pt, outer sep=0pt, circle, fill=black},}
\nodes
\end{tikzpicture}
\hskip 8em
\begin{tikzpicture}[scale=1]
\tikzset{insep/.style={inner sep=2pt, outer sep=0pt, circle, fill=black},}
\nodes
\fill [gray!20]
(a) circle (0.5)
(b) circle (0.5)
(c) circle (0.5)
(d) circle (0.5)
;
\draw (b)--(c);
\nodes
\end{tikzpicture}
\vskip 4em
\begin{tikzpicture}[scale=1]
\tikzset{insep/.style={inner sep=2pt, outer sep=0pt, circle, fill=black},}
\nodes
\fill [gray!20]
(a) circle (1)
(b) circle (0.5)
(c) circle (0.5)
(d) circle (1)
;
\draw (b)--(c) (a)--(b);
\nodes
\end{tikzpicture}
\hskip 8em
\begin{tikzpicture}[scale=1]
\tikzset{insep/.style={inner sep=2pt, outer sep=0pt, circle, fill=black},}
\nodes
\fill [gray!40]
(a) circle (1.25)
(b) circle (0.75)
(c) circle (0.75)
;
\fill [gray!20]
(a) circle (1)
(b) circle (0.5)
(c) circle (0.5)
(d) circle (1.25)
;
\draw (b)--(c) (a)--(b) (a)--(d);
\nodes
\end{tikzpicture}


\end{center}


\newpage
\question{}{
Try the algorithm on the following example:
\begin{center}
\begin{tikzpicture}[scale=1]
\tikzset{insep/.style={inner sep=2pt, outer sep=0pt, circle, fill},}
\draw  (0,0) node[insep](a){}   +(45:1) node[insep]{}   +(-45:1) node[insep]{}  (a);
\draw  (-2,0) node[insep](b){}   +(45:-1) node[insep]{}   +(-45:-1) node[insep]{}  (b);
\draw (a) (b);
\end{tikzpicture}
\end{center}

}

\question{}{
Show that there exists a matching $M$ such that 
$\sum_{e\in M} c(e)   \leq \sum_{e\in F} c(e)$.
}
\solution{
Use example and previous question to demonstrate the steps!\\
We provide an algorithmic way and use two steps\\
(a) If there are is an edge $e$ in $F$ such that $F-e$ is still an even forest, remove $e$.\\
(b) Otherwise consider $F'$, which is $F$ without leaves. Let $v$ be a leaf in $F'$. 
Since the edge of $F'$ incident with $v$ is not removed in (a), $v$ must have at least two
pendant leaves $x$ and $y$ in $F$ that were removed.
Replace edges $xv$ and $yv$ by edge $xy$. \\
Notice none of these operations increase the cost of $F$ and if none is applicable, $F$ is a matching.
}


More precise way of computing the cost of the resulting lower bound $LB$ on the matching:
\question{}{
Show that 
\[
LB = \sum_k \varepsilon^k | \{ C \in \mathcal C^k: |C| \text{ odd }\}|
\]
where $\mathcal C^k$ and $\varepsilon^k$ are from the $k$th iterations of the algorithm.
}
\solution{
This exactly corresponds to how much each each control zone and moat grows in each iterations and they must all be crossed.
}


\question{}{
Show that the forest $F^\star$ obtained  from $F$ by \emph{dropping even edges} has cost at most 
$2 \left(\sum_v y_v + \sum_D Y_D\right)$.\\ Cost of $F^\star$ is $\sum_{e\in F^\star} c(e)$.\\
More precisely, consider every edge $e \in F^\star$ of the forest and decompose its cost into small pieces.
\[
c_e^k = \begin{cases} 0 & \text{ if } C_i = C_j \\
                     \varepsilon^k(parity(C_i)+parity(C_j)) & \text{otherwise}
                     \end{cases}
\]
Now 
\[
c(e) = \sum_k c_e^k
\]
Goal is to show for all $k$
\[
\sum_{e} c_{e}^k \leq 2  \varepsilon^k | \{ c \in \mathcal C^k: |C| \text{ odd }\}|,
\]
which gives
\[
cost(M)  \leq cost(F^\star) = \sum_k \sum_{e} c_{e}^k \leq 2  \sum_k \varepsilon^k | \{ C \in \mathcal C^k: |C| \text{ odd }\}|  = 2LB.
\]
}
\solution{
See the book.
}



\end{document}





