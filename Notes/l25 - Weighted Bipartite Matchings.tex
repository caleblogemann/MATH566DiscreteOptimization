\documentclass[11pt]{article}
\usepackage[utf8]{inputenc}
\usepackage[czech, english]{babel}
\usepackage[left=2cm,right=2cm,top=2cm,bottom=2cm]{geometry}
\usepackage{amssymb}
\usepackage{amsthm}
\usepackage{amsmath}
\usepackage{mathtools}
\usepackage{tikz}
\usetikzlibrary{shapes.geometric}
\usetikzlibrary{shapes.multipart}
\usetikzlibrary{arrows}
\usetikzlibrary{graphs}
\usepackage{hyperref}


% Itemize environment with small skips
\newenvironment{packeditemize}{
\begin{itemize}
  \setlength{\itemsep}{1pt}
  \setlength{\parskip}{0pt}
  \setlength{\parsep}{0pt}
}{\end{itemize}}


% Fancy footnote....
\usepackage{fancyhdr}
\pagestyle{fancy}
\usepackage{lastpage}
\rfoot{MATH 566 - 25, page \thepage/\pageref{LastPage}}
\cfoot{}
\rhead{}
\lhead{}
\renewcommand{\headrulewidth}{0pt}
\renewcommand{\footrulewidth}{0pt}


\newtheorem{theorem*}{Theorem} 

% Itemize environment with small skips
\newenvironment{packedenumerate}{
\begin{enumerate}
  \setlength{\itemsep}{1pt}
  \setlength{\parskip}{0pt}
  \setlength{\parsep}{0pt}
}{\end{enumerate}}


\newcommand{\vc}[1]{\ensuremath{\vcenter{\hbox{#1}}}}

\newcounter{excounter}
\setcounter{excounter}{1}
\newcommand\question[2]{\vskip 1em  \noindent\textbf{\arabic{excounter}\addtocounter{excounter}{1}:} \emph{#1} \noindent#2}
\newcommand\solution[1]{\vskip 0.5em \noindent\textbf{Solution:} #1}
%\newcommand\like{\par \noindent\emph{(This question is: good - bad - ugly? Difficulty: 0-10:\hskip 3em )}} 

\newcommand\lecturer[1]{\textbf{#1}}
\newcommand\hideforstudent[1]{#1}


%\renewcommand\lecturer[1]{{\color{white} \textbf{#1} }}
%\renewcommand\hideforstudent[1]{{\color{white}  }}
%\renewcommand\solution[1]{{\color{white} \vskip 0.5em \noindent\textbf{Solution:} #1 }}



\setlength{\parindent}{0cm}
\setlength{\parskip}{0.1cm}

\begin{document}

Fall  2016, MATH-566

\centerline{{\Large \textbf{Minimum-Weight Perfect Matching in Bipartite Graphs}}}

Source: Bill,Bill,Bill

Let $G=(V,E)$ be a graph. Let $c:E \rightarrow \mathbb{R}^+$ is a cost.
Find a perfect matching $M$ that is minimizing the sum of costs of the edges
in the matching.

\question{}{
Find minimum-weight perfect matching in the following graph:
\begin{center}
\tikzset{insep/.style={inner sep=1.5pt, outer sep=0pt, circle, fill},}
\begin{tikzpicture}[scale=1.5]
\draw   (0,0) node[insep,label=left: ](r){} 
++(45:1)  node[insep,label=above: ](a){}
++(0:1)  node[insep,label=above: ](b){}
(r) ++(-45:1)  node[insep,label=below: ](d){}
++(0:1)  node[insep,label=below: ](f){}
++(45:1) node[insep,label=above: ](g){}
++(0:1) node[insep,label=above: ](h){}
(f) ++(0:1.5) node[insep,label=below: ](i){}
;
\draw
(r) -- node[above]{3} (a)
(r) -- node[below]{4} (d)
(a) -- node[above]{4} (b)
(d) -- node[above]{5} (f)
(g) -- node[above]{5} (h)
(b) -- node[above]{5} (g)
(f) -- node[above]{4} (g)
(f) -- node[above]{6} (i)
(h) -- node[right]{7} (i)
;
\end{tikzpicture}
\end{center}
}

\question{}{
Write the  minimum-weight perfect matching as an integer program $(IP)$ on a graph $G=(V,E)$.
}
\solution{
\[
(IP)
\begin{cases}
\text{minimize}  &  \sum_{e \in E} c(e) x_e \\
\text{subject to} & \sum_{e \in \delta(v)} x_e =  1 \text{ for all } v \in V\\
                          &  \mathbf{x} \in \{0,1\}^{|E|},
\end{cases}
\]
}


\question{}{
Consider a relaxation of $(IP)$ to a linear program $(P$) and write the dual $(D)$ of $(P)$.
}
\solution{
\[
(P)
\begin{cases}
\text{minimize}  &  \sum_{e \in E} c(e) x_e \\
\text{subject to} & \sum_{e \in \delta(v)} x_e =  1 \text{ for all } v \in V\\
                          & x_e \geq 0 \text{ for all } e \in E
\end{cases}
\]
\[
(D)
\begin{cases}
\text{maximize}  &  \sum_{v \in V} y_v \\
\text{subject to} & y_{u}  + y_v \leq  c(uv) \text{ for all } uv \in E\\
                          & y_v \in \mathbb{R} \text{ for all } v \in V
\end{cases}
\]


}

\textbf{Theorem} Birkhoff: If $G$ is a bipartite graph, then solution to $(P)$ is integral. (Why?)\\
\hideforstudent{Use minimum cost flow.}

\question{}{
Formulate complementary slackness conditions for optimal solution  $\mathbf{x}$ of $(P)$
and optimal solution $\mathbf{y}$ of $(D)$.
}
\solution{\\
If $x_e > 0$, then $y_{u}  + y_v = c(uv)$.\\
If $y_{u}  + y_v < c(uv)$ then  $x_e = 0$, then.\\
}



Algorithm idea: Maintain an optimal solution to $(D)$, create a solution to $(P)$ whose value
is matching the dual solution.

\question{}{
Find initial solutions to $(P)$ and $(D)$, where solution to $(D)$ is feasible and the
solutions satisfy complementary slackness. (Solution to $(P)$ does not have to be feasible.) 
}
\solution{
$\textbf{x} = 0$, $\textbf{y} = 0$ will do.
}



\question{}{
If the solution to $(D)$ is fixed, which edges can be used in matching? (Denote the edges by $E_=$.)
}
\solution{
All edges $e=uv$, where  $y_{u}  + y_v = c(e)$ can have $x_e > 0$.
}


Algorithm sketch, suppose a perfect matching exists.
\begin{packeditemize}
\item start with initial solution $\mathbf{x}$,$\mathbf{y}$.
\item Take edges $E_=$ and try to find a perfect matching $M$ by growing augmenting forests
\item If $M$ is not perfect, there are some outer of $F$ vertices adjacent to edges in $E$ but not in $E_=$.
\item Update $\mathbf{y}$ to allow more edges in $E_=$ and repeat.
\end{packeditemize}

\question{}{
What to do if there is no perfect matching in $E_=$? Consider the following example.
Number on edge $e$ is $c(e)$, number at vertex $v$ is  $y_v$. How to modify $\mathbf{y}$
to allow the tree to grow?
\begin{center}
\tikzset{insep/.style={inner sep=1.5pt, outer sep=0pt, circle, fill},}
\begin{tikzpicture}[scale=1.4]
\draw   (0,0) node[insep,label=left:1](r){} 
++(45:1)  node[insep,label=above:2](a){}
++(0:1)  node[insep,label=above:2](b){}
(r) ++(-45:1)  node[insep,label=below:3](d){}
++(0:1)  node[insep,label=below:2](f){}
++(45:1) node[insep,label=above:1](g){}
++(0:1) node[insep,label=above:4](h){}
(f) ++(0:1.5) node[insep,label=below:1](i){}
;
\draw
(r) -- node[above]{3} (a)
(r) -- node[below]{4} (d)
;
\draw[line width=1pt]
(a) -- node[above]{4} (b)
(d) -- node[above]{5} (f)
(g) -- node[above]{5} (h)
;
\draw[dashed]
(b) -- node[above]{5} (g)
(f) -- node[above]{4} (g)
(f) -- node[above]{6} (i)
(h) -- node[right]{7} (i)
;
\end{tikzpicture}
\end{center}
}

\solution{
\begin{center}
\tikzset{insep/.style={inner sep=1.5pt, outer sep=0pt, circle, fill},}
\begin{tikzpicture}[scale=1.4]
\draw   (0,0) node[insep,label=left:+](r){} 
++(45:1)  node[insep,label=above:-](a){}
++(0:1)  node[insep,label=above:+](b){}
(r) ++(-45:1)  node[insep,label=below:-](d){}
++(0:1)  node[insep,label=below:+](f){}
++(45:1) node[insep,label=above:1](g){}
++(0:1) node[insep,label=above:4](h){}
(f) ++(0:1.5) node[insep,label=below:1](i){}
;
\foreach \x in {a,b,d,r,f,g,h,i}{
\draw[black] (\x) node[insep]{};
} 
\draw[dashed,color=black]
(r) -- node[above]{3} (a)
(r) -- node[below]{4} (d)
(a) -- node[above]{4} (b)
(d) -- node[above]{5} (f)
(g) -- node[above]{5} (h)
(b) -- node[above]{5} (g)
(f) -- node[above]{4} (g)
(f) -- node[above]{6} (i)
(h) -- node[right]{7} (i)
;
\end{tikzpicture}
\hskip 4em
\begin{tikzpicture}[scale=1.4]
\draw   (0,0) node[insep,label=left:2](r){} 
++(45:1)  node[insep,label=above:1](a){}
++(0:1)  node[insep,label=above:3](b){}
(r) ++(-45:1)  node[insep,label=below:2](d){}
++(0:1)  node[insep,label=below:3](f){}
++(45:1) node[insep,label=above:1](g){}
++(0:1) node[insep,label=above:4](h){}
(f) ++(0:1.5) node[insep,label=below:1](i){}
;
\draw
(r) -- node[above]{3} (a)
(r) -- node[below]{4} (d)
(f) -- node[above]{4} (g)
;
\draw[line width=1pt]
(a) -- node[above]{4} (b)
(d) -- node[above]{5} (f)
(g) -- node[above]{5} (h)
;
\draw[dashed]
(b) -- node[above]{5} (g)
(f) -- node[above]{6} (i)
(h) -- node[right]{7} (i)
;
\end{tikzpicture}
\end{center}
\begin{center}
\tikzset{insep/.style={inner sep=1.5pt, outer sep=0pt, circle, fill},}
\begin{tikzpicture}[scale=1.4]
\draw   (0,0) node[insep,label=left:4](r){} 
++(45:1)  node[insep,label=above:-1](a){}
++(0:1)  node[insep,label=above:5](b){}
(r) ++(-45:1)  node[insep,label=below:0](d){}
++(0:1)  node[insep,label=below:5](f){}
++(45:1) node[insep,label=above:-1](g){}
++(0:1) node[insep,label=above:6](h){}
(f) ++(0:1.5) node[insep,label=below:1](i){}
;
\draw
(r) -- node[above]{3} (a)
(r) -- node[below]{4} (d)
(f) -- node[above]{4} (g)
(f) -- node[above]{6} (i)
;
\draw[line width=1pt]
(a) -- node[above]{4} (b)
(d) -- node[above]{5} (f)
(g) -- node[above]{5} (h)
;
\draw[dashed]
(b) -- node[above]{5} (g)
(h) -- node[right]{7} (i)
;
\end{tikzpicture}
\hskip 4em
\tikzset{insep/.style={inner sep=1.5pt, outer sep=0pt, circle, fill},}
\begin{tikzpicture}[scale=1.4]
\draw   (0,0) node[insep,label=left:4](r){} 
++(45:1)  node[insep,label=above:-1](a){}
++(0:1)  node[insep,label=above:5](b){}
(r) ++(-45:1)  node[insep,label=below:0](d){}
++(0:1)  node[insep,label=below:5](f){}
++(45:1) node[insep,label=above:-1](g){}
++(0:1) node[insep,label=above:6](h){}
(f) ++(0:1.5) node[insep,label=below:1](i){}
;
\draw
(r) -- node[above]{3} (a)
(f) -- node[above]{4} (g)
(d) -- node[above]{5} (f)
;
\draw[line width=1pt]
(a) -- node[above]{4} (b)
(g) -- node[above]{5} (h)
(f) -- node[above]{6} (i)
(r) -- node[below]{4} (d)
;
\draw[dashed]
(b) -- node[above]{5} (g)
(h) -- node[right]{7} (i)
;
\end{tikzpicture}
\end{center}
}

Recall that during growing the tree, we encountered edges like $e_5$ that were not possible
to drop from the tree. The above algorithm does not work for edges with $e_5$.\begin{center}
\tikzset{insep/.style={inner sep=1.5pt, outer sep=0pt, circle, fill},}
\tikzset{inwhite/.style={inner sep=1.5pt, outer sep=0pt, circle, fill=white,draw},}
\begin{tikzpicture}[scale=1]
\draw  
(0,0) node[inwhite](x1){}
(-0.5,1) node[insep](x2){}
 (0.5,1) node[insep](x3){}
(-0.5,2) node[inwhite](x4){}
 (0.5,2) node[inwhite](x5){}
 
 (3,0) node[inwhite](y1){}
 (2,1) node[insep](y2){}
 (3,1) node[insep](y3){}
 (4,1) node[insep](y4){}
 (2,2) node[inwhite](y5){}
 (3,2) node[inwhite](y6){}
 (4,2) node[inwhite](y7){}
;
\draw(x1)--(x2) (x1)--(x3) (y1)--(y2) (y1)--(y3) (y1)--(y4) ;
\draw[line width=1.6pt] (x2)--(x4)  (x3)--(x5)  (y2)--(y5) (y3)--(y6)(y4)--(y7);
\draw[line width=2pt,dotted,color=red] (x4) --node[above]{$e_5$} (x5)  
%(x3) --node[below]{$e_4$} (y5)
;
\end{tikzpicture}
\end{center}
Our algorithm works only for \textbf{bipartite} graphs. Can be (nontrivially) generalized for
all graphs. (The linear program has to be stronger by adding more constraints - for all odd vertex subsets, at least one edge is in the cut, blossoms need to be treated carefully.)
\end{document}






