\documentclass[11pt]{article}
\usepackage[utf8]{inputenc}
\usepackage[czech, english]{babel}
\usepackage[left=2cm,right=2cm,top=2cm,bottom=2cm]{geometry}
\usepackage{amssymb}
\usepackage{amsthm}
\usepackage{amsmath}
\usepackage{tikz}


% Itemize environment with small skips
\newenvironment{packeditemize}{
\begin{itemize}
  \setlength{\itemsep}{1pt}
  \setlength{\parskip}{0pt}
  \setlength{\parsep}{0pt}
}{\end{itemize}}


% Fancy footnote....
\usepackage{fancyhdr}
\pagestyle{fancy}
\usepackage{lastpage}
\rfoot{MATH 566 - 06, page \thepage/\pageref{LastPage}}
\cfoot{}
\rhead{}
\lhead{}
\renewcommand{\headrulewidth}{0pt}
\renewcommand{\footrulewidth}{0pt}


\newtheorem{theorem*}{Theorem} 

% Itemize environment with small skips
\newenvironment{packedenumerate}{
\begin{enumerate}
  \setlength{\itemsep}{1pt}
  \setlength{\parskip}{0pt}
  \setlength{\parsep}{0pt}
}{\end{enumerate}}


\newcounter{excounter}
\setcounter{excounter}{1}
\newcommand\question[2]{\vskip 1em  \noindent\textbf{\arabic{excounter}\addtocounter{excounter}{1}:} \emph{#1} \noindent#2}
\newcommand\solution[1]{\vskip 0.5em \noindent\textbf{Solution:} #1}
\newcommand\like{\par \noindent\emph{(This question is: good - bad - ugly? Difficulty: 0-10:\hskip 3em )}} 

\newcommand\lecturer[1]{\textbf{#1}}
\newcommand\hideforstudent[1]{#1}


\renewcommand\lecturer[1]{{\color{white} \textbf{#1} }}
\renewcommand\hideforstudent[1]{{\color{white} #1 }}
\renewcommand\solution[1]{{\color{white} #1 }}



\setlength{\parindent}{0cm}
\setlength{\parskip}{0.1cm}

\begin{document}

Fall  2016, MATH-566

\centerline{{\Large \textbf{Farkas Lemma and proof of duality}}}


\textbf{Farkas Lemma:} 
Let $A \in \mathbb{R}^{m\times n}$ and $\mathbf{b} \in \mathbb{R}^m$. Exactly one of the following holds
\begin{packeditemize}
\item $\exists \mathbf{x} \in \mathbb{R}^n$ such that $A\mathbf{x} = \mathbf{b}$ and $\mathbf{x} \geq \mathbf{0}$\\
\item $\exists \mathbf{y} \in \mathbb{R}^m$ such that $\mathbf{y}^TA \geq \mathbf{0}^T$ and $\mathbf{y}^T\mathbf{b} < 0$
\end{packeditemize}

\question{}{Is it possible to satisfy both conditions at the same time?}
\solution{
No, consider
$\mathbf{y}^TA\mathbf{x} = \mathbf{y}^T\mathbf{b}$.
The left-hand side is $\geq 0$ while the right-hand side is negative.
}


A (convex) \textbf{cone} is a set $C \in \mathbb{R}^d$ for which $\mathbf{x},\mathbf{y} \in C$ and $a,b \geq 0$ implies $a\mathbf{x}+b\mathbf{y} \in C$.\\
A cone $C$ generated by  $X=\{\mathbf{a}_1,\ldots,\mathbf{a}_n\} \subseteq \mathbb{R}^d$ are are all linear combinations
of vectors in $X$ with nonnegative coefficients
\[
C = \{ t_1\mathbf{a}_1+t_2\mathbf{a}_2+\cdots+t_n\mathbf{a}_n: t_i \geq 0 \} \subseteq \mathbb{R}^d
\]
\begin{center}
\begin{tikzpicture}
  \shade[bottom color=gray!50,top color=gray!30] 
      (-0.6,1.2) -- (0,0) -- (1.2,1.2);
\draw[-latex](0,0) -- (-0.5,1) node[label=left:{$\mathbf{a}_1$}]{};
\draw[-latex](0,0) -- (1,1) node[label=right:{$\mathbf{a}_3$}]{};
\draw[-latex](0,0) -- (0.5,1) node[label=above:{$\mathbf{a}_2$}]{};
\draw (0,1) node {$C$};
\draw (0,0) node[label=below:{$\mathbf{0}$}] {};
\end{tikzpicture}
\end{center}
Convex cone can be defined for any generating set $X$. If $X$ is finite, then $C$ is closed.

\textbf{Geometric of Farkas Lemma}:
Let $\mathbf{a}_1,\ldots,\mathbf{a}_n,\mathbf{b} \in \mathbb{R}^m$. 
Let $C$ be the convex cone generated by $\mathbf{a}_i$s.
Exactly one of the following holds
\begin{packeditemize}
\item $\mathbf{b} \in C$ \\
\item exists hyperplane $H$ such that $\mathbf{0} \in H$ and $H$ strictly separates $\mathbf{b}$ from $C$. 
That is $H = \{\mathbf{x}: \mathbf{h}^T\mathbf{x} = 0\}$ and $\forall i,  \mathbf{h}^T\mathbf{a}_i \geq 0$ and
$\mathbf{h}^T\mathbf{b} < 0$.
\end{packeditemize}
\begin{center}
\begin{tikzpicture}
  \shade[bottom color=gray!50,top color=gray!30] 
      (-0.6,1.2) -- (0,0) -- (1.2,1.2);
\draw[-latex](0,0) -- (-0.5,1);
\draw[-latex](0,0) -- (1,1);
\draw[-latex](0,0) -- (0.5,1);
\draw (0,1) node {$C$};
\draw[dotted,line width=1pt](0,0)  +(30:-1) -- +(30:2) node[label=right:{$H$}]{};
\fill(1,0) circle (2pt)  node[label=right:{$\mathbf{b}$}]{};
\draw (0,0) node[label=below:{$\mathbf{0}$}] {};
\end{tikzpicture}
\end{center}

\question{}{Prove Farkas lemma using separation theorem. (What the separation gives?)}
\solution{
From separation theorem, exists $\mathbf{h} \in \mathbb{R}^m$ and $z \in \mathbb{R}$ such that
$\forall \mathbf{x} \in C, \mathbf{h}^T\mathbf{x} > z$ and $\mathbf{h}^T\mathbf{b} < z$.
Since $\mathbf{0} \in C$, we get $\mathbf{h}^T\mathbf{0} =  0 > z$. We can try to replace $z$ by $0$
and get not strict separation for the cone. 

What if $\exists \mathbf{x} \in C$  such that $\mathbf{h}^T\mathbf{x} < 0$?
Then $1000 \mathbf{x} \in C$ and $1000 \mathbf{h}^T\mathbf{x} < z$ if 1000 big enough.
Hence we can let $z=0$.
}


Reformulations of Farkas lemma:
\begin{itemize}
\item $A\mathbf{x} = \mathbf{b}$ has a non-negative solution iff $\forall \mathbf{y} \in \mathbb{R}^m$ with $\mathbf{y}^TA \geq \mathbf{0}^T$
also $\mathbf{y}^T\mathbf{b} \geq 0$.
\item $A\mathbf{x} \leq \mathbf{b}$ has a non-negative solution iff $\forall \mathbf{y} \in \mathbb{R}^m$, $\mathbf{y} \geq \mathbf{0}$ 
with $\mathbf{y}^TA \geq \mathbf{0}^T$ also satisfies $\mathbf{y}^T\mathbf{b} \geq \mathbf{0}$.
\item  $A\mathbf{x} \leq \mathbf{b}$ has a solution iff $\forall \mathbf{y} \in \mathbb{R}^m$, $\mathbf{y} \geq \mathbf{0}$  with $\mathbf{y}^TA = \mathbf{0}^T$ also satisfies $\mathbf{y}^T\mathbf{b} \geq \mathbf{0}$.
\end{itemize}

\newpage
Lets have linear programs 
\[ \text{maximize } \textbf{c}^T\textbf{x} \text{ subject to } A\mathbf{x} \leq \mathbf{b} \text{ and  } \mathbf{x} \geq \mathbf{0}  \text{ \hskip 3cm }  (P)  \]
\[ \text{minimize } \textbf{b}^T\textbf{y} \text{ subject to } A^T\mathbf{y} \geq \mathbf{c} \text{ and  } \mathbf{y} \geq \mathbf{0}  \text{ \hskip 3cm }  (D) \]

Lemma (Weak Duality):
Let $\mathbf{x}$ and $\mathbf{y}$ be feasible solutions of $(P)$ and $(D)$. Then
\[
\mathbf{c}^T\mathbf{x} \leq \mathbf{b}^T\mathbf{y}
\]
\question{}{
Prove the weak duality
}
\solution{
\[
\mathbf{c}^T\mathbf{x} = \mathbf{x}^T\mathbf{c} \leq  \mathbf{x}^TA^T\mathbf{y} = (A\mathbf{x})^T\mathbf{y} \leq \mathbf{b}^T\mathbf{y}
\]
}

\vskip 2em

\textbf{Proof of the duality theorem point 4. from the Farkas lemma. ($\textbf{c}^T\textbf{x}^\star = \textbf{b}^T\textbf{y}^\star.$)}

Let $\mathbf{x}^\star$ be optimal solution. Let $\gamma = \mathbf{c}^T\mathbf{x}$. 
\question{}{ Are there solutions to $A\mathbf{x} \leq \mathbf{b}$ and $\mathbf{c}^T\mathbf{x} \geq \gamma$? }
\solution{Yes, $\mathbf{x}^\star$.}
\question{}{ Are there solutions to $A\mathbf{x} \leq \mathbf{b}$ and $\mathbf{c}^T\mathbf{x} \geq \gamma+\varepsilon$, where $\varepsilon > 0$? }
\solution{No, contradiction with $\mathbf{x}^\star$ being optimal.}

Let $\hat{A} = \binom{A}{-\mathbf{c}^T}$ and $\hat{\mathbf{b}}_{\varepsilon} = \binom{\mathbf{b}}{-\gamma-\varepsilon}$.
\question{}{Apply Farkas Lemma on $\hat{A}\mathbf{x} \leq  \hat{\mathbf{b}}_{\varepsilon}$ (which version?, write
$\hat{\mathbf{y}}$ from FL as $(\mathbf{u},z)\in  \mathbb{R}^{m+1}$ ?)}
\solution{
FL: implies there exists $\hat{\mathbf{y}} \in \mathbb{R}^{m+1}$ such that
$\hat{\mathbf{y}} \geq \mathbf{0}$, $\hat{\mathbf{y}}^T\hat{A} \geq  \mathbf{0}^T$ and  $\hat{\mathbf{y}}^T\hat{\mathbf{b}}_{\varepsilon} < 0$.

If we assign $(\mathbf{u},z) = \hat{\mathbf{y}}$ we get
\[
\mathbf{u}^TA - z\cdot \mathbf{c}^T \geq \mathbf{0}^T \text{ and } \mathbf{u}^T\mathbf{b} - z(\gamma+\varepsilon) < 0.
\]
Which can be rewritten as 
\[
A^T\mathbf{u}  \geq  z\cdot \mathbf{c}  \text{ and } \mathbf{u}^T\mathbf{b}  <  z(\gamma+\varepsilon).
\]
Divide by $z$ and we get
\[
A^T\frac{\mathbf{u}}{z}  \geq \mathbf{c}  \text{ and } \frac{\mathbf{u}}{z}^T\mathbf{b}  <  (\gamma+\varepsilon).
\]
Let $\mathbf{y}_{\varepsilon} = \frac{\mathbf{u}}{z}$. Then
\[
\forall \varepsilon > 0, \exists \mathbf{y}_{\varepsilon}, A^T\mathbf{y}_{\varepsilon}  \geq \mathbf{c}  \text{ and } \mathbf{y}_{\varepsilon}^T\mathbf{b}  <  (\gamma+\varepsilon).
\]
By taking limit for $\varepsilon \rightarrow 0$, we get that there exists $\mathbf{y}^\star$ such that $A^T\mathbf{y}^\star  \geq \mathbf{c}  \text{ and } \mathbf{b}^T\mathbf{y}^\star  \leq  \gamma$. By weak duality $\mathbf{b}^T\mathbf{y}^\star  = \gamma$ and $\mathbf{y}^\star $ is an optimal solution.
}
\question{}{What happens if $z = 0$? (Hint: Use Farkas lemma again with $\varepsilon = 0$.)}
\solution{
Use Farkas Lemma with $\varepsilon = 0$.
It changes to $\forall$. In particular, it holds for $(\mathbf{u},z) = \hat{\mathbf{y}}$ and implies $\mathbf{u}^T\mathbf{b}  \geq  z\gamma$.
If $z=0$, we would get a contradiction with $\mathbf{u}^T\mathbf{b}  <  z(\gamma+\varepsilon)$.
}

\end{document}



