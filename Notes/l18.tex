\documentclass[11pt]{article}
\usepackage[utf8]{inputenc}
\usepackage[czech, english]{babel}
\usepackage[left=2cm,right=2cm,top=2cm,bottom=2cm]{geometry}
\usepackage{amssymb}
\usepackage{amsthm}
\usepackage{amsmath}
\usepackage{tikz}
\usetikzlibrary{shapes.geometric}
\usetikzlibrary{shapes.multipart}
\usepackage{hyperref}


% Itemize environment with small skips
\newenvironment{packeditemize}{
\begin{itemize}
  \setlength{\itemsep}{1pt}
  \setlength{\parskip}{0pt}
  \setlength{\parsep}{0pt}
}{\end{itemize}}


% Fancy footnote....
\usepackage{fancyhdr}
\pagestyle{fancy}
\usepackage{lastpage}
\rfoot{MATH 566 - 18, page \thepage/\pageref{LastPage}}
\cfoot{}
\rhead{}
\lhead{}
\renewcommand{\headrulewidth}{0pt}
\renewcommand{\footrulewidth}{0pt}


\newtheorem{theorem*}{Theorem} 

% Itemize environment with small skips
\newenvironment{packedenumerate}{
\begin{enumerate}
  \setlength{\itemsep}{1pt}
  \setlength{\parskip}{0pt}
  \setlength{\parsep}{0pt}
}{\end{enumerate}}


\newcommand{\vc}[1]{\ensuremath{\vcenter{\hbox{#1}}}}

\newcounter{excounter}
\setcounter{excounter}{1}
\newcommand\question[2]{\vskip 1em  \noindent\textbf{\arabic{excounter}\addtocounter{excounter}{1}:} \emph{#1} \noindent#2}
\newcommand\solution[1]{\vskip 0.5em \noindent\textbf{Solution:} #1}
%\newcommand\like{\par \noindent\emph{(This question is: good - bad - ugly? Difficulty: 0-10:\hskip 3em )}} 

\newcommand\lecturer[1]{\textbf{#1}}
\newcommand\hideforstudent[1]{#1}


\renewcommand\lecturer[1]{{\color{white} \textbf{#1} }}
\renewcommand\hideforstudent[1]{{\color{white} #1 }}
%\renewcommand\solution[1]{{\color{white} \vskip 0.5em \noindent\textbf{Solution:} #1 }}



\setlength{\parindent}{0cm}
\setlength{\parskip}{0.1cm}

\begin{document}

Fall  2016, MATH-566

\centerline{{\Large \textbf{Gomory-Hu Trees}}}

Let $G=(V,E)$ be an \textbf{undirected} graph and $u: E \rightarrow \mathbb{R}_+$ be capacities on edges.

\textbf{Problem:} Compute minimum $s$-$t$-cut for all pairs $(s,t) \in V^2$. 

Simple solution: Run $\binom{n}{2}$ times maximum-flow algorithm (it gives minimum cut too).

Better solution: Run $(n-1)$ times maximum-flow algorithm. Due to Gomory-Hu.

Denote the minimum capacity of $s$-$t$-cut by $\lambda_{st}$.


\question{}{
Show that for any $i,j,k \in V(G)$, $\lambda_{ik} \geq \min\{\lambda_{ij}, \lambda_{jk}\}$.
}
\solution{
Consider minimum $i$-$k$-cut $\delta(A)$, where $i \in A \subset V$.
If $j \in k$, then $A$ is also $j$-$k$-cut. Hence $\lambda_{ik} \geq \lambda_{jk}$. The other
case is symmetrical.
}

A tree $T$ is a \textbf{Gomory-Hu Tree} for $(G,u)$ if $V(T) = V(G)$ and $\forall s,t \in V$
\[
\lambda_{st} = \min_{e \in E(P_{st})} u(\delta_G(C_e)),
\]
where $P_{st}$ is the unique $s$-$t$-path in $T$,
$C_e$ is the set of vertices in the same connected component of $T-e$ as $s$
and  $\delta_G(C_e)$ is the cut defined by $C_e$ in $G$.

Example of $G$, where $u: E \rightarrow 1$ is a constant. The tree $T$ is then a star.
\begin{center}
\tikzset{insep/.style={inner sep=2pt, outer sep=0pt, circle, fill},}
\begin{tikzpicture}[scale=0.8]
\draw (0,0)node[insep](a){} -- (2,0) node[insep](x){};
\draw (0,1)node[insep](b){} -- (2,1) node[insep](y){};
\draw (0,2)node[insep](c){} -- (2,2) node[insep](z){};
\draw (a)--(y)--(c)--(x)--(b)--(z)--(a);
\end{tikzpicture}
\hskip 3em
\begin{tikzpicture}[scale=0.8]
\draw (0,0)node[insep](a){}  (2,0) node[insep](x){};
\draw (0,1)node[insep](b){}  (2,1) node[insep](y){};
\draw (0,2)node[insep](c){}  -- (2,2) node[insep](z){};
\draw (x)--(c)--(y) (b) -- (c) to[bend right=30] (a);
\end{tikzpicture}
\end{center}

\question{}{
Find Gomory-Hu tree for the following graph with weights on edges.
\begin{center}
\tikzset{insep/.style={inner sep=2pt, outer sep=0pt, circle, fill},}
\begin{tikzpicture}[scale=1.5]
\draw (0,0)node[insep](x){} -- node[right]{6} ++(50:1) node[insep](y){}
(x)  -- node[left]{4} ++(130:1) node[insep](z){}
(z) -- node[above]{3} (y)
(z)  -- node[left]{6} ++(50:1) node[insep](w){}
(y) -- node[right]{3} (w)

 (2,0)node[insep](a){} -- node[right]{6} ++(50:1) node[insep](b){}
(a)  -- node[left]{4} ++(130:1) node[insep](c){}
(b) -- node[above]{3} (c)
(c)  -- node[left]{6} ++(50:1) node[insep](d){}
(b) -- node[right]{3} (d)

(a)-- node[below]{5} (x)
(w)-- node[above]{5} (d)
;
\end{tikzpicture}
\hskip 4em
\begin{tikzpicture}[scale=1.5]
\draw (0,0)node[insep](x){} ++(50:1) node[insep](y){}
(x)  ++(130:1) node[insep](z){}
(z)   ++(50:1) node[insep](w){}
 (2,0)node[insep](a){}  ++(50:1) node[insep](b){}
(a)   ++(130:1) node[insep](c){}
(c)   ++(50:1) node[insep](d){}
;
\end{tikzpicture}
\end{center}
}


Algorithm:
\begin{enumerate}
\item $T=(\{V\},\emptyset)$
\item while exists $X \in V(T)$, where $|X| \geq 2$,
\item \hskip 4em pick any $s$, $t$ in $X$   \label{alg:X}
\item \hskip 4em contract vertices of all nodes other  than $X$
\item \hskip 4em find minimum $s$-$t$-cut $A \cup B = V$
\item \hskip 4em replace $X$ in $V(T)$ by edge $\{\{A\cap X\}, \{B \cap X\}\}$.
\end{enumerate}

\newpage

Sketch of one iteration:
\begin{center}
\vc{
\begin{tikzpicture}
\draw (0,0) node[draw,circle](x){$X$};
\draw (x)  ++(45:1.5) node[draw,circle](c){$C$};
\draw (x)  ++(135:1.5) node[draw,circle](d){$D$} ++(180:1.5) node[draw,circle](e){$E$};
\draw (x)  ++(225:1.5) node[draw,circle](f){$F$} ++(180:1.5) node[draw,circle](g){$G$};
\draw (x) --(c) (x)--(d)--(e) (x)--(f)--(g);
\end{tikzpicture}
}
\hskip 3em
\vc{
\begin{tikzpicture}
\draw (0,0) node [circle split,draw](x){$A\cap X$ \nodepart{lower} $B\cap X$};
\draw (x)  ++(45:1.5) node[draw,circle](c){$C$};
\draw (x)  ++(135:1.5) node[draw,circle](d){$D$} ++(180:1.5) node[draw,circle](e){$E$};
\draw (x)  ++(225:1.5) node[draw,circle](f){$F$} ++(180:1.5) node[draw,circle](g){$G$};
\draw (x) --(c) (x)--(d)--(e) (x)--(f)--(g);
\end{tikzpicture}
}
\hskip 3em
\vc{
\begin{tikzpicture}
\draw (0,0) node [circle,draw](a){$A\cap X$};
\draw (0,-2) node [circle,draw](b){$B\cap X$};
\draw (a)  ++(45:1.5) node[draw,circle](c){$C$};
\draw (a)  ++(135:1.5) node[draw,circle](d){$D$} ++(180:1.5) node[draw,circle](e){$E$};
\draw (b)  ++(225:1.5) node[draw,circle](f){$F$} ++(180:1.5) node[draw,circle](g){$G$};
\draw (a) --(c) (a)--(d)--(e) (b)--(f)--(g) (a)--(b);
\end{tikzpicture}
}
\end{center}

\question{}{
Run the algorithm on the graph from question 2.
\begin{center}
\tikzset{insep/.style={inner sep=2pt, outer sep=0pt, circle, fill},}
\begin{tikzpicture}[scale=1.5]
\draw (0,0)node[insep](x){} -- node[right]{6} ++(50:1) node[insep](y){}
(x)  -- node[left]{4} ++(130:1) node[insep](z){}
(z) -- node[above]{3} (y)
(z)  -- node[left]{6} ++(50:1) node[insep](w){}
(y) -- node[right]{3} (w)

 (2,0)node[insep](a){} -- node[right]{6} ++(50:1) node[insep](b){}
(a)  -- node[left]{4} ++(130:1) node[insep](c){}
(b) -- node[above]{3} (c)
(c)  -- node[left]{6} ++(50:1) node[insep](d){}
(b) -- node[right]{3} (d)

(a)-- node[below]{5} (x)
(w)-- node[above]{5} (d)
;
\end{tikzpicture}
\hskip 4em
\begin{tikzpicture}[scale=1.5]
\draw (0,0)node[insep](x){} ++(50:1) node[insep](y){}
(x)  ++(130:1) node[insep](z){}
(z)   ++(50:1) node[insep](w){}
 (2,0)node[insep](a){}  ++(50:1) node[insep](b){}
(a)   ++(130:1) node[insep](c){}
(c)   ++(50:1) node[insep](d){}
;
\end{tikzpicture}
\end{center}
}



\question{(Cuts are submodular)}{
That is, let $A,B \subset V$. Show that
\[
u (\delta(A \cup B)) + u (\delta(A\cap B)) \leq u(\delta(A)) + u(\delta(B)).
\]
}
\solution{
All edges on the left are counted on the right but edges going between $A \setminus B$ and $B \setminus A$
are missing on the left hand side. The dashed edges are missing.
\begin{center}
\begin{tikzpicture}
\draw(0,0) ellipse(1cm and 0.5cm);
\draw(1,0) ellipse(1cm and 0.5cm);
\draw (-0.5,0) node {$A$} (1.5,0) node {$B$};
\draw (-0.2,0.2)-- ++(150:1) (1.7,0.2)-- ++(30:1) (0.5,0.2) -- ++(90:1) (0.6,-0.1) -- ++(0:0.6) (0.4,-0.1) -- ++(180:0.6)
;
\draw[dashed, line width =1pt] 
(0,-0.4) to[bend right=40] (1,-0.4)
;
\end{tikzpicture}
\end{center}
}


\question{Algorithm creates optimal cuts}{
Let $s,t \in V$ and let $A \subset V$ such that $\delta(A)$ is a minimum $s$-$t$-cut. 
Let $s',t' \in V \setminus A$. Let $(G',u')$ be obtained from $G$ by contracting vertices
of $A$ into one vertex $a'$. Let $K \subset(G')$ such that $\delta_{G'}(K \cup \{a'\})$ is
a minimum $s'$-$t'$-cut in $G'$. Show that $\delta_{G}(K \cup A)$ is a minimum $s'$-$t'$-cut
in $G$. \\
Proof beginning: Assume $\delta(C)$ is a minimum $s'$-$t'$ cut in $G$. 
Show that $\delta(C\cup A)$ is also a minimum $s'$-$t'$ cut in $G$.  Wlog $s \in A\cap C$.
\begin{center}
\tikzset{insep/.style={inner sep=2pt, outer sep=0pt, circle, fill},}
\begin{tikzpicture}
\draw
(0,0) circle (2cm)
(0,-2.3) -- (0,2.3) node[left]{$A$} node[right]{$V(G)\setminus A$}
(0,0) ++(225:1) node[insep,label=below:$s$]{}
(0,0) ++(315:1) node[insep,label=below:$s'$]{}
(0,0) ++(45:1) node[insep,label=above:$t'$]{}
;
\draw[dashed]
(-2.3,0) -- (2.3,0) node[above]{$V(G)\setminus C$} node[below]{$C$}
;
\end{tikzpicture}
\end{center}
} 
\solution{
By submodularity, we have
\[
u (\delta(A \cup C)) + u (\delta(A\cap C)) \leq u(\delta(A)) + u(\delta(C))
\]
Observe that $\delta(A\cap C)$ is an $s$-$t$-cut. Hence
$u(\delta(A)) \leq u(\delta(A\cap C))$. Therefore $u (\delta(A \cup C))  \leq u(\delta(C))$
and $\delta(A \cup C)$ is also minimum $s'$-$t'$-cut.
}


\question{}{
Let $T$ be a tree during the run of algorithm. Let $e \in E(T)$ be any edge of $T$. Denote the endpoints of $e$ by  $X$ and $Y$.
Show that there are vertices $x \in X$ and $y \in Y$ such that $e$ describes a minimum $x$-$y$ cut.

\emph{The algorithm produces tree that works like Gomory-Hu tree at least for vertices
adjacent in the tree.}
}

Proof start:
At the beginning of the algorithm (or after the first iteration, the observation is true.
Let $X$ and $s$, $t$ be from step \ref{alg:X}. The new edge $A\cap X$ to $B\cap X$ is correct due to $s,t$.
Edges not incident with $X$ are also correct. Remaining are edges that used to be incident 
with $X$ but now are incident with $A\cap X$ (or $B \cap X$).

Suppose the edge $YX$ had vertices $y$ and $x$. If $Y$ is in the  $A$ part of $s$-$t$-cut
and $x$ is in the other part, the edge $Y,(X \cap A)$ needs to be verified to satisfy the conclusion.
\begin{center}
\tikzset{insep/.style={inner sep=2pt, outer sep=0pt, circle, fill},}
\begin{tikzpicture}
\draw (0,0) node [circle,draw,inner sep=17pt](a){} node[below]{$A\cap X$};
\draw (2,0) node [circle,draw,inner sep=17pt](b){}   node[below]{$B\cap X$};
\draw (a)  ++(180:2) node[draw,circle,inner sep=17pt](c){} node[below]{$Y$};
\draw (c) -- (a) -- (b);
\draw (-2,0.2)  node[insep,label=above:$y$]{};
\draw (0,0.2)  node[insep,label=above:$s$]{};
\draw (1.7,0.2)  node[insep,label=above:$t$]{};
\draw (2.3,0.2)  node[insep,label=above:$x$]{};
\end{tikzpicture}
\end{center}

\solution{
We show $\lambda_{sy} = \lambda_{yt}$, which gives the desired conclusion.
We get
\[
\lambda_{sy} \geq \min\{ \lambda_{st}, \lambda_{tx}, \lambda_{xy} \}
\]
If we contract $B \cap X$ to one vertex, $\lambda_{sy}$ does not change.
Hence
\[
\lambda_{sy} \geq \min\{ \lambda_{st}, \lambda_{xy} \}
\]
Now $A$-$B$-cut separates $x$ and $y$, hence $\lambda_{st} \geq \lambda_{xy}$ and we have
\[
\lambda_{sy} \geq \lambda_{xy} \}
\]
On the other hand, $X$-$Y$ cut has size $\lambda_{xy}$
and it is also $s$-$y$-cut. Hence \[
\lambda_{sy} \leq \lambda_{xy} 
\]
and the equality is proved.
}

\newpage

\question{}{
Show that the tree produced by the algorithm is indeed a Gomory-Hu tree.
}
\solution{
Let $u,v$ be two vertices. Let $P_{uv}= (u=x_1,x_2,\ldots,x_n=v)$ be the path with endpoints $u$ and $v$ in $T$.

Observe
\[
\lambda_{uv} \geq \min_{1 \leq i \leq n-1} \{ \lambda_{x_ix_{i+1}} \}
\]
Since $\lambda_{x_ix_{i+1}}$ is obtained by edge $x_ix_{i+1}$ in the tree, we are done.
}

\vfill


\newpage

\question{Gomory-Hu Tree}{
Construct Gomory-Hu Tree for the following graph using the algorithm
from the class. Numbers
on edges correspond to capacities.
\begin{center}
\tikzset{insep/.style={inner sep=2pt, outer sep=0pt, circle, fill},}
\begin{tikzpicture}
\draw
\foreach \x/\a in {0/a,60/b,120/c,180/d,240/e,300/f}{
(\x:2.5) node[insep](\a){}
}
(a)--node[right]{2} (b) -- node[above]{1} (c) -- node[left]{2}(d) -- node[below]{2}(e) -- node[below]{1}(f) -- node[right]{1}(a)
(b) -- node[below]{3} (d) -- node[below]{2}(f) -- node[right]{3}(b)
;
\end{tikzpicture}
\end{center}
% TODO: Check your answer with Sage. Is the tree unique?
}
\solution{

Here is a more detailed outline of the run of the algorithm. 
First we assign labels to vertices.
\tikzstyle{vertex_style}=[circle, draw, inner sep=2pt]
\tikzstyle{dot_style}=[circle, draw,fill=black,radius=0.002pt,inner sep=1pt]

% For vertical centering
%\newcommand{\vc}[1]{\ensuremath{\vcenter{\hbox{#1}}}}

\begin{center}
\begin{tikzpicture}[scale=0.8]

\node[vertex_style](1) at (2,2) {$ v_1 $};
\node[vertex_style](2) at (4,2) {$ v_2 $};
\node[vertex_style](3) at (0,0) {$ v_3 $};
\node[vertex_style](4) at (6,0) {$ v_4 $};
\node[vertex_style](5) at (2,-2) {$ v_5 $};
\node[vertex_style](6) at (4,-2) {$ v_6 $};

\draw (1) --node[above]{$1$} (2);
\draw (2)--node[right]{$3$} (6) ;
\draw (2)--node[right]{$2$} (4) ;
\draw (3) --node[below]{$3$} (2);
\draw (3)--node[above]{$2$} (6) ;
\draw (3)--node[above]{$2$} (1) ;
\draw (3)--node[below]{$2$} (5) ;
\draw (4)  --node[right]{$1$} (6);
\draw (5) --node[below]{$ 1 $} (6);
\end{tikzpicture}
\end{center}
Start with $ T=(\{V(G)\},\emptyset) $:
\vc{
\begin{tikzpicture}
\node[vertex_style] at (0,0) {\begin{tikzpicture}
\node[dot_style](1) at (0.5,0.5) {};
\node[dot_style](2) at (1,0.5) {};
\node[dot_style](3) at (0,0) {};
\node[dot_style](4) at (1.5,0) {};
\node[dot_style](5) at (0.5,-0.5) {};
\node[dot_style](6) at (1,-0.5) {};

\draw (1) -- (2);
\draw (2)--(6) ;
\draw (2)--(4) ;
\draw (3) --(2);
\draw (3)--(6) ;
\draw (3)--(1) ;
\draw (3)--(5) ;
\draw (4)  --(6);
\draw (5) --(6);
\end{tikzpicture}};
\end{tikzpicture}
}
pick $ v_1,v_2 $, then the cut is
\vc{
\begin{tikzpicture}
\node[dot_style](1) at (1,1) {};
\node[dot_style](2) at (2,1) {};
\node[dot_style](3) at (0,0) {};
\node[dot_style](4) at (3,0) {};
\node[dot_style](5) at (1,-1) {};
\node[dot_style](6) at (2,-1) {};

\draw (1) -- (2);
\draw (2)--(6) ;
\draw (2)--(4) ;
\draw (3) --(2);
\draw (3)--(6) ;
\draw (3)--(1) ;
\draw (3)--(5) ;
\draw (4)  --(6);
\draw (5) --(6);
\draw[color=red,line width=1pt] (0,0.3)--(1.7,1.2);
\end{tikzpicture}
}\\
 Then $ T $ is 
 \vc{
 \begin{tikzpicture}
 \node[vertex_style](X) at (0,0) {\begin{tikzpicture}
 
 \node[dot_style](2) at (1,0.5) {};
 \node[dot_style](3) at (0,0) {};
 \node[dot_style](4) at (1.5,0) {};
 \node[dot_style](5) at (0.5,-0.5) {};
 \node[dot_style](6) at (1,-0.5) {};
 
 \draw (2)--(6) ;
 \draw (2)--(4) ;
 \draw (3) --(2);
 \draw (3)--(6) ;
 \draw (3)--(5) ;
 \draw (4)  --(6);
 \draw (5) --(6);
 \end{tikzpicture}};
 \node[vertex_style](1) at (-2,0) {$ v_1 $};
 \draw (1)--node[above]{3}(X);
 \end{tikzpicture}
 }
 pick $ v_2,v_3 $, then the cut is
 \vc{
 \begin{tikzpicture}
 \node[dot_style](1) at (1,1) {};
 \node[dot_style](2) at (2,1) {};
 \node[dot_style](3) at (0,0) {};
 \node[dot_style](4) at (3,0) {};
 \node[dot_style](5) at (1,-1) {};
 \node[dot_style](6) at (2,-1) {};
 
 \draw (1) -- (2);
 \draw (2)--(6) ;
 \draw (2)--(4) ;
 \draw (3) --(2);
 \draw (3)--(6) ;
 \draw (3)--(1) ;
 \draw (3)--(5) ;
 \draw (4)  --(6);
 \draw (5) --(6);
 \draw[color=red, line width=1pt] (1.5,-1.2)--(1.5,1.2);
 \end{tikzpicture}
 }\\
 Then $ T $ is
 \vc{
  \begin{tikzpicture}
  \node[vertex_style](X) at (0,0) {\begin{tikzpicture}
  
  \node[dot_style](2) at (1,0.5) {};
  \node[dot_style](4) at (1.5,0) {};
  \node[dot_style](6) at (1,-0.5) {};
  
  \draw (2)--(6) ;
  \draw (2)--(4) ;
  \draw (4)  --(6);
  \end{tikzpicture}};
  \node[vertex_style](1) at (-5,0) {$ v_1 $};
  \node[vertex_style](Y) at (-3,0) {\begin{tikzpicture}
    
    \node[vertex_style](3) at (0,0.5) {$ v_3 $};
    \node[vertex_style](5) at (0,-0.5) {$ v_5 $};
    
    \draw (3)--(5) ;
    \end{tikzpicture}};
  \draw (1)--node[above]{3}(Y);
  \draw(X)--node[above]{7}(Y);
  \end{tikzpicture}
  }
pick $ v_3,v_5 $, then the cut is
\vc{
 \begin{tikzpicture}
 \node[dot_style](1) at (1,1) {};
 \node[vertex_style](3) at (0,0) {$ v_3 $};
 \node[dot_style](246) at (2,0) {};
 \node[vertex_style](5) at (1,-1) {$ v_5 $};
 
 \draw (3)--node[above]{5}(246) ;
 \draw (246)--node[above]{1}(1) ;
 \draw (3)--node[left]{2}(1) ;
 \draw (3)--node[below]{2}(5) ;
 \draw (5) --node[below]{1}(246);
 \draw[color=red, line width=1pt] (0,-0.5)--(2,-0.5);
 \end{tikzpicture}
 }
 \\
 Then $ T $ is
 \vc{
   \begin{tikzpicture}
   \node[vertex_style](X) at (0,0) {\begin{tikzpicture}
   
   \node[dot_style](2) at (1,0.5) {};
   \node[dot_style](4) at (1.5,0) {};
   \node[dot_style](6) at (1,-0.5) {};
   
   \draw (2)--(6) ;
   \draw (2)--(4) ;
   \draw (4)  --(6);
   \end{tikzpicture}};
   \node[vertex_style](1) at (-3,0) {$ v_1 $};
   \node[vertex_style](3) at (-1.5,0) {$ v_3 $};
   \node[vertex_style](5) at (-1.5,-1) {$ v_5 $};
   \draw(X)--node[above]{7}(3);
   \draw(1)--node[above]{3}(3);
   \draw(5)--node[right]{3}(3);
   \end{tikzpicture}
   }
   pick $ v_2,v_4 $, then the cut is
   \vc{
   \begin{tikzpicture}
   \node[dot_style](135) at (0,0){};
   \node[vertex_style](2) at (1,1){$ v_2 $};
   \node[vertex_style](4) at (2,0){$ v_4 $};
   \node[vertex_style](6) at (1,-1){$ v_6 $};
   \draw(2)--node[above]{4}(135);
   \draw(2)--node[above]{2}(4);
   \draw(2)--node[right]{3}(6);
   \draw(6)--node[below]{1}(4);
   \draw(6)--node[below]{3}(135);
   \draw[color=red, line width=1pt](1.4,-1.2)--(1.4,1.2);
   \end{tikzpicture}
   }
   \\
 Then $ T $ is
\vc{
   \begin{tikzpicture}
   \node[vertex_style](X) at (0,0) {\begin{tikzpicture}
   
   \node[vertex_style](2) at (0,0.5) {$ v_2 $};
   \node[vertex_style](6) at (0,-0.5) {$ v_6 $};
   
   \draw (2)--(6) ;
   \end{tikzpicture}};
   \node[vertex_style](1) at (-3.5,0) {$ v_1 $};
   \node[vertex_style](3) at (-2,0) {$ v_3 $};
   \node[vertex_style](5) at (-2,-1.3) {$ v_5 $};
   \node[vertex_style](4) at (2,0){$ v_4 $};
   \draw(X)--node[above]{7}(3);
   \draw(1)--node[above]{3}(3);
   \draw(5)--node[right]{3}(3);
   \draw(X)--node[above]{3}(4);
   \end{tikzpicture}
}
   pick $ v_2,v_6 $, then the cut is
\vc{
   \begin{tikzpicture}
	 \node[dot_style](135) at (0,0){};
     \node[vertex_style](2) at (1,1){$ v_2 $};
     \node[dot_style](4) at (2,0){};
     \node[vertex_style](6) at (1,-1){$ v_6 $};
     \draw(2)--node[above]{4}(135);
     \draw(2)--node[above]{2}(4);
     \draw(2)--node[right]{3}(6);
     \draw(6)--node[below]{1}(4);
     \draw(6)--node[below]{3}(135);
     
     \node(l)at (0,-1){};
     \node(r)at (2,-1){};
     \draw[color=red, line width=1pt](l.90)..controls +(90:5mm) and +(90:10mm)..(r.90);
    
   \end{tikzpicture}
 }\\
    Then $ T $ is
    \vc{
      \begin{tikzpicture}
      \node[vertex_style](2) at (0,0) {$ v_2 $};
      \node[vertex_style](1) at (-3,0) {$ v_1 $};
      \node[vertex_style](3) at (-1.5,0) {$ v_3 $};
      \node[vertex_style](5) at (-1.5,-1.5) {$ v_5 $};
      \node[vertex_style](4) at (1.5,0){$ v_4 $};
      \node[vertex_style](6) at (0,-1.5){$ v_6 $};
      \draw(2)--node[above]{7}(3);
      \draw(1)--node[above]{3}(3);
      \draw(5)--node[right]{3}(3);
      \draw(2)--node[above]{3}(4);
      \draw(2)--node[right]{7}(6);
      
      \end{tikzpicture}}
      
So in the form of $ G $:
\vc{
      \begin{tikzpicture}[thick,scale=0.8]
      
      \node[vertex_style](1) at (2,2) {$ v_1 $};
      \node[vertex_style](2) at (4,2) {$ v_2 $};
      \node[vertex_style](3) at (0,0) {$ v_3 $};
      \node[vertex_style](4) at (6,0) {$ v_4 $};
      \node[vertex_style](5) at (2,-2) {$ v_5 $};
      \node[vertex_style](6) at (4,-2) {$ v_6 $};
      
      
      \draw (2)--node[right]{$7$} (6) ;
      \draw (2)--node[right]{$3$} (4) ;
      \draw (3) --node[below]{$7$} (2);
      \draw (3)--node[above]{$3$} (1) ;
      \draw (3)--node[below]{$3$} (5) ;
      \end{tikzpicture}
      }
      
Note that the tree is not unique and that there are many different ones.
The tree depends on the order of cuts. But it should still have
a path of length 2 with edges of value 7 and three additional leaves with edges
of value 3.\\
}    


%\emph{Next time: Minimum Cost Flows - only briefly}

\end{document}






