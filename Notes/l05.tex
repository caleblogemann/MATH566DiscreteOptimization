\documentclass[11pt]{article}
\usepackage[utf8]{inputenc}
\usepackage[czech, english]{babel}
\usepackage[left=2cm,right=2cm,top=2cm,bottom=2cm]{geometry}
\usepackage{amssymb}
\usepackage{amsthm}
\usepackage{amsmath}
\usepackage{tikz}


% Itemize environment with small skips
\newenvironment{packeditemize}{
\begin{itemize}
  \setlength{\itemsep}{1pt}
  \setlength{\parskip}{0pt}
  \setlength{\parsep}{0pt}
}{\end{itemize}}


% Fancy footnote....
\usepackage{fancyhdr}
\pagestyle{fancy}
\usepackage{lastpage}
\rfoot{MATH 566 - 05, page \thepage/\pageref{LastPage}}
\cfoot{}
\rhead{}
\lhead{}
\renewcommand{\headrulewidth}{0pt}
\renewcommand{\footrulewidth}{0pt}


\newtheorem{theorem*}{Theorem} 

% Itemize environment with small skips
\newenvironment{packedenumerate}{
\begin{enumerate}
  \setlength{\itemsep}{1pt}
  \setlength{\parskip}{0pt}
  \setlength{\parsep}{0pt}
}{\end{enumerate}}


\newcounter{excounter}
\setcounter{excounter}{1}
\newcommand\question[2]{\vskip 1em  \noindent\textbf{\arabic{excounter}\addtocounter{excounter}{1}:} \emph{#1} \noindent#2}
\newcommand\solution[1]{\vskip 0.5em \noindent\textbf{Solution:} #1}
\newcommand\like{\par \noindent\emph{(This question is: good - bad - ugly? Difficulty: 0-10:\hskip 3em )}} 

\newcommand\lecturer[1]{\textbf{#1}}
\newcommand\hideforstudent[1]{#1}


%\renewcommand\lecturer[1]{{\color{white} \textbf{#1} }}
%\renewcommand\hideforstudent[1]{{\color{white} #1 }}
%\renewcommand\solution[1]{{\color{white} #1 }}



\setlength{\parindent}{0cm}
\setlength{\parskip}{0.1cm}

\begin{document}

Fall  2016, MATH-566

\centerline{{\Large \textbf{Separation theorem}}}

How to show that two convex sets are disjoint?

\begin{theorem*}
Let $C,D \subseteq \mathbb{R}^d$ are convex sets and $C \cap D = \emptyset$ then there exists a hyperplane separating $C$ and $D$.
That is, exists $\mathbf{a} \in \mathbb{R}^d$, $b \in \mathbb{R}$ such that \\
$\forall \mathbf{x} \in C,  \mathbf{a}^T \mathbf{x} \leq b$ \\
$\forall \mathbf{x} \in D,  \mathbf{a}^T \mathbf{x} \geq b$ 
\end{theorem*}
Separation can be strict if $C$ and $D$ closed and one bounded.
\begin{center}
\begin{tikzpicture}
\draw(0,0) node[circle,draw]{$C$} (2,0) node[rectangle,draw]{$D$};
\draw(1,-1) -- (1,1) node[label=right:{$\textbf{a}^T\textbf{x}=\textbf{b}$}]{} ;
\end{tikzpicture}
\end{center}

\question{}{Why is the theorem true if $C$ and $D$ are compact?}

\solution{
If both $C$ and $D$ compact, consider $\mathbf{c} \in C$ and $\mathbf{d} \in D$ such that $\|\mathbf{c}-\mathbf{d}\|$ is minimized.
That is, $\mathbf{c}$ and $\mathbf{d}$ are the closest points. They exist since $C$ and $D$ are compact and the
distance is a continuous function. Let $z = \frac{\mathbf{c}+\mathbf{d}}{2}$, that is, $z$ is in the middle
between $\mathbf{c}$ and $\mathbf{d}$. Considet a hyperplane $H$ perpendicular to segment $\mathbf{c}\mathbf{d}$
and containing $\mathbf{z}$. Claim is that it is the desired hyperplane $H$. 
If $H$ is not separating, then we find a contradiction with $\mathbf{c}$ and $\mathbf{d}$ being the closest.
}

\question{}{Why is the theorem true if $C$ compact and $D$ closed?}
\solution{
Assume $D_M$ being $D$ intersected with a huge, but still bounded set $M$. Then we can still argue that there are two
closest points and eventually it will be the closest points even if $M$ grows.
\vskip 4em
}

\question{}{Why is the theorem true in general?}
\solution{
Suppose $C$ and $D$ are general convex sets. We create a sequence of compact sets, that approximate $C_i$ and $D_i$ that are in limit going to $C$ and $D$. We can separate  $C_i$ and $D_i$  by the first case by a hyperplane $H_i$. It it possible to show that $H_i$ is converging and it is converging to a separating hyperplane for $C$ and $D$.

Recall that $B(\mathbf{s},r) = \{ \mathbf{x} : \|\mathbf{x}-\mathbf{s}\| \leq r  \}$ is a ball centered at $\mathbf{s}$ of radius $r$.
Suppose $\mathbf{0} \in C$. We define $C_1 \subseteq C_2 \subseteq C_3 \subseteq \cdots$ by letting $C_i = \left(1-\frac{1}{i}\right)C \cap B(\mathbf{0}, i)$. 
Notice that $C_i$'s are compact and $C = \cup_i C_i$.

We create similar sets for $D_i$ and  then $H_i$ is a hyperplane separating $C_i$ from $D_i$.
}

\solution{
Note that compactness of at least one of the sets is needed for for strict separation.
}





\end{document}



