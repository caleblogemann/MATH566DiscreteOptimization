\documentclass[11pt]{article}
\usepackage[utf8]{inputenc}
\usepackage[czech, english]{babel}
\usepackage[left=2cm,right=2cm,top=2cm,bottom=2cm]{geometry}
\usepackage{amssymb}
\usepackage{amsthm}
\usepackage{amsmath}
\usepackage{tikz}
\usetikzlibrary{shapes.geometric}
\usepackage{hyperref}


% Itemize environment with small skips
\newenvironment{packeditemize}{
\begin{itemize}
  \setlength{\itemsep}{1pt}
  \setlength{\parskip}{0pt}
  \setlength{\parsep}{0pt}
}{\end{itemize}}


% Fancy footnote....
\usepackage{fancyhdr}
\pagestyle{fancy}
\usepackage{lastpage}
\rfoot{MATH 566 - 12, page \thepage/\pageref{LastPage}}
\cfoot{}
\rhead{}
\lhead{}
\renewcommand{\headrulewidth}{0pt}
\renewcommand{\footrulewidth}{0pt}


\newtheorem{theorem*}{Theorem} 

% Itemize environment with small skips
\newenvironment{packedenumerate}{
\begin{enumerate}
  \setlength{\itemsep}{1pt}
  \setlength{\parskip}{0pt}
  \setlength{\parsep}{0pt}
}{\end{enumerate}}


\newcounter{excounter}
\setcounter{excounter}{1}
\newcommand\question[2]{\vskip 1em  \noindent\textbf{\arabic{excounter}\addtocounter{excounter}{1}:} \emph{#1} \noindent#2}
\newcommand\solution[1]{\vskip 0.5em \noindent\textbf{Solution:} #1}
\newcommand\like{\par \noindent\emph{(This question is: good - bad - ugly? Difficulty: 0-10:\hskip 3em )}} 

\newcommand\lecturer[1]{\textbf{#1}}
\newcommand\hideforstudent[1]{#1}


%\renewcommand\lecturer[1]{{\color{white} \textbf{#1} }}
%\renewcommand\hideforstudent[1]{{\color{white} #1 }}
%\renewcommand\solution[1]{{\color{white} \vskip 0.5em \noindent\textbf{Solution:} #1 }}



\setlength{\parindent}{0cm}
\setlength{\parskip}{0.1cm}

\begin{document}

Fall  2016, MATH-566

\centerline{{\Large \textbf{Shortest path}}}


Source: Chapter 2.2 (Bills), Chapter 7 of Combinatorial Optimization (Korte)


\textbf{Shortest path}\\
\emph{Input:}     Graph $G=(V,E)$, costs $c: E \rightarrow \mathbb{R}$, and $s,t \in V$.\\
\emph{Output:}  $s$-$t$-path $P$, where $\sum_{e \in P}c(e)$ is minimized.


\question{}{
Find shortest (lowest cost) $s$-$t$-paths in the following graphs
\begin{center}
\begin{tikzpicture}[scale=2]
\draw (0,0) grid (2,2);
\fill
\foreach \x in {0,1,2}{
\foreach \y in {0,1,2}{ 
(\x,\y) circle (1.5pt)
}
}
;
\draw 
(0,0) node[left]{$s$}
(2,2) node[right]{$t$}
(0.5,0) node[below]{1}
(1.5,0) node[below]{2}
(0.5,1) node[below]{6}
(1.5,1) node[below]{3}
(0.5,2) node[below]{9}
(1.5,2) node[below]{7}
(0,0.5) node[left]{13}
(0,1.5) node[left]{8}
(1,0.5) node[left]{22}
(1,1.5) node[left]{10}
(2,0.5) node[left]{14}
(2,1.5) node[left]{18}
;
\end{tikzpicture}
\hskip 3em
\begin{tikzpicture}[scale=2]
\draw (0,0) grid (2,2);
\fill
\foreach \x in {0,1,2}{
\foreach \y in {0,1,2}{ 
(\x,\y) circle (1.5pt)
}
}
;
\draw 
(0,0) node[left]{$s$}
(2,2) node[right]{$t$}
(0.5,0) node[below]{1}
(1.5,0) node[below]{2}
(0.5,1) node[below]{6}
(1.5,1) node[below]{3}
(0.5,2) node[below]{-9}
(1.5,2) node[below]{7}
(0,0.5) node[left]{13}
(0,1.5) node[left]{-8}
(1,0.5) node[left]{22}
(1,1.5) node[left]{10}
(2,0.5) node[left]{14}
(2,1.5) node[left]{18}
;
\end{tikzpicture}
\end{center}
}


Cost $c$ is called \emph{conservative} if there is no circuit of negative total weight.

\textbf{Bellman's principle:}
Let $s,\ldots,v,w$ be the least cost $s$-$w$-path of length $k$.
The $s,\ldots,v$ is the least cost $s$-$v$-path of length $k-1$.

\question{Prove Bellman's principle.}
\solution{
By contradiction. If there is a lesser cost path to $v$, we could find a lesser cost path to $w$.
}

Notice: This gives a recursion for computing the shortest path.

\textbf{Dijkstra's algorithm}\\
$c:E \rightarrow \mathbb{R}_+$, computes shortest $s$-$t$-path from $s$ to ALL other vertices $t \in V$.

\begin{enumerate}
\item $l(s) := 0$; $\forall v\neq s$ $l(v) = +\infty$
\item $R = \emptyset$
\item while $R \neq V$
\item  \hskip 3em  find $v\in V -R$ with minimum $l(v)$
\item  \hskip 3em  $R := R \cup \{v\}$
\item  \hskip 3em  $\forall vw \in E$, $l(w) = \min\{l(w), l(v)+c(v,w)\}$ 
\end{enumerate}

$R$ \ldots vertices with final number;
$l$ \ldots upper bound on the cost;\\
Running time $O(n^2)$ easily, $O(m+n\log n)$ with Fibonacci heaps.

\newpage
\question{}{ Run Dijkstra's algorithm on the following graph
\begin{center}
\begin{tikzpicture}[scale=2]
\draw (0,0) grid (2,1);
\fill
\foreach \x in {0,1,2}{
\foreach \y in {0,1}{ 
(\x,\y) circle (1.5pt)
}
}
;
\draw 
(0,0) node[left]{$s$}
(1,1) node[above]{$t$}
(0.5,0) node[below]{1}
(1.5,0) node[below]{2}
(0.5,1) node[below]{7}
(1.5,1) node[below]{3}
(0,0.5) node[left]{13}
(1,0.5) node[left]{22}
(2,0.5) node[left]{13}
;
\end{tikzpicture}
\end{center}
}

\question{}{
How to get shortest $s$-$v$-path? 
}
\solution{}{
Remember previous vertex. In step 6. of the algorithm, remember why the value was changed. So called \emph{predecessor}.
}


\question{}{
Why is the algorithm correct?
(show that if $v \in R$, then $l(v) = $ cost for $s$-$v$-path.)
}
\solution{
Could be done for example by contradiction. Suppose that there
is a closer one. Then take the one with lowest $s$-$v$-costs that is not
the same as the shortest path. And look at the predecessor on the shortest
$s$-$v$-path. It gives contradiction with the run of the algorithm.
}

\question{}{
Why Dijkstra's algorithm does not work for negative costs?
}
\solution{
For simplicity consider directed graph problem. The assumption that
we can fix a cost of the lowest visited so far is not true.
\begin{center}
\begin{tikzpicture}
\draw(0,0) node[inner sep=1.5pt,fill,circle,label=left:$s$](a){} --node[pos=0.5,below]{4} (1,0) node[inner sep=1.5pt,fill,circle](b){}
--node[pos=0.5,below]{-3} (2,0) node[inner sep=1.5pt,fill,circle,label=right:$t$](c){}
(a) to[bend left] node[pos=0.5,above]{2}  (c)
;
\end{tikzpicture}
\end{center}
}



\textbf{Moore-Bellman-Ford Algorithm}\\
$c:E \rightarrow \mathbb{R}$, computes shortest $s$-$t$-path from $s$ to ALL other vertices $t \in V$ \textbf{OR} finds a cycle of negative cost. Assume $|V(G)| = n$.

\begin{enumerate}
\item $l(s) := 0$; $\forall v\neq s$ $l(v) = +\infty$
\item repeat $n-1$ times:    // computes the costs
\item  \hskip 2em  $\forall vw \in E$, 
\item  \hskip 4em           if $l(w) > l(v)+c(v,w)$ 
\item  \hskip 6em                $l(w) := l(v)+c(v,w)$;  $p(w) = v$
\item    $\forall vw \in E$,   // check for a negative cycle
\item  \hskip 2em           if $l(w) > l(v)+c(v,w)$ then found negative cycle
\end{enumerate}

Note: $l$ gives the least cost, while $p$ gives the $\textbf{previous}$ vertex / \textbf{predecesor} on the shortest path from $s$.

\question{}{
What is the time complexity of the algorithm if $G$ has $m$ edges and $n$ vertices?
}
\solution{
$O(nm)$.
}

\question{}{
Why the algorithm detects a negative cycle and why the algorithm works?
}

\end{document}
\solution{
\vskip 5em
}


\end{document}


\newpage
\textbf{Linear programming formulation:}\\
Suppose every edge has an orientation (direction). This gives a directed graph.
Normal graph can be converted to directed by adding two opposite edges.
Note: Previous algorithms work on directed graphs.

\question{}{
Create a linear program solving the shortest path problem. 
Hints: Minimize, overall cost, for every edge decide if it is in the path or not, make sure that
the path starts at $s$ (and ends at $t$). Make sure that the path does not stop at any other vertex (use that edges are oriented and you know incomming and leaving edges).
}
\solution{
\[
\begin{cases}
\text{minimize}   &  \sum_{e \in E} c(e)\cdot x_e \\
\text{subject to}  & \sum_{(s,v) \in E} x_{s,v} = 1 \\
                           & \sum_{(v,t) \in E} x_{v,t} = 1 \\
                           & \sum_{(u,v) \in E} x_{u,v} =  \sum_{((v,w) \in E) x_{v,w}} \text{ for all } v \neq s,t\\
                           & \cdot x_e  \geq 0 \text{ for all } e \in E
\end{cases}
\]
}

\question{}{
Write the linear program for graph with directed edges $E=\{su,sv,uv,ut,vt\}$, where
the costs are $c(su) = 2, c(sv) = 5, c(uv) = 1, c(ut) = 6, c(vt) = 3$.
}
\solution{

}


\question{}{
Write the dual linear program for shortest path.
}
\solution{
\[
\begin{cases}
\text{minimize}   & y_t - y_s \\
\text{subject to}  & y_u - y_v \leq c(v,u) \text{ for all  } (u,v) \in E \\
                           & y_s = 0  \text{ add this boy... } 
\end{cases}
\]
}

\question{}{
Interpret the dual program.
}

%\question{}{
%Does the LP work for negative costs?
%}
%\solution{
%No.
%}

\end{document}






