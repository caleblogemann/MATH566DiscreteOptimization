\documentclass[11pt]{article}
\usepackage[utf8]{inputenc}
\usepackage[czech, english]{babel}
\usepackage[left=2cm,right=2cm,top=2cm,bottom=2cm]{geometry}
\usepackage{amssymb}
\usepackage{amsthm}
\usepackage{amsmath}
\usepackage{tikz}
\usetikzlibrary{shapes.geometric}
\usetikzlibrary{shapes.multipart}
\usetikzlibrary{arrows}
\usepackage{hyperref}


% Itemize environment with small skips
\newenvironment{packeditemize}{
\begin{itemize}
  \setlength{\itemsep}{1pt}
  \setlength{\parskip}{0pt}
  \setlength{\parsep}{0pt}
}{\end{itemize}}


% Fancy footnote....
\usepackage{fancyhdr}
\pagestyle{fancy}
\usepackage{lastpage}
\rfoot{MATH 566 - 21, page \thepage/\pageref{LastPage}}
\cfoot{}
\rhead{}
\lhead{}
\renewcommand{\headrulewidth}{0pt}
\renewcommand{\footrulewidth}{0pt}


\newtheorem{theorem*}{Theorem} 

% Itemize environment with small skips
\newenvironment{packedenumerate}{
\begin{enumerate}
  \setlength{\itemsep}{1pt}
  \setlength{\parskip}{0pt}
  \setlength{\parsep}{0pt}
}{\end{enumerate}}


\newcommand{\vc}[1]{\ensuremath{\vcenter{\hbox{#1}}}}

\newcounter{excounter}
\setcounter{excounter}{1}
\newcommand\question[2]{\vskip 1em  \noindent\textbf{\arabic{excounter}\addtocounter{excounter}{1}:} \emph{#1} \noindent#2}
\newcommand\solution[1]{\vskip 0.5em \noindent\textbf{Solution:} #1}
%\newcommand\like{\par \noindent\emph{(This question is: good - bad - ugly? Difficulty: 0-10:\hskip 3em )}} 

\newcommand\lecturer[1]{\textbf{#1}}
\newcommand\hideforstudent[1]{#1}


%\renewcommand\lecturer[1]{{\color{white} \textbf{#1} }}
%\renewcommand\hideforstudent[1]{{\color{white} #1 }}
%\renewcommand\solution[1]{{\color{white} \vskip 0.5em \noindent\textbf{Solution:} #1 }}



\setlength{\parindent}{0cm}
\setlength{\parskip}{0.1cm}

\begin{document}

Fall  2016, MATH-566

\centerline{{\Large \textbf{Integer Programming - Unimodularity}}}

Source: Bill,Bill,Bill,Alex book, chapter 6.5

Problem:
\[
(IP)
\begin{cases}
\text{maximize} & \mathbf{c}^T\mathbf{x} \\
\text{subject to} & A\mathbf{x} \leq \mathbf{b},
\end{cases}
\]
where $\mathbf{c} \in \mathbb{Z}^n, \mathbf{b} \in \mathbb{Z}^m, A \in \mathbb{Z}^{m \times n}$, and $\mathbf{x} \in \mathbb{Z}^n$.

Let $P = \{\mathbf{x} \in \mathbb{R}^n: A\mathbf{x} \leq \mathbf{b} \}$ be a polyhedron.
Let $P_I = conv(\{\mathbf{x} \in \mathbb{Z}^n: A\mathbf{x} \leq \mathbf{b} \})$ be the convex hull of integer points in $P$. 
If $A$ and $\mathbf{b}$ are rational, $P$ is called \emph{rational polyhedra}.
%(if $P$ is rational, then $P_I$ is also a polyhedron, if $P$ not rational, $P_I$ does not have to be polyhedron)
\begin{center}
\tikzset{insep/.style={inner sep=2pt, outer sep=0pt, circle, fill},}
\begin{tikzpicture}[scale=1]
\draw[fill=gray!20]
(0,1) -- (2,1) -- (2,2) -- (1,2) -- cycle
;
\foreach \x in {0,1,2,3}{
\foreach \y in {0,1,2,3}{
\draw (\x,\y) node[insep](x\x\y){};
}
\draw
(0,1) -- (2.7,0.5) -- (2.6,2.8) -- (0.3,2.2) -- cycle
;
\draw
(1.5,1.5) node{$P_I$}
(2.4,0.8) node{$P$}
;
}

\end{tikzpicture}
\end{center}
Clearly, $P_I \subseteq P$. Polyhedron $P$ is \textbf{integral} if $P = P_I$.
(or if every face of  $P$ contains an integral vector)
\\
If $P$ is integral, then $(IP)$ can be solved as linear programming.

\question{}{
Why $(IP)$ cannot be always solved by linear program and then rounding?
}
\solution{
Rounding may not be possible.
\tikzset{insep/.style={inner sep=2pt, outer sep=0pt, circle, fill},}
\vc{
\begin{tikzpicture}[scale=1]
\foreach \x in {0,1,2,3}{
\foreach \y in {0,1}{
\draw (\x,\y) node[insep](x\x\y){};
}
\draw
(0,0.2) -- (2.5,0.5) -- (0,0.8)
;
\draw
(0,0.5) node{$P$}
;
}
\end{tikzpicture}
}
}

\textbf{Theorem 6.22}
Rational polytope $P$ is integral iff for all  $\mathbf{w} \in \mathbb{Z}^n$, the value of $\max\{ \mathbf{w}^T\mathbf{x}: \mathbf{x} \in P \}$ is $\in \mathbb{Z}$.

\question{}{
Prove Theorem 6.22. One direction is easy.
Other direction: Let $\mathbf{v} \in P$ be the unique optimal solution corresponding to 
$\mathbf{w}$.\ldots} show $\mathbf{v}$ has integer coordinates.
\solution{ By multiplying $\mathbf{w}$ by a constant assume that for all other vertices $\mathbf{u} \neq\mathbf{v}$:
\[
\mathbf{w}^T\mathbf{v} > \mathbf{w}^T\mathbf{u} + \mathbf{u}_1-\mathbf{v}_1
\]
Hence $\mathbf{v}$ is optimal also for $\mathbf{z} = (\mathbf{w}_1+1,\mathbf{w}_2,\ldots)$.
Then $\mathbf{z}^T\mathbf{v} = \mathbf{w}^T\mathbf{v} + \mathbf{v}_1$
Since we assumed $\mathbf{z}^T\mathbf{v}$ and $\mathbf{w}^T\mathbf{v}$ are integral, 
also $\mathbf{v}_1$ is integral. Repeat for other components of $\mathbf{v}$.
}

\textbf{What guarantees and integral polyhedra?}

Recall $A^{-1} = \frac{1}{\det(A)} A^\star$, where $A_{i,j}^\star = \det(A_{-i,-j})$.

For square matrices:

\textbf{Theorem 6.23}
Let $A \in \mathbb{Z}^{m \times m}$. Then $A^{-1}\mathbf{b}$ is integral for every $\mathbf{b} \in \mathbb{Z}^n$ iff $\det(A) \in \{1,-1\}$. 

\question{}{Prove Theorem 6.23}
\solution{
$\Leftarrow$ Let $\det(A) = \pm1$. By Cramer's rule, also $A^{-1}$ is integral. Hence $A^{-1}\mathbf{b}$ is integral.\\
$\Rightarrow$ If $\mathbf{b}$ is $i$th unit vector, then $A^{-1}\mathbf{b}$ is $i$th column of $A^{-1}$. Hence $A^{-1}$ is integral and $\det(A^{-1})$ is an integer. 
Since $1=\det(AA^{-1}) = \det(A)\cdot\det(A^{-1})$ and $\det(A) \in \mathbb{Z}$  we conclude that   $\det(A) = \det(A^{-1}) = \pm1$.
}

\vskip 1em 

For rectangular matrix:\\
Matrix $A \in \mathbb{Z}^{m\times n}$ of full row rank is \textbf{unimodular} if every 
$m\times m$ basis of $A$ (full rank square submatrix) has determinant $\pm 1$. 

\vskip 1em 

\textbf{Theorem 6.24}
Let $A \in \mathbb{Z}^{m \times n}$ be of full row rank.
Polyhedron  $P = \{\mathbf{x}: A\mathbf{x} = \mathbf{b}, \mathbf{x} \geq \mathbf{0}\}$
is integral for every $\mathbf{b} \in \mathbb{Z}^m$ iff $A$ is unimodular.  

\question{}{Prove Theorem 6.24}
\solution{
Recall in LP: A solution $\mathbf{x}$ is called basic feasible solution if $\mathbf{x}$ has at most $m$ non-zero entires and the columns of $A$ corresponding to these entries are linearly independent.

$\Leftarrow$ Let $\overline{\mathbf{x}} \in P$ be a vertex (using $\mathbf{x} \geq 0$).  Pick basis $B$ corresponding to $\overline{\mathbf{x}}$ - pick columns where $\overline{\mathbf{x}}$ is nonzero (and extend). 
Use Theorem 6.23 on
$B\overline{\mathbf{x}} = \mathbf{b}$. 

$\Rightarrow$ Let $B$ be a base of $A$ and pick any $\mathbf{v} \in \mathbb{Z}^n$. 
Goal is to show that $B^{-1}\mathbf{v} \in \mathbb{Z}^m$ and use Theorem 6.23
to show $\det(B) = \pm 1$.
Choose $\mathbf{y} \in \mathbb{Z}^m$ such that $B^{-1}\mathbf{v}  + \mathbf{y} \geq  \mathbf{0}$.
Let $\mathbf{b} = B(B^{-1}\mathbf{v}  + \mathbf{y}) = \mathbf{v}  + B\mathbf{y} \in \mathbb{Z}^m$. Add zero components to $(B^{-1}\mathbf{v}  + \mathbf{y})$, which gives $\mathbf{z}\in \mathbf{R}^n$ such that $A\mathbf{z} = \mathbf{b}$. Now $\mathbf{z} \in P$ and it corresponds to a basic feasible solution, hence $\mathbf{z} \in \mathbb{Z}^n$. Therefore $B^{-1}\mathbf{v} \in \mathbb{Z}^m$.
}


\vskip 1em 

Matrix $A \in \mathbb{Z}^{m\times n}$ is \textbf{totally unimodular} if every square submatrix has determinant in $\{0,1,-1\}$. (all entries of $A$ are in $\{0,1,-1\}$. )

\vskip 1em 

HW question: $A$ is totally unimodular iff $[A\ I]$ is unimodular (where $I$ is $m \times m$ unit matrix).

\vskip 1em 

\textbf{Theorem 6.25}
Let $A \in \mathbb{Z}^{m \times n}$.
Polyhedron $P=\{A\mathbf{x} \leq \mathbf{b}, \mathbf{x} \geq \mathbf{0}\}$
is integral for every $\mathbf{b} \in \mathbb{Z}^m$ iff $A$ is totally unimodular.  

\vskip 1em

\textbf{Theorem 6.26}
Let $A \in \mathbb{Z}^{m \times n}$.
Polyhedron $P=\{A\mathbf{x} \leq \mathbf{b}\}$ is integral for every $\mathbf{b} \in \mathbb{Z}^m$ iff $A$ is totally unimodular.  

\vskip 1em

Note: Matrix $A$ being totally unimodular is decidable in polynomial time (algorithm by Seymour).

\question{}{
Let $A$ have values $\{0,1,-1\}$ and every column has at most one 1 and at most one -1. Show that $A$ is totally unimodular. \emph{Hint: induction.}
}
\solution{
Let $N$ be a $k\times k$ submatrix. If $k=1$ clear.  If column with at most one non-zero, expand the determinant. If all columns have 1 and -1, matrix is signular.
}

\vskip 1em

Example: Incidence matrix $M\in \mathbb{R}^{|V|\times |E|}$ of directed graph $G=(V,E)$ is totally unimodular.
Matrix $M$ is indexed by $V$ and $E$. Edge $e=\overrightarrow{uv}\in E$
gives entries 
$M_{ue} = -1$ and $M_{ve} = 1$.

\begin{center}
\tikzset{insep/.style={inner sep=2pt, outer sep=0pt, circle, fill},}
\begin{tikzpicture}
\draw[-triangle 45]
(0,0) node[insep,label=below:$v_1$]{}  -- node[above]{$e_1$} (1,0) node[insep,label=below:$v_2$]{} 
;
\draw[-triangle 45]
(1,0) --  node[above]{$e_2$} (2,0) node[insep,label=below:$v_3$]{} 
;
\end{tikzpicture}
\hskip 3em
\begin{tabular}{c|cc}
          &   $e_1$   &  $e_2$ \\ \hline
$v_1$   & -1     &     \\
$v_2$   &  1      &  -1    \\ 
$v_3$   &         &   1    
\end{tabular}
\end{center}

\textbf{Theorem}
Matrix $A \in \mathbb{Z}^{m \times n}$ is totally unimodular iff for every $R \subseteq \{1,\ldots,m\}$ there is a partition $R = R_1 \cup R_2$ such that for all $j \in 1\ldots n$.
\[
\sum_{i\in R_1}a_{ij} - \sum_{i \in R_2}a_{ij} \in \{-1,0,1\}
\]

Example: Incidence matrix $M \in \mathbb{R}^{|V|\times |E|}$ of (undirected) bipartite graph $G=(V,E)$ is totally unimodular.
$M_{eu} = M_{ev} = 1$ for every $e = uv \in E$.


\vfill

\emph{Next time: Integer Programming - Branch\&Bound}


\end{document}

\end{document}






