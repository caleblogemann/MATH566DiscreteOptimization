\documentclass[11pt, oneside]{article}
\usepackage[letterpaper, margin=2cm]{geometry}
\usepackage{MATH566}
%\usepackage{sagetex}

\begin{document}
\noindent \textbf{\Large{Caleb Logemann \\
MATH 566 Discrete Optimization\\
Homework 9
}}

%\lstinputlisting[language=Sage]{03_2.sage}
\begin{enumerate}
  \item % #1
    Implement Directed Minimum Mean Cycle in Sage. 
    Before implementing the algorithm, show that it is possible to slightly
    modify the algorithm.
    Instead of adding an extra vertex $s$ and edges from $s$ to all other
    vertices, it is possible to simply assign $F_0(v) = 0$ for all $v \in V$ at
    the beginning.
    This avoids the hassle with adding an extra vertex.
    But it requires an argument that the algorithm is still correct.

  \item % #2
    Show that in integer program, it is possible to express the following constraint:
    \[
      x \in [100,200]  \cup [300,400]
    \]
    in other words
    \[
      100 \leq x \leq 200   \text{ or } 300 \leq x \leq 400
    \]
    How to express the constraint \emph{without} using \emph{or}?\\
    % Hint: use additional integer variable z \in \{0,1\}, consider z and (1-z).

  \item % #3 Done
    Determine which of the matrices below are (i) unimodular, (ii) totally
    unimodular, or (iii) neither.
    Be sure to explain your answer.

    \begin{center}
      \begin{tabular}[h]{c@{\qquad}c@{\qquad}c}
        $\left(\begin{array}{rrrr}
        1 & -1 & -1 & 0\\
        -1 & 0 & 0 & 1\\
        0 & 1 & 0 & -1\\
        0 & 0 & 1 & 0\end{array}\right)$ & 
        $\left(\begin{array}{cccc}
        1 & 0 & 1 & 0\\
        0 & 1 & 0 & 0\\
        0 & 0 & 1 & 1\\
        1 & 1 & 0 & 1\end{array}\right)$ &
        $\left(\begin{array}{ccccc}
        0 & 1 & 0 & 0 & 0  \\
        1 & 0 & 1 & 0  & 0 \\
        0 & 1 & 0 & 1 & 1\\
        0 & 0 & 1 & 0 & 1\\
        0 & 0 & 1 & 1 & 0\end{array}\right) $
        \\
        \textbf{a.} &
        \textbf{b.}  &
        \textbf{c.} 
      \end{tabular}
    \end{center}

    \begin{enumerate}
      \item[(i)]
        Since these are all square matrices, they will be unimodular if their
        determinant is $\pm 1$.
        So I compute the determinants of these 3 matrices.
        First I will compute the determinant of $(a)$.
        \begin{align*}
          \det(a) &= \begin{vmatrix}
             1 & -1 & -1 &  0 \\
            -1 &  0 &  0 &  1 \\
             0 &  1 &  0 & -1 \\
             0 &  0 &  1 &  0
          \end{vmatrix}
          \intertext{Expand along first column}
          \det(a) &= 1 \times
          \begin{vmatrix}
            0 &  0 &  1 \\
            1 &  0 & -1 \\
            0 &  1 &  0
          \end{vmatrix}
          - (-1) \times
          \begin{vmatrix}
            -1 & -1 &  0 \\
             1 &  0 & -1 \\
             0 &  1 &  0
          \end{vmatrix}
          \intertext{Expand along the first column for the first determinant
            and along the last column for the second determinant}
          &=
          1 \times \p{-1 \times 
          \begin{vmatrix}
            0 &  1 \\
            1 &  0
          \end{vmatrix}}
          - (-1) \times \p{1 \times
          \begin{vmatrix}
            -1 & -1 \\
             0 &  1
          \end{vmatrix}} \\
          &= 
          1 \times \p{-1 \times -1} - (-1) \times \p{1 \times -1} \\
          &= 1 - 1 \\
          &= 0
        \end{align*}
        Since the determinant of $(a)$ is 0 this matrix is not unimodular.

        Second I will compute the determinant of $(b)$.
        \begin{align*}
          \det(b) &=
          \begin{vmatrix}
            1 & 0 & 1 & 0 \\
            0 & 1 & 0 & 0 \\
            0 & 0 & 1 & 1 \\
            1 & 1 & 0 & 1
          \end{vmatrix}
          \intertext{I will first expand along the first row.}
          \det(b) &= 1 \times
          \begin{vmatrix}
            1 & 0 & 0 \\
            0 & 1 & 1 \\
            1 & 0 & 1
          \end{vmatrix}
          + 1 \times
          \begin{vmatrix}
            0 & 1 & 0 \\
            0 & 0 & 1 \\
            1 & 1 & 1
          \end{vmatrix}
          \intertext{I will expand the first determinant along the first row
            and the second determinant along the first column}
          \det(b) &= 1 \times \p{1 \times
          \begin{vmatrix}
            1 & 1 \\
            0 & 1
          \end{vmatrix}}
          + 1 \times \p{1 \times
          \begin{vmatrix}
            1 & 0 \\
            0 & 1 \\
          \end{vmatrix}} \\
          &= 1 \times \p{1 \times 1} + 1 \times \p{1 \times 1} \\
          &= 1 + 1 \\
          &= 2
        \end{align*}
        Since the determinant of $(b)$ is 2 this matrix is not unimodular.

        Lastly I will compute the determinant of $(c)$.
        \begin{align*}
          \det(c) &=
          \begin{vmatrix}
            0 & 1 & 0 & 0 & 0 \\
            1 & 0 & 1 & 0 & 0 \\
            0 & 1 & 0 & 1 & 1 \\
            0 & 0 & 1 & 0 & 1 \\
            0 & 0 & 1 & 1 & 0
          \end{vmatrix}
          \intertext{First I will expand along the first column}
          \det(c) &= -1 \times
          \begin{vmatrix}
            1 & 0 & 0 & 0 \\
            1 & 0 & 1 & 1 \\
            0 & 1 & 0 & 1 \\
            0 & 1 & 1 & 0
          \end{vmatrix}
          \intertext{Next I will expand along the first row}
          \det(c) &= -1 \times \p{1 \times
          \begin{vmatrix}
            0 & 1 & 1 \\
            1 & 0 & 1 \\
            1 & 1 & 0
          \end{vmatrix}}
          \intertext{Expanding along the first column gives}
          \det(c) &= -1 \times \p{1 \times \p{-1 \times
          \begin{vmatrix}
            1 & 1 \\
            1 & 0
          \end{vmatrix}
          + 1 \times
          \begin{vmatrix}
            1 & 1 \\
            0 & 1 \\
          \end{vmatrix}}} \\
          &= -1 \times \p{1 \times \p{-1 \times -1 + 1 \times 1}} \\
          &= -1 \times \p{1 \times \p{1 + 1}} \\
          &= -1 \times \p{1 \times 2} \\
          &= -1 \times 2 \\
          &= -2
        \end{align*}
        Since the determinant of $(c)$ is -2 this matrix is not unimodular.

        In summary none of the matrices are unimodular.

      \item[(ii)]
        In order for a matrix to be totally unimodular every square submatrix
        must have determinant $-1$, $0$, or $1$.
        Note that since matrices $(b)$ and $(c)$ don't have a determinant
        $-1$, $0$, or $1$ when considered as a whole matrix they cannot be
        totally unimodular.
        Matrix $(a)$ which has determinant 0 can potentially be totally
        unimodular.
        In fact we see that each column has exactly one $1$ and one $-1$, so
        by a theorem in the notes $(a)$ is totally unimodular.

      \item[(iii)]
        We have shown that $(a)$ is totally unimodular but not unimodular.
        However $(b)$ and $(c)$ are neither unimodular nor totally unimodular.
    \end{enumerate}

  \item % #4
    Show that $A \in \mathbb{Z}^{m \times n}$ is totally unimodular iff $[A\ I]$
    is unimodular (where $I$ is $m \times m$ unit matrix).

  \item % #5 Done
    Find a unimodular matrix $A$, that is not totally unimodular.

    Consider the matrix
    \[
      A =
      \begin{matrix}
        9 & 7 \\
        5 & 4
      \end{matrix}
    \]
    The matrix $A$ is unimodular because
    $A \in \ZZ^{2 \times 2}$ and
    $\det(A) = 9 \times 4 - 5 \times 7 = 36 - 35 = 1$.
    However $A$ is not totally unimodular because not all of the
    entries of $A$ are $-1$, $0$, $1$.
\end{enumerate}
\end{document}
