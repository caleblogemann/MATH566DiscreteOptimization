\documentclass[11pt]{article}
\usepackage{amsmath, amssymb, amsthm}
\usepackage{fullpage}
\usepackage{tikz}
\usetikzlibrary{shadings}
\usepackage{amsmath}
\usepackage{hyperref}

\newcounter{excounter}
\setcounter{excounter}{1}
\newcommand\question[2]{\vskip 1em  \noindent\textbf{\arabic{excounter}\addtocounter{excounter}{1}:} (\emph{#1}) \newline \noindent#2}
\newcommand\hint[1]{ \newline \noindent\textit{Hint: #1}}
\newcommand\solution[1]{\vskip 0.5em \noindent\textbf{Solution:}\par\noindent#1}
\renewcommand\solution[1]{ }



\begin{document}
%\pagestyle{empty}
\large

{\bf MATH-566 \hspace{1cm} HW 3}
\vskip 1em
Due \textbf{Sep 21} before class. Just bring it before the class and it will be collected there.
%The solution has to be typed (using \LaTeX or print of the sage program).


\question{Farkas Lemma}{
Show that 
\begin{center}
 $A\mathbf{x} = \mathbf{b}$ has a non-negative solution iff $\forall \mathbf{y} \in \mathbb{R}^m$ with $\mathbf{y}^TA \geq \mathbf{0}^T$
implies $\mathbf{y}^T\mathbf{b} \geq 0$
\end{center}
implies
\begin{center}
$A\mathbf{x} \leq \mathbf{b}$ has a non-negative solution iff $\forall \mathbf{y} \in \mathbb{R}^m$, $\mathbf{y} \geq \mathbf{0}$ 
with $\mathbf{y}^TA \geq \mathbf{0}^T$  implies $\mathbf{y}^T\mathbf{b} \geq \mathbf{0}$.
\end{center}
}



\question{Fitting line as linear program}{
Some university in Iowa  was measuring the loudness of the fan's screaming during
the first touchdown of the local team. The measurements contain loudness in dB and the number
of people at the stadium in thousands. 
\begin{center}
\begin{tabular}{|r|c|c|c|c|c|c|}
\hline
\# fans &    53  &   55   &  59    &   61.5  & 61.5  \\  
\hline
dB       &   90   &   94   &  95    &  100  & 105  \\
\hline
\end{tabular}
\end{center}
Find a line $y = ax+b$ \emph{best} fitting the data. There are several different
notions of best fitting. Commonly used is least squares that is minimizing 
$\sum_{i}(ax_i+b - y_i)^2$. But big outliers move the result a lot (and it is troublesome
to do it using linear programming).
Use the one that minimizes the sum of differences. That is
\[
\sum_{i}|ax_i+b - y_i|.
\]
Write a linear program that solves the problem and solve it for the ``measured'' data.

\emph{(Fun facts: 
The Seattle Seahawks, who boast that their fans caused a small earthquake after a 2011 touchdown, acclaimed their crowd’s record 136.6-decibel noise level this September after an effort orchestrated by the fan group Volume 12.
The loudest crowd roar at a sports stadium is 142.2 db and was achieved by fans of the Kansas City Chiefs, at Arrowhead Stadium in Kansas City, Missouri, on 29 September 2014.)}
}



\question{Scheduling participants}{
Suppose you are preparing a schedule for classes.
You have fixed number of classes and students.
Every student told you which classes (s)he wants to attend.
However, you do not have enough time slots to run all classes sequentially so you need to
make some classes run in parallel.
Create a schedule and argue why it is the best schedule in the sense that
people \emph{as few conflicts as possible}.
You should be able  to justify the optimality of the schedule in some sense.

Make schedule with 3 and with 4 timeslots. Assume that there is no limit on
how many classes can run in one timeslot.

Row corresponds to one class, columns corresponds to one students. 1 means
the student wants to attend the class.
}

{
\tiny
\begin{verbatim}
[[0,0,0,1,0,1,0,1,0,0,1,0,1,0,0,1,1,1,1,1,1,1,1,0,1,1,1,1,0,0,1,0,0,0,1,1,1,1,1,1,0,1,1,1,1,1,1,0,0,1,0,1,0,0],
[1,1,1,1,0,1,0,1,0,0,0,1,1,1,0,0,0,0,0,0,1,0,1,0,0,0,0,0,1,1,0,1,0,0,1,0,0,0,1,0,1,0,1,0,0,0,0,1,0,0,1,1,1,0],
[1,0,1,0,1,0,1,0,1,1,1,1,0,0,1,1,1,1,1,1,0,0,0,1,1,1,0,1,0,0,0,1,0,0,1,1,0,1,1,1,0,1,1,1,1,1,0,0,0,1,1,0,0,0],
[0,0,1,0,0,0,1,0,0,1,1,0,0,0,0,1,0,0,0,0,0,0,0,0,1,1,0,0,0,1,0,0,1,0,0,0,1,0,0,0,0,0,0,0,1,1,1,0,1,1,0,0,0,0],
[1,0,0,1,0,1,0,0,0,1,0,0,1,0,0,0,0,0,0,0,0,1,0,1,0,0,1,0,1,0,1,1,0,1,0,0,0,0,0,0,0,0,0,0,0,0,0,0,1,0,0,0,1,0],
[0,1,1,1,1,0,0,0,1,1,1,1,1,0,1,0,1,1,0,1,0,0,0,0,0,1,0,0,0,1,1,0,0,1,1,0,1,1,1,1,0,0,0,1,1,1,1,0,0,1,0,1,0,1],
[0,0,1,1,0,1,1,0,0,1,1,1,0,1,1,0,1,1,0,1,0,1,1,0,0,1,1,0,0,0,1,0,0,0,0,0,0,1,0,0,0,0,1,1,0,1,1,0,0,1,0,1,1,0],
[1,0,0,0,0,1,1,0,1,0,0,0,0,0,0,0,0,0,1,0,1,0,0,0,0,0,1,1,0,0,0,1,0,0,0,1,0,0,0,0,0,0,1,0,1,0,1,0,1,0,0,0,0,0],
[0,1,0,0,0,1,1,0,1,0,1,0,0,0,0,1,1,1,0,1,0,1,1,1,1,1,0,0,0,0,1,0,1,0,0,0,1,0,1,1,0,0,1,1,1,0,1,0,0,0,0,1,0,1],
[1,1,1,1,0,1,0,1,0,0,0,1,1,1,0,0,0,0,0,0,1,0,1,0,0,0,0,0,1,1,0,1,0,0,1,0,0,0,1,0,1,0,1,0,0,0,0,1,0,0,1,1,1,0],
[1,0,1,0,1,0,1,0,1,1,1,1,0,0,1,1,1,1,1,1,0,0,0,0,1,1,0,1,0,0,0,1,0,0,1,1,0,1,1,1,0,1,1,1,1,1,0,0,0,1,1,0,0,0],
[0,0,1,0,0,0,1,0,0,1,1,0,0,0,0,1,0,0,0,0,0,0,0,0,1,0,0,0,0,0,0,0,0,0,0,0,1,0,0,0,0,0,0,0,1,1,1,0,1,1,0,0,0,0],
[1,0,0,1,0,1,0,0,0,1,0,0,1,0,0,0,0,0,0,0,0,1,0,0,0,0,1,0,1,0,1,1,1,0,0,0,0,0,0,0,0,0,0,0,0,0,0,0,1,0,0,0,1,0]]
\end{verbatim}
}
See the latex source on black board to copy the matrix easily.
I suggest to use Sage (or Matlab) to deal with the data.

PS: These data are real - they are not made up.


\end{document}










