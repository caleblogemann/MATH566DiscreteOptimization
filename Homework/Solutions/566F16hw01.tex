\documentclass[12pt]{article}
\usepackage{amsmath, amssymb, amsthm}
\usepackage{fullpage}
\usepackage{tikz}
\usepackage{amsmath}
\usepackage{hyperref}


\begin{document}

{ \bf Math-566 Homework \#1 (\emph{Practical} application of linear programming)}\\
\vskip 1em
\textit{
I will finish the homework before 9am Aug 31. I will type the solution (means not hand written).
}

\textbf{Story:}
Your lecturer wishes to save some money because he wants to buy a nice new digital camera.
He thinks that he may save on food. Suppose that he is your customer and he will pay you in credits
needed to pass MATH-566, if he will be happy with your result. Like any
other customer, he does not really know what he wants. He wants to save
money, eat enough of the basic nutrients. His weight is 165lb after dinner
and he is around 30 years old. 

\vskip 1em

\textbf{How to do it:} 
\begin{itemize}
\item Considered four nutrients out of (un)saturated fat, calories, fibers, iron, salt, proteins, and carbs.
\item Find out what is recommended daily intake of the nutrients for his body weight and age.
\item Go to your favorite shop(s) in Ames (Walmart, Target, Aldi, Fareway, \ldots, McDonalds, other fast food - these are the shops where the lecturer is willing to go shopping) and pick at least 15 different kinds of food. \emph{(Not $15\times$ cookies! Give some variety, at most 4 items from fast food chains.)}
\item Summarize what is the amount of the nutrients in each of the food.
\item Formulate a linear program $(LP)$ for finding a diet for time of one year such that:
\begin{itemize}
  \item the cost of food is minimized
  \item he gets at least recommended daily intake (salt times 1.4 recommended daily  intake)
\end{itemize}
\item Solve $(LP)$ using some solver - for example \url{http://apmonitor.com}, Mathematica, Sage or Excel.
\item Summarize the results. Write, what is the cost for the whole year as well as for just one day.
\item Solve the diet problem for yourself too! (change recommended daily intakes to your body and age, solve resulting (LP), summarize results)
\end{itemize}
More notes - provide references (http links) to anything you are using. 
Include also the formulation of $(LP)$ which you were using as an input for your solver.
It is assumed that the homework will be typed. If the output of the solver is that the program
has no solution, then you did not formulate the program correctly or you require some constraint
like say salt is at least 5mg but no food has salt. \\
Bon App\'etit!

\end{document}
