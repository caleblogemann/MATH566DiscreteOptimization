\documentclass[12pt]{article}
\usepackage{amsmath, amssymb, amsthm}
\usepackage{fullpage}
\usepackage{tikz}
\usepackage{amsmath}
\usepackage{hyperref}

\newcounter{excounter}
\setcounter{excounter}{1}
\newcommand\question[2]{\vskip 1em  \noindent\textbf{\arabic{excounter}\addtocounter{excounter}{1}:} (\emph{#1}) \newline \noindent#2}
\newcommand\hint[1]{ \newline \noindent\textit{Hint: #1}}
\newcommand\solution[1]{\vskip 0.5em \noindent\textbf{Solution:}\par\noindent#1}
\renewcommand\solution[1]{ }



\begin{document}
\pagestyle{empty}
\large

{\bf MATH-566 \hspace{1cm} HW 2}
\vskip 1em
Due \textbf{Sep 14} before class. Just bring it before the class and it will be collected there.
%The solution has to be typed (using \LaTeX or print of the sage program).

\question{Separation of sets}{
(a) Give an example of bounded convex sets $C$ and $D$ in $\mathbb{R}^d$, where $C \cap D = \emptyset$
but there are no hyperplane strictly separating $C$ and $D$. That is
$\mathbf{a} \in \mathbb{R}^d$ and $b \in \mathbb{R}$ such that
\begin{align*}
\mathbf{a}^T\mathbf{x} &> b, \forall \mathbf{x}\in C
&
&\text{and}
&
\mathbf{a}^T\mathbf{x} &< b, \forall \mathbf{x}\in D.
\end{align*}

(b) Give an example of closed convex sets $C$ and $D$ that cannot be strictly separated.
}



\question{Dualization}{
Dualize your diet problem. Take data of your diet problem from HW1 and create dual of the program and solve it.
Answer should contain solution of the dual and interpretation of the result of the dual (what do the numbers mean).
}


\question{Integer program}{
Paper mill manufactures rolls of paper of a standard width 3 meters. But customers want to buy paper rolls of shorter width, and the mill has to cut such rolls from the 3 m rolls. One 3 m roll can be cut, for instance, into two rolls 93 cm wide, one roll of width 108 cm, and a rest of 6 cm (which goes to waste).
Let us consider an order of
 \begin{itemize}
\item 97 rolls of width 135 cm,
\item 610 rolls of width 108 cm,
\item 395 rolls of width 93 cm, and
 \item 211 rolls of width 42 cm.
 \end{itemize}
What is the smallest number of 3 m rolls that have to be cut in order to satisfy this order, and how should they be cut?\\
 \emph{(Hints: 1m = 100cm, you may need to consider that variables are integers instead of just real numbers. APMonitor can
 do it by prefix int in variable and Sage by using \emph{new\_variable(integer=True))}.}
}




\end{document}










