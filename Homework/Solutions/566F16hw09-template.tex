\documentclass[11pt]{article}
\usepackage{amsmath, amssymb, amsthm}
\usepackage{fullpage}
\usepackage{tikz}
\usetikzlibrary{shapes.geometric}
\usetikzlibrary{arrows}
\usepackage{amsmath}
\usepackage{hyperref}

\newcounter{excounter}
\setcounter{excounter}{1}
\newcommand\question[2]{\vskip 1em  \noindent\textbf{\arabic{excounter}\addtocounter{excounter}{1}:} (\emph{#1}) \newline \noindent#2}
\newcommand\hint[1]{ \newline \noindent\textit{Hint: #1}}
\newcommand\solution[1]{\vskip 0.5em \noindent\textbf{Solution:}\par\noindent#1}

%\renewcommand\solution[1]{ }



\begin{document}
\large

{\bf MATH-566 \hspace{1cm} HW 09}
\vskip 1em
Due \textbf{Nov 16} before class (regularly). Just bring it before the class and it will be collected there.



\question{Directed Minimum Mean Cycle}{
Implement Directed Minimum Mean Cycle in Sage. 
Before implementing the algorithm, show that it is possible to slightly modify the algorithm.
Instead of adding an extra vertex $s$ and edges from $s$ to all other vertices, it is possible
to simply assign $F_0(v) = 0$ for all $v \in V$ at the beginning. This avoids the hassle with adding an extra
vertex. But it requires an argument that the algorithm is still correct.\\
Feel free to use any part of the Sage template. If you don't like the outline I made, don't use it.
}



\question{Strength of integer programming}{
Show that in integer program, it is possible to express the following constraint:
\[
x \in [100,200]  \cup [300,400]\]
in other words
\[
100 \leq x \leq 200   \text{ or } 300 \leq x \leq 400
\]
How to express the constraint \emph{without} using \emph{or}?\\ \emph{Hint: use additional integer variable $z \in \{0,1\}$, consider $z$ and $(1-z)$.}
}





\question{What is unimodular?}{
Determine which of the matrices below are (i) unimodular, (ii) totally unimodular, or (iii) neither.
Be sure to explain your answer.

\begin{center}
\begin{tabular}[h]{c@{\qquad}c@{\qquad}c}
$\left(\begin{array}{rrrr}
1 & -1 & -1 & 0\\
-1 & 0 & 0 & 1\\
0 & 1 & 0 & -1\\
0 & 0 & 1 & 0\end{array}\right)$ & 
$\left(\begin{array}{cccc}
1 & 0 & 1 & 0\\
0 & 1 & 0 & 0\\
0 & 0 & 1 & 1\\
1 & 1 & 0 & 1\end{array}\right)$ &
$\left(\begin{array}{ccccc}
0 & 1 & 0 & 0 & 0  \\
1 & 0 & 1 & 0  & 0 \\
0 & 1 & 0 & 1 & 1\\
0 & 0 & 1 & 0 & 1\\
0 & 0 & 1 & 1 & 0\end{array}\right) $
\\
\textbf{a.} &
\textbf{b.}  &
\textbf{c.} 
\end{tabular}
\end{center}
}



\question{Unimodular and totally unimodular}{
Show that $A \in \mathbb{Z}^{m \times n}$ is totally unimodular iff $[A\ I]$ is unimodular (where $I$ is $m \times m$ unit matrix).
}


\question{Not all unimodular is totally unimodular}{
Find a unimodular matrix $A$, that is not totally unimodular.
}


\end{document}
