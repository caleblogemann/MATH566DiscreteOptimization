\documentclass[11pt]{article}
\usepackage{amsmath, amssymb, amsthm}
\usepackage{fullpage}
\usepackage{tikz}
\usetikzlibrary{shapes.geometric}
\usetikzlibrary{arrows}
\usepackage{amsmath}
\usepackage{hyperref}

\newcounter{excounter}
\setcounter{excounter}{1}
\newcommand\question[2]{\vskip 1em  \noindent\textbf{\arabic{excounter}\addtocounter{excounter}{1}:} (\emph{#1}) \newline \noindent#2}
\newcommand\hint[1]{ \newline \noindent\textit{Hint: #1}}
\newcommand\solution[1]{\vskip 0.5em \noindent\textbf{Solution:}\par\noindent#1}

\renewcommand\solution[1]{ }



\begin{document}
\pagestyle{empty}
\large

{\bf MATH-566 \hspace{1cm} Midterm 1}
\vskip 1em
Due \textbf{Oct 7} before class. Just bring it before the class and it will be collected there.
If you need more time (some unexpected even may happen), let me know BEFORE the deadline.

\question{LP application}{
Suppose you are making a schedule for an airport.
There are $n$ arriving flights. Every airplane $j$ has a possible time arrival  in interval $[a_j,b_j]$
(plane can fly faster or slower). 
Determine the actual arrival schedule for each airplane such that the smallest gap between consecutive flights is maximized
and for all  $j$, airplane $j$ arrives before $j+1$.
Formulate a linear program that solves the problem.

\emph{
(Example: Suppose there are three airplanes. They have arrival intervals $[1,5],[2,7],[6,7]$.
Then we can assign arrival times to the airplanes, for example $2,4.5,6.2$. The smallest
gap in this schedule is $1.7$ between the second and third airplane. The number $1.7$
is the number we want to maximize. Notice that we do not allow schedule $4,2,7$, where the first
airplane arrives AFTER the second one although the it would be feasible with respect to $[a_i,b_i]$s
(it is easier to solve if the order is fixed).)
}
}






\question{Simplex Method}{
Solve the following linear program $(P)$ using simplex method.
\[
(P) = 
\begin{cases}
\text{maximize}  &x_1+x_2 \\
\text{subject to}  & x_1 \leq 1 \\
                          & -x_1 + x_2 \leq 1 \\
                          & x_1,x_2 \geq 0
\end{cases}
\]
Check your solution using computer program (APMonitor, Sage,\ldots).
Plot the set of feasible solutions and mark the optimum.
Solving using simplex method means make the sequence of simplex tables.
}



\question{Special Minimum Spanning Tree Algorithm}{
Consider the following algorithm. Input is a connected graph $G=(V,E)$ and a cost function $c:E \rightarrow \mathbb{R}$.
Start with $H$ being a copy of $G$.
First, the edges $E$ are ordered such that $c(e_1) \geq c(e_2) \geq \ldots \geq c(e_m)$.
Then process edges one by one according to the ordering.
Processing edge $e_i$ means looking if $H-e_i$ connected. If $H-e_i$ is connected, then $e_i$ is
removed from $H$. Otherwise $e_i$ is kept in $H$. After all edges are processed, the resulting $H$ is the output.
Now you can pick what to do. Either a) or b):\\
a) implement the algorithm and use as inputs the same graph we used for the minimum spanning tree\\
b) prove that the algorithm produces minimum spanning tree.
}

\question{Alternative MST problem}{
Consider the following problem. 
Input is a connected graph $G=(V,E)$ and a cost function $c:E \rightarrow \mathbb{R}$.
Let $T$ be a spanning tree of $G$. The cost of $T$ is defined as the largest cost of an edge
in $T$: 
\[
c(T) = \max\{c(e): e \in E(T)   \}.
\]
Problem is to find a minimum spanning tree with respect to $c$\\
Do both a) and b):\\
a) formulate the problem using integer programming\\
b) Find an algorithm for solving this problem in polynomial time and prove its correctness. 
}

\end{document}











