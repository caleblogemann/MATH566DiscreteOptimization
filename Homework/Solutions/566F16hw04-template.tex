\documentclass[11pt]{article}
\usepackage{amsmath, amssymb, amsthm}
\usepackage{fullpage}
\usepackage{tikz}
\usetikzlibrary{shadings}
\usetikzlibrary{shapes.geometric}
\usetikzlibrary{arrows}
\usepackage{amsmath}
\usepackage{hyperref}

\newcounter{excounter}
\setcounter{excounter}{1}
\newcommand\question[2]{\vskip 1em  \noindent\textbf{\arabic{excounter}\addtocounter{excounter}{1}:} (\emph{#1}) \newline \noindent#2}
\newcommand\hint[1]{ \newline \noindent\textit{Hint: #1}}
\newcommand\solution[1]{\vskip 0.5em \noindent\textbf{Solution:}\par\noindent#1}
%\renewcommand\solution[1]{ }



\begin{document}
%\pagestyle{empty}
\large

{\bf MATH-566 \hspace{1cm} HW 4}
\vskip 1em
Due \textbf{Sep 28} before class. Just bring it before the class and it will be collected there.
%The solution has to be typed (using \LaTeX or print of the sage program).


\question{Basic Feasible Solution Solver}{
Write a program that will solve an instance of linear programming
by enumerating basic feasible solutions and keeping the best one.
Input to the program is matrix $A \in \mathbb{R}^{m\times n}$ and vectors $\mathbf{b} \in \mathbb{R}^{m}$ and $\mathbf{c}\in\mathbb{R}^{m}$. Solve
\[
(P) \begin{cases} 
\begin{array}{lrccc} 
 \text{maximize}  &\mathbf{c}^T\mathbf{x}  & &  \\
 \text{subject to}  & A\mathbf{x} & = & \mathbf{b} \\
                           & \mathbf{x}   & \geq & 0 \\
\end{array}\\
 \end{cases}
\]
A template for sage follows. I suggest you copy it from the LaTeX source file....
}
\begin{verbatim}
# https://wiki.sagemath.org/quickref?action=AttachFile&do=get&target=quickref-linalg.pdf
# I also suggest to google itertools.combinations
#
# This HW is about writing a program that solves linear programming problems where
# the input is A,b,c and you want to find x >= 0 that minimizes c^Tx if Ax = b.
#
# You should write a solver that enumerates all basic feasible solutions and picks
# the best one. Recall that you get a basic feasible solution by picking m columns
# of A that are linearly independent and solve for x.
#
# The same is also implemented using sage solver so you can check if your program
# produced the correct result.
#
import itertools

# This is where you will write your program
def solve_linear_program(A,b,c):
    m = len(b)
    n = len(c)
    if A.nrows() != m or A.ncols() != n or m > n:
        print "Invalid input"
        return [None,None]

    min_value = None
    min_x     = None
  
    # Here comes your code where you compute min_value and min_x
      
    return [min_value,min_x]

# This is solver using sage 
def solve_using_sage(A,b,c):
    m = len(b)
    n = len(c)
    p = MixedIntegerLinearProgram(maximization=False);
    x = p.new_variable(nonnegative=True);
    for i in range(m):
        p.add_constraint( b[i] == sum([A[i,j]*x[j] for j in range(n)]) )
    p.set_objective(sum( [ c[j]*x[j] for j in range(n)] ))
    return p.solve()
        
# This is running your and also sage solver and prints the results
def test_solver(A,b,c):
    min_value, min_x = solve_linear_program(A,b,c)
    min_value_sage = solve_using_sage(A,b,c)
    print "Optimal solution has value", min_value_sage," You computed",min_value," for x=",min_x
    


######### Several test data

# Input
A = matrix(QQ,[[1,0,1,0,0],[1,1,0,1,0],[-1,1,0,0,1]])
b = vector(QQ,[4,5,1])
c = vector(QQ,[-1,-2,0,0,0])

test_solver(A,b,c)


# Input
A = matrix(QQ,[[1,-2,1,0,0],[1,1,0,1,0],[-1,1,0,0,1]])
b = vector(QQ,[4,2,4])
c = vector(QQ,[-1,2,0,0,0])

test_solver(A,b,c)


# Input
A = matrix(QQ,[[1,-2,6,6,1,1,0,0],[1,1,2,2,-2,0,1,0],[-1,1,3,3,6,0,0,1]])
b = vector(QQ,[4,2,5])
c = vector(QQ,[-1,2,-2,-2,1,0,0,0])

test_solver(A,b,c)
\end{verbatim}


\question{Using simplex method}
{
Convert the following program to equational form (add $x_3,x_4,x_5$) and solve it using the simplex method.
\[
(P) \begin{cases} 
\begin{array}{lrccccc} 
 \text{maximize}  &x_1  &+ & 2x_2    &        &   \\
 \text{subject to}  & x_1 &   &            &\leq  & 4 \\
                           & x_1 &+ &  x_2         & \leq & 5 \\
                           & -x_1 &+ &  x_2         & \leq & 1 \\
\end{array}\\
\text{ \hskip 8em }
 x_1,x_2 \geq 0
 \end{cases}
\]
Use Bland's rule for selecting pivots. That is, pivot on variable with lowest possible index.

Use the last tableau to argue that the solution is indeed optimal.

Plot the set of feasible solutions of $(P)$ and mark  the solutions obtained after
each iteration of the simplex method.
}


\question{Simplex method test}{
Use simplex method on the following program:
\[
(P) \begin{cases} 
\begin{array}{lrccccc} 
 \text{maximize}  &x_1  & &    &        &   \\
 \text{subject to}  & x_1 &-&  x_2   &\leq  &1 \\
                           & -x_1 &+ &  x_2         & \leq & 2 \\
\end{array}\\
\text{ \hskip 8em }
 x_1,x_2 \geq 0
 \end{cases}
\]
What is happening in the computation?
}




\question{Ellipsoid method for solving linear programs}{
How would you solve a program $(P)  = \text{maximize } \mathbf{c}^T\mathbf{x} \text{ s.t. } A\mathbf{x} \leq \mathbf{b}, \mathbf{x} \geq \mathbf{0}$
using the ellipsoid method with an $\varepsilon > 0$ error?\\
(Suppose both $(P)$ and its dual $(D)$ are superconsistent.)\\
\emph{(Hint: How to formulate the linear program as finding a point in a polytope? 
Use dual program and $\varepsilon$ to guarantee full dimension.)}
}



\end{document}










