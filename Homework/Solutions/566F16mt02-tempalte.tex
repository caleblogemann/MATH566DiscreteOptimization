\documentclass[11pt]{article}
\usepackage{amsmath, amssymb, amsthm}
\usepackage{fullpage}
\usepackage{tikz}
\usetikzlibrary{shapes.geometric}
\usetikzlibrary{arrows}
\usepackage{amsmath}
\usepackage{hyperref}

\newcounter{excounter}
\setcounter{excounter}{1}
\newcommand\question[2]{\vskip 1em  \noindent\textbf{\arabic{excounter}\addtocounter{excounter}{1}:} (\emph{#1}) \newline \noindent#2}
\newcommand\hint[1]{ \newline \noindent\textit{Hint: #1}}
\newcommand\solution[1]{\vskip 0.5em \noindent\textbf{Solution:}\par\noindent#1}

\renewcommand\solution[1]{ }



\begin{document}
\pagestyle{empty}
\large

{\bf MATH-566 \hspace{1cm} Midterm 2}
\vskip 1em
Due \textbf{Oct 18} before class. Just bring it before the class and it will be collected there.
If you need more time (some unexpected even may happen), let me know BEFORE the deadline.



\question{Can you use IP?}{
\emph{Girth} of a graph $G=(V,E)$ is the length of the shortest cycle. 
Notice cycle in undirected graph has at least 3 vertices.
Formulate an  integer program that is solving the problem of shortest cycle
in a graph.\\
Hint: Put variables on vertices rather than edges.
}

\question{Do you understand augmentations in minimum cost flow?}{
Recall that in Minimum Cost Flow, we were augmenting on minimum mean cycle.
Show that if we allow augmentation on any augmenting cycle, the algorithm
will not work right (bad convergence). \\
\emph{Hint: Show a construction and sequence
of bad choices for augmenting cycle. Recall Ford-Fulkerson algorithm problem.
}
}


\question{Graph Algorithms}{
Suppose you have a graph $G=(V,E)$. Every edge is assigned a cost $c: E \rightarrow \mathbb{R}$.
How to pick edges, such that the sum of costs of the picked edges is maximized
and the picked edges do not contain a cycle (i.e., the set of picked edges forms a forest)?
More formally, 
\[
\text{maximize} \left\{  \sum_{e \in X} c(e):  X \subseteq E, X \text{ has no cycle}  \right\}.
\]
Describe how to use one/some of the algorithms we had in class to solve this problem.\\
(I mean, how to modify the input to this problem such that it can be solved by some other algorithm?
Alternatively, you can also describe an algorithm.)

\emph{Example: Notice that the edges can have negative weight. The thick edges in the following graph show one of the optimal solutions.
(There are two optimal solutions. Your algorithm should find one.)
\begin{center}
\tikzset{insep/.style={inner sep=2pt, outer sep=0pt, circle, fill},}
\begin{tikzpicture}
\draw (0,0)node[insep](a){}  -- node[left]{$2$}  (0,1) node[insep](b){}
-- node[above]{$-1$}  (1,1) node[insep](c){} 
-- node[left]{$1$}  (1,0) node[insep](d){} 
-- node[below]{$-1$}  (a)
(c) -- node[above]{$1$}  (3,0.5) node[insep](e){} 
-- node[below]{$2$}  (d)
;
\draw[line width=2pt]
(a)--(b) (c)--(d)--(e);
;
\end{tikzpicture}
\end{center}
}
}

\end{document}










