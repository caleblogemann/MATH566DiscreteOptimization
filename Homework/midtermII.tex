\documentclass[11pt, oneside]{article}
\usepackage[letterpaper, margin=2cm]{geometry}
\usepackage{MATH566}
%\usepackage{sagetex}

\begin{document}
\noindent \textbf{\Large{Caleb Logemann \\
MATH 566 Discrete Optimization\\
Midterm II
}}

%\lstinputlisting[language=Sage]{03_2.sage}
\begin{enumerate}
  \item % #1 Done
    \emph{Girth} of a graph $G=(V,E)$ is the length of the shortest cycle. 
    Notice cycle in undirected graph has at least 3 vertices.
    Formulate an  integer program that is solving the problem of shortest cycle
    in a graph.
    % Hint: Put variables on vertices rather than edges.

    In order to create this integer program, I will create a binary variable for
    each vertex in the graph $G$.
    Let $x_v \in \br{0, 1}$, denote whether or not vertex $v$ is in the
    shortest cycle.
    Note that if $v$ is in the shortest cycle then at least two of $v$'s
    neighbors must also be in the cycle.
    If only one neighbor was in the cycle, then there would be a path into
    $v$ but not out of $v$, so it could not be a cycle.
    If there weren't any neighbors in the cycle, then $v$ would be disconnected
    from the cycle.
    This reasoning can be expressed in the following constraint.
    \[
      \sum{u \in N(v)}{}{x_u} \ge 2
    \]
    where $N(v)$ is the set of vectors that are neighbors of $v$.
    This is also assuming that $v$ is in the cycle so $x_v = 1$.
    If $v$ is not in the cycle then $x_v = 0$ and there may or may not
    be neighbors of $v$ in the cycle.
    This does not require a constraint but a trivial constraint can be used
    to describe this situation.
    \[
      \sum{u \in N(v)}{}{x_u} \ge 0
    \]
    This is clearly true because all $x_u$ are binary variables.
    Now it is necessary to be able to describe both of these situations
    depending on the value of $x_v$.
    Note that if $x_v = 0$, then $2x_v = 0$ and if $x_v = 1$, then $2x_v = 2$.
    Using this fact both constraints can be expressed as
    \[
      \sum{u \in N(v)}{}{x_u} \ge 2x_v
    \]
    This constraint forces there to be two neighbors in the cycle if $v$ is in
    the cycle, but doesn't enforce anything if $v$ is not in the cycle.

    Now in order to find the girth of a graph, we wish to detect the smallest
    cycle.
    Therefore the integer program should minimize the number of vertices in the
    cycle or equivalently the objective function should be
    \[
      \text{minimize} \sum*{v \in V}{}{x_v}
    \]

    Just using the constraints given above and this objective function the
    integer program will simply set all $x_v$ equal to zero.
    In order to force the program to find a nontrivial cycle we must force that
    at least 3 vertices are in the cycle.
    No cycle and can created with 0, 1, or 2 vertices.
    This constraint can be written as
    \[
      \sum*{v \in V}{}{x_v} \ge 3
    \]

    Therefore the full integer program is
    \[
      (P) =
      \begin{cases}
        \text{minimize}    & \sum*{v \in V}{}{x_v} \\
        \text{subject to}  & \sum{v \in V}{}{x_v} \ge 3 \\
                           & \sum{u \in N(v)}{}{x_u} \ge 2x_v \quad \forall v \in V \\
                           & x_v \ge 0 \quad \forall v \in V \\
                           & x_v \le 1 \quad \forall v \in V \\
                           & x_v \in \ZZ \quad \forall v \in V \\
      \end{cases}
    \]

  \item % #2
    Recall that in Minimum Cost Flow, we were augmenting on minimum mean cycle.
    Show that if we allow augmentation on any augmenting cycle, the algorithm
    will not work right (bad convergence).
    % Hint: Show a construction and sequence of bad choices for augmenting cycle.
    % Recall Ford-Fulkerson algorithm problem.

  \item % #3
    Suppose you have a graph $G=(V,E)$.
    Every edge is assigned a cost $c: E \rightarrow \RR$.
    How to pick edges, such that the sum of costs of the picked edges is maximized
    and the picked edges do not contain a cycle (i.e., the set of picked
    edges forms a forest)?
    More formally, 
    \[
      \text{maximize} \set{\sum*{e \in X}{}{c(e)}:  X \subseteq E, X \text{ has no cycle}}.
    \]
    Describe how to use one/some of the algorithms we had in class to solve this problem.\\
    (I mean, how to modify the input to this problem such that it can be solved by some other algorithm?
    Alternatively, you can also describe an algorithm.)

    \emph{
      Example: Notice that the edges can have negative weight.
      The thick edges in the following graph show one of the optimal solutions.
      (There are two optimal solutions. Your algorithm should find one.)
      \begin{center}
        \tikzset{insep/.style={inner sep=2pt, outer sep=0pt, circle, fill},}
        \begin{tikzpicture}
          \draw (0,0)node[insep](a){}  
          -- node[left]{$2$}  (0,1) node[insep](b){}
          -- node[above]{$-1$}  (1,1) node[insep](c){} 
          -- node[left]{$1$}  (1,0) node[insep](d){} 
          -- node[below]{$-1$}  (a)
          (c) -- node[above]{$1$}  (3,0.5) node[insep](e){} 
          -- node[below]{$2$}  (d)
          ;
          \draw[line width=2pt]
          (a)--(b) (c)--(d)--(e);
          ;
        \end{tikzpicture}
      \end{center}
    }
\end{enumerate}
\end{document}
