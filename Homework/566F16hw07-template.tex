\documentclass[11pt]{article}
\usepackage{amsmath, amssymb, amsthm}
\usepackage{fullpage}
\usepackage{tikz}
\usetikzlibrary{shapes.geometric}
\usetikzlibrary{arrows}
\usepackage{amsmath}
\usepackage{hyperref}
\usepackage{verbatim}

\newcounter{excounter}
\setcounter{excounter}{1}
\newcommand\question[2]{\vskip 1em  \noindent\textbf{\arabic{excounter}\addtocounter{excounter}{1}:} (\emph{#1}) \newline \noindent#2}
\newcommand\hint[1]{ \newline \noindent\textit{Hint: #1}}
\newcommand\solution[1]{\vskip 0.5em \noindent\textbf{Solution:}\par\noindent#1}

\renewcommand\solution[1]{ }



\begin{document}
\large

{\bf MATH-566 \hspace{1cm} HW 7}
\vskip 1em
Due \textbf{Nov 2} before class. Just bring it before the class and it will be collected there.


\question{Nice cuts}{
A cut of $G$ is \emph{minimal} if there is no cut of $G$ properly contained in it. Prove that the random contraction algorithm returns only minimal cuts.
}

\question{Deterministic Minimum Cuts}{
Implement Node Identification Minimum Cut Algorithm.
Try it on the following graph:
\begin{center}
\tikzset{insep/.style={inner sep=2pt, outer sep=0pt, circle, fill},}
\begin{tikzpicture}[scale=1.3]
\draw
(0,0) node[insep,label=below:$g$](g){}
(2,0) node[insep,label=below:$f$](f){}
(2,1) node[insep,label=right:$e$](e){}
(2,2) node[insep,label=right:$a$](a){}
(1,3) node[insep,label=above:$b$](b){}
(0,1) node[insep,label=below left:$d$](d){}
(0,2) node[insep,label=above:$c$](c){}
(-1,1) node[insep,label=left:$h$](h){}
;
\draw
\foreach \x/\y/\t in {g/f/6, d/e/3, c/a/2, c/b/3, b/a/5,h/c/2, h/d/1}{
(\x)--node[pos=0.5,label=above:$\t$]{} (\y)
}
(h)--node[pos=0.5,label=below:$5$]{} (g)
\foreach \x/\y/\t in {g/d/2, d/c/4, f/e/2, e/a/3}{
(\x)--node[pos=0.5,label=right:$\t$]{} (\y)
}
;
\end{tikzpicture}
\end{center}
}

\question{Random Contraction Algorithm}{
Implement Random Contraction Algorithm.
Try it on the following graph. 
Run it many times and based on your experiment conclude what is the probability that that your algorithm succeeds on this graph.
\begin{center}
\tikzset{insep/.style={inner sep=2pt, outer sep=0pt, circle, fill},}
\begin{tikzpicture}[scale=1.3]
\draw
(0,0) node[insep,label=below:$g$](g){}
(2,0) node[insep,label=below:$f$](f){}
(2,1) node[insep,label=right:$e$](e){}
(2,2) node[insep,label=right:$a$](a){}
(1,3) node[insep,label=above:$b$](b){}
(0,1) node[insep,label=below left:$d$](d){}
(0,2) node[insep,label=above:$c$](c){}
(-1,1) node[insep,label=left:$h$](h){}
;
\draw
\foreach \x/\y/\t in {g/f/6, d/e/3, c/a/2, c/b/3, b/a/5,h/c/2, h/d/1}{
(\x)--node[pos=0.5,label=above:$\t$]{} (\y)
}
(h)--node[pos=0.5,label=below:$5$]{} (g)
\foreach \x/\y/\t in {g/d/2, d/c/4, f/e/2, e/a/3}{
(\x)--node[pos=0.5,label=right:$\t$]{} (\y)
}
;
\end{tikzpicture}
\end{center}
}

\question{Gomory-Hu Tree}{
Construct Gomory-Hu Tree for the following graph using the algorithm
from the class. 
Numbers on edges correspond to capacities.
Show Show steps after every new cut.
\begin{center}
\tikzset{insep/.style={inner sep=2pt, outer sep=0pt, circle, fill},}
\begin{tikzpicture}
\draw
\foreach \x/\a in {0/a,60/b,120/c,180/d,240/e,300/f}{
(\x:2.5) node[insep](\a){}
}
(a)--node[right]{1} (b) -- node[above]{3} (c) -- node[left]{2}(d) -- node[below]{1}(e) -- node[below]{3}(f) -- node[right]{1}(a)
(b) -- node[below]{2} (d) -- node[below]{1}(f) -- node[right]{2}(b)
;
\end{tikzpicture}
\end{center}
}

Check your answer with Sage using method \verb|Graph.gomory_hu_tree()|. Do not implement it yourself, just run it.

\end{document}













