\documentclass[11pt, oneside]{article}
\usepackage[letterpaper, margin=2cm]{geometry}
\usepackage{MATH566}
%\usepackage{sagetex}

\begin{document}
\noindent \textbf{\Large{Caleb Logemann \\
MATH 566 Discrete Optimization\\
Homework 3
}}

%\lstinputlisting[language=Matlab]{H01_23.m}
\begin{enumerate}
    \item % #1
        Show that
        \begin{center}
            $A\v{x} = \v{b}$ has a nonnegative solution iff
            $\forall \v{y} \in \RR^m$ with $\v{y}^T A \ge \v{0}^T$ implies
            $\v{y}^T \v{b} \ge 0$
        \end{center}
        implies
        \begin{center}
            $A\v{x} \le \v{b}$ has a nonnegative solution iff
            $\forall \v{y} \in \RR^m$, $\v{y} \ge \v{0}$ with $\v{y}^T A \ge \v{0}^T$
            implies $\v{y}^T \v{b} \ge 0$.
        \end{center}

    \item % #2
        Some university in Iowa was measuring the loudness of the fan’s
        screaming during the first touchdown of the local team.
        The measurements contain loudness in dB and the number of people at the
        stadium in thousands
        \begin{center}
            \begin{tabular}{*{6}c}
                \toprule
                \# fans & 53 & 55 & 59 & 61.5 & 61.5 \\
                \midrule
                dB     & 90 & 94 & 95 &  100 &  105 \\
                \bottomrule
            \end{tabular}
        \end{center}
        Find a line $y = ax + b$ best fitting the data.
        There are several different notions of best fitting.
        Commonly used is least squares that is minimizing $\sum*{i}{}{\p{ax_i + b - y_i}^2}$.
        But big outliers move the result a lot (and it is troublesome to do it
        using linear programming).
        Use the one that minimizes the sum of differences. That is
        \[
            \sum{i}{}{\abs{a x_i + b - y_i}}
        \]
        Write a linear program that solves the problem and solve if for the
        ``measured'' data.

    \item % #3
        Suppose you are preparing a schedule for classes.
        You have fixed number of classes and students.
        Every student told you which classes (s)he wants to attend.
        However, you do not have enough time slots to run all classes
        sequentially so you need to make some classes run in parallel.
        Create a schedule and argue why it is the best schedule in the sense
        that people as few conflicts as possible.
        You should be able to justify the optimality of the schedule in some sense.
        Make schedule with 3 and with 4 timeslots.
        Assume that there is no limit on how many classes can run in one timeslot.

        Use the following data
        {
        \tiny
            \begin{verbatim}
                [[0,0,0,1,0,1,0,1,0,0,1,0,1,0,0,1,1,1,1,1,1,1,1,0,1,1,1,1,0,0,1,0,0,0,1,1,1,1,1,1,0,1,1,1,1,1,1,0,0,1,0,1,0,0],
                [1,1,1,1,0,1,0,1,0,0,0,1,1,1,0,0,0,0,0,0,1,0,1,0,0,0,0,0,1,1,0,1,0,0,1,0,0,0,1,0,1,0,1,0,0,0,0,1,0,0,1,1,1,0],
                [1,0,1,0,1,0,1,0,1,1,1,1,0,0,1,1,1,1,1,1,0,0,0,1,1,1,0,1,0,0,0,1,0,0,1,1,0,1,1,1,0,1,1,1,1,1,0,0,0,1,1,0,0,0],
                [0,0,1,0,0,0,1,0,0,1,1,0,0,0,0,1,0,0,0,0,0,0,0,0,1,1,0,0,0,1,0,0,1,0,0,0,1,0,0,0,0,0,0,0,1,1,1,0,1,1,0,0,0,0],
                [1,0,0,1,0,1,0,0,0,1,0,0,1,0,0,0,0,0,0,0,0,1,0,1,0,0,1,0,1,0,1,1,0,1,0,0,0,0,0,0,0,0,0,0,0,0,0,0,1,0,0,0,1,0],
                [0,1,1,1,1,0,0,0,1,1,1,1,1,0,1,0,1,1,0,1,0,0,0,0,0,1,0,0,0,1,1,0,0,1,1,0,1,1,1,1,0,0,0,1,1,1,1,0,0,1,0,1,0,1],
                [0,0,1,1,0,1,1,0,0,1,1,1,0,1,1,0,1,1,0,1,0,1,1,0,0,1,1,0,0,0,1,0,0,0,0,0,0,1,0,0,0,0,1,1,0,1,1,0,0,1,0,1,1,0],
                [1,0,0,0,0,1,1,0,1,0,0,0,0,0,0,0,0,0,1,0,1,0,0,0,0,0,1,1,0,0,0,1,0,0,0,1,0,0,0,0,0,0,1,0,1,0,1,0,1,0,0,0,0,0],
                [0,1,0,0,0,1,1,0,1,0,1,0,0,0,0,1,1,1,0,1,0,1,1,1,1,1,0,0,0,0,1,0,1,0,0,0,1,0,1,1,0,0,1,1,1,0,1,0,0,0,0,1,0,1],
                [1,1,1,1,0,1,0,1,0,0,0,1,1,1,0,0,0,0,0,0,1,0,1,0,0,0,0,0,1,1,0,1,0,0,1,0,0,0,1,0,1,0,1,0,0,0,0,1,0,0,1,1,1,0],
                [1,0,1,0,1,0,1,0,1,1,1,1,0,0,1,1,1,1,1,1,0,0,0,0,1,1,0,1,0,0,0,1,0,0,1,1,0,1,1,1,0,1,1,1,1,1,0,0,0,1,1,0,0,0],
                [0,0,1,0,0,0,1,0,0,1,1,0,0,0,0,1,0,0,0,0,0,0,0,0,1,0,0,0,0,0,0,0,0,0,0,0,1,0,0,0,0,0,0,0,1,1,1,0,1,1,0,0,0,0],
                [1,0,0,1,0,1,0,0,0,1,0,0,1,0,0,0,0,0,0,0,0,1,0,0,0,0,1,0,1,0,1,1,1,0,0,0,0,0,0,0,0,0,0,0,0,0,0,0,1,0,0,0,1,0]]
            \end{verbatim}
        }
        Row corresponds to one class, columns corresponds to one students.
        1 means the student wants to attend the class.

\end{enumerate}
\end{document}
