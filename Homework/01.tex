\documentclass[11pt, oneside]{article}
\usepackage[letterpaper, margin=2cm]{geometry}
\usepackage{MATH566}

\begin{document}
\noindent \textbf{\Large{Caleb Logemann \\
MATH 566 Discrete Optimization\\
Homework 1
}}

%\lstinputlisting[language=Matlab]{H01_23.m}

\noindent In order to find the recommended daily intake of nutrients, I used the website
\\ \href{https://fnic.nal.usda.gov/fnic/interactiveDRI/}{https://fnic.nal.usda.gov/fnic/interactiveDRI/}.
This website gives recommended daily intake based on height, weight, age,
gender, and activity level.
The following table summarizes the recommended daily intake for the professor.
    \begin{center}
        \begin{tabular}{lr}
            \toprule
            Nutrient      & Daily Requirement \\
            \midrule
            Calories      & 2779 kcal \\
            Carbohydrates & 383 grams  \\
            Fiber         & 38 grams \\
            Protein       & 60 grams \\
            Sodium        & 1.5 grams \\
            \bottomrule
        \end{tabular}
    \end{center}

\noindent I then went to Walmart to do some grocery shopping. 
The following table summarizes the nutritional content and cost of 15
different foods.

    \begin{center}
        \begin{tabular}{lccccccc}
            \toprule
            Food                 & Calories & Carbohydrates & Fiber & Protein &  Sodium &   Cost \\
            \midrule
            Gatorade (20 oz)     &      125 &          35 g &   0 g &     0 g & 0.275 g & \$0.75 \\ % 1
            Peanut Butter (32 g) &      190 &           7 g &   2 g &     7 g & 0.130 g & \$0.13 \\ % 2
            Apple sauce (128 g)  &      110 &          27 g &   1 g &     0 g &     0 g & \$0.34 \\ % 3
            Blackberry Jam (20 g)&       50 &          13 g &   0 g &     0 g &     0 g & \$0.13 \\ % 4
            Frozen Waffles (70 g)&      190 &          27 g &   1 g &     4 g & 0.360 g & \$0.48 \\ % 5
            Granola Bar          &      100 &          18 g &   1 g &     2 g &  0.06 g & \$0.17 \\ % 6
            Corn (125 g)         &       60 &           9 g &   2 g &     2 g &   0.2 g & \$0.21 \\ % 7
            Green Beans (120 g)  &       20 &           4 g &   2 g &     1 g &  0.38 g & \$0.21 \\ % 8
            Peas (125 g)         &       70 &          12 g &   3 g &     4 g &  0.37 g & \$0.21 \\ % 9
            FiberOne Cereal (1/2 cup)&   60 &          25 g &  14 g &     2 g & 0.105 g & \$0.25 \\ % 10
            Cocunut Milk (80 ml) &      120 &           2 g &   0 g &     1 g & 0.025 g & \$0.42 \\ % 11
            Tortilla (49 g)      &      140 &          24 g &   1 g &     4 g &  0.44 g & \$0.26 \\ % 12
            Turkey Breast (52 g) &       80 &           1 g &   0 g &     8 g &  0.69 g & \$0.50 \\ % 13
            Roast Beef (56 g)    &       60 &           0 g &   0 g &    11 g &  0.52 g & \$1.15 \\ % 14
            Hamburger Buns (1.6 oz)&    150 &          25 g &   1 g &     4 g & 0.025 g & \$0.29 \\ % 15
            \bottomrule
        \end{tabular}
    \end{center}

\noindent The following model for APMonitor was used to solve this linear program.

\begin{verbatim}
Model Diet
    Variables
        x[1:15] = 0, >=0
    End Variables

    Equations
        125*x[1] + 190*x[2] + 110*x[3] + 50*x[4] + 190*x[5] + 100*x[6] + 60*x[7]
        + 20*x[8] + 70*x[9] + 60*x[10] + 120*x[11] + 140*x[12] + 80*x[13]
        + 60*x[14] + 150*x[15] >= 2779
        35*x[1] + 7*x[2] + 27*x[3] + 13*x[4] + 27*x[5] + 18*x[6] + 9*x[7]
        + 4*x[8] + 12*x[9] + 25*x[10] + 2*x[11] + 24*x[12] + 1*x[13] + 25*x[15] >= 383
        2*x[2] + x[3] + x[5] + x[6] + 2*x[7] + 2*x[8] + 3*x[9] + 14*x[10]
        + x[12] + x[15] >= 38
        7*x[2] + 4*x[5] + 2*x[6] + 2*x[7] + 1*x[8] + 4*x[9] + 2*x[10] + x[11]
        + 4*x[12] + 8*x[13] + 11*x[14] + 4*x[15] >= 60
        0.275*x[1] + 0.13*x[2] + 0.36*x[5] + 0.06*x[6] + 0.2*x[7] + 0.38*x[8]
        + 0.37*x[9] + 0.105*x[10] + 0.025*x[11] + 0.44*x[12] + 0.69*x[13]
        + 0.52*x[14] + 0.025*x[15] >= 1.5*1.4
        minimize 0.75*x[1] + 0.13*x[2] + 0.34*x[3] + 0.13*x[4] + 0.48*x[5]
        + 0.17*x[6] + 0.21*x[7] + 0.21*x[8] + 0.21*x[9] + 0.25*x[10]
        + 0.42*x[11] + 0.26*x[12] + 0.5*x[13] + 1.15*x[14] + 0.29*x[15]
    End Equations
End Model
\end{verbatim}

\noindent The solution that APMonitor found was
\begin{verbatim}
Objective Value = 3.95434986
diet.slk_1 = 0.0
diet.slk_2 = 0.0
diet.slk_3 = 0.0
diet.slk_4 = 12.1406002045
diet.slk_5 = 0.0
diet.x[10] = 0.752789795399
diet.x[11] = 0.0
diet.x[12] = 0.859616100788
diet.x[13] = 0.0
diet.x[14] = 0.0
diet.x[15] = 0.0
diet.x[1] = 0.0
diet.x[2] = 4.66463518143
diet.x[3] = 0.0
diet.x[4] = 0.0
diet.x[5] = 0.0
diet.x[6] = 17.2720603943
diet.x[7] = 0.0
diet.x[8] = 0.0
diet.x[9] = 0.0
\end{verbatim}

This means that for \$3.95 a day, the professor should eat .75 servings of
FiberOne cereal, .86 tortillas, 4.66 servings of peanut butter and 17.27 granola bars.

The following table summarizes my daily recommended intake of nutrients.
\begin{center}
    \begin{tabular}{lr}
        \toprule
        Nutrient      & Daily Requirement \\
        \midrule
        Calories      & 3397 kcal \\
        Carbohydrates & 467 grams  \\
        Fiber         & 38 grams \\
        Protein       & 73 grams \\
        Sodium        & 1.5 grams \\
        \bottomrule
    \end{tabular}
\end{center}

In order to solve this linear program for my daily requirements, I used the
following model.
\begin{verbatim}
Model Diet
    Variables
        x[1:15] = 0, >=0
    End Variables

    Equations
        125*x[1] + 190*x[2] + 110*x[3] + 50*x[4] + 190*x[5] + 100*x[6] + 60*x[7]
        + 20*x[8] + 70*x[9] + 60*x[10] + 120*x[11] + 140*x[12] + 80*x[13]
        + 60*x[14] + 150*x[15] >= 3397
        35*x[1] + 7*x[2] + 27*x[3] + 13*x[4] + 27*x[5] + 18*x[6] + 9*x[7]
        + 4*x[8] + 12*x[9] + 25*x[10] + 2*x[11] + 24*x[12] + 1*x[13] + 25*x[15] >= 467
        2*x[2] + x[3] + x[5] + x[6] + 2*x[7] + 2*x[8] + 3*x[9] + 14*x[10]
        + x[12] + x[15] >= 38
        7*x[2] + 4*x[5] + 2*x[6] + 2*x[7] + 1*x[8] + 4*x[9] + 2*x[10] + x[11]
        + 4*x[12] + 8*x[13] + 11*x[14] + 4*x[15] >= 73
        0.275*x[1] + 0.13*x[2] + 0.36*x[5] + 0.06*x[6] + 0.2*x[7] + 0.38*x[8]
        + 0.37*x[9] + 0.105*x[10] + 0.025*x[11] + 0.44*x[12] + 0.69*x[13]
        + 0.52*x[14] + 0.025*x[15] >= 1.5
        minimize 0.75*x[1] + 0.13*x[2] + 0.34*x[3] + 0.13*x[4] + 0.48*x[5]
        + 0.17*x[6] + 0.21*x[7] + 0.21*x[8] + 0.21*x[9] + 0.25*x[10]
        + 0.42*x[11] + 0.26*x[12] + 0.5*x[13] + 1.15*x[14] + 0.29*x[15]
    End Equations
End Model
\end{verbatim}

The solution to this linear program was
\begin{verbatim}
Objective Value = 4.76215906
diet.slk_1 = 0.0
diet.slk_2 = 0.0
diet.slk_3 = 0.0
diet.slk_4 = 12.5778903961
diet.slk_5 = 0.643295824528
diet.x[10] = 0.260098397732
diet.x[11] = 0.0
diet.x[12] = 0.0
diet.x[13] = 0.0
diet.x[14] = 0.0
diet.x[15] = 0.0
diet.x[1] = 0.0
diet.x[2] = 5.44681596756
diet.x[3] = 0.0
diet.x[4] = 0.0
diet.x[5] = 0.0
diet.x[6] = 23.4649906158
diet.x[7] = 0.0
diet.x[8] = 0.0
diet.x[9] = 0.0
\end{verbatim}

This means that for \$4.76 a day, I should eat .26 servings of FiberOne cereal,
5.45 servings of peanut butter, and 23.465 granola bars.
\end{document}
