\documentclass[11pt, oneside]{article}
\usepackage[letterpaper, margin=2cm]{geometry}
\usepackage{MATH566}

\begin{document}
\noindent \textbf{\Large{Caleb Logemann \\
MATH 566 Discrete Optimization\\
Homework 2
}}

%\lstinputlisting[language=Matlab]{H01_23.m}
\begin{enumerate}
    \item % #1
        \begin{enumerate}
            \item[(a)]
                Give an example of bounded convex sets $C$ and $D$ in $\RR^d$,
                where $C \cap D = \varnothing$, but there is no hyperplane
                stricly seperating $C$ and $D$.
                That is find $a \in \RR^d$ and $b \in \RR$, such that
                \[
                    \v{a}^T\v{x} > b, \forall \v{x} \in C \quad \text{and} \quad \v{a}^T\v{x} < b, \forall \v{x} \in D
                \]

            \item[(b)]
                Give an example of closed convex sets $C$ and $D$ that cannot
                be strictly seperated.
        \end{enumerate}

    \item % #2
        Dualize your diet problem.
        Take the data from your diet problem from HW1, create the dual of the
        program, and solve it.
        Your answer should contain the solution of the dual and the
        interpretation of the results of the dual.

    \item % #3
        A paper mill manufactures rolls of paper of a standard width, 3 meters.
        But customers want to buy rolls of shorter width, and the mill has to
        cut such rolls from the 3m rolls.
        Let us consider an order of
        \begin{itemize}
            \item 97 rolls of width 135 cm,
            \item 610 rolls of width 108 cm,
            \item 395 rolls of width 93 cm, and
            \item 211 rolls of width 42 cm.
        \end{itemize}
        What is the smallest number of 3m rolls that have to be cut in order to
        satisfy this order?

        First we must enumerate all of the different ways that a 3m roll can be
        cut up into these lengths.
        \begin{enumerate}
            \item $c_1 = 2 \times 135$
            \item $c_2 = 135 + 108 + 42$
            \item $c_3 = 135 + 93 + 42$
            \item $c_4 = 135 + 3 \times 42$
            \item $c_5 = 2 \times 108 + 2 \times 42$
            \item $c_6 = 108 + 2 \times 93$
            \item $c_7 = 108 + 93 + 2 \times 42$
            \item $c_8 = 108 + 4 \times 42$
            \item $c_9 = 3 \times 93$.
            \item $c_{10} = 2 \times 93 + 2 \times 42$
            \item $c_{11} = 93 + 4 \times 42$
            \item $c_{12} = 7 \times 42$
        \end{enumerate}
        Any other cuts leave excess paper that could be used for the order so
        won't be necessary.

        Then we can create the integer linear program.
        \begin{align*}
            \min* \sum{i = 1}{12}{c_i} \\
            s.t. 
            2c_1 + c_2 + c_3 + c_4 &\ge 97 \\
            c_2 + 2c_5 + c_6 + c_7 + c_8 &\ge 610 \\
            c_3 + 2 c_6 + c_7 + 3c_9 + 2c_{10} + c_{11} &\ge 395 \\
            c_2 + c_3 + 3c_4 + 2c_5 + 2c_7 + 4c_8 + 2c_{10} + 4c_{11} + 7c_{12} &\ge 211
        \end{align*}

\end{enumerate}
\end{document}
