\documentclass[11pt, oneside]{article}
\usepackage[letterpaper, margin=2cm]{geometry}
\usepackage{MATH566}

\begin{document}
\noindent \textbf{\Large{Caleb Logemann \\
MATH 566 Discrete Optimization\\
Homework 2
}}

%\lstinputlisting[language=Matlab]{H01_23.m}
\begin{enumerate}
    \item % #1

    \item % #2

        The original diet problem can be written as
        \begin{align*}
            \min* \v{c}^T \v{x} \\
            s.t. A\v{x} \ge \v{b}
        \end{align*}
        where
        \begin{align*}
            A =
            \begin{bmatrix}
                125 & 35 & 0 &  0 & 0.275 \\ % 1
                190 &  7 & 2 &  7 & 0.130 \\ % 2
                110 & 27 & 1 &  0 &     0 \\ % 3
                 50 & 13 & 0 &  0 &     0 \\ % 4
                190 & 27 & 1 &  4 & 0.360 \\ % 5
                100 & 18 & 1 &  2 &  0.06 \\ % 6
                 60 &  9 & 2 &  2 &   0.2 \\ % 7
                 20 &  4 & 2 &  1 &  0.38 \\ % 8
                 70 & 12 & 3 &  4 &  0.37 \\ % 9
                 60 & 25 & 4 &  2 & 0.105 \\ % 10
                120 &  2 & 0 &  1 & 0.025 \\ % 11
                140 & 24 & 1 &  4 &  0.44 \\ % 12
                 80 &  1 & 0 &  8 &  0.69 \\ % 13
                 60 &  0 & 0 & 11 &  0.52 \\ % 14
                150 & 25 & 1 &  4 & 0.025    % 15
            \end{bmatrix}
            \v{b} =
            \begin{bmatrix}
                2779 \\
                 383 \\
                  38 \\
                  60 \\
                 1.5
            \end{bmatrix}
            \v{c} =
            \begin{bmatrix}
                0.75 \\ % 1
                0.13 \\ % 2
                0.34 \\ % 3
                0.13 \\ % 4
                0.48 \\ % 5
                0.17 \\ % 6
                0.21 \\ % 7
                0.21 \\ % 8
                0.21 \\ % 9
                0.25 \\ % 10
                0.42 \\ % 11
                0.26 \\ % 12
                0.50 \\ % 13
                1.15 \\ % 14
                0.29    % 15
            \end{bmatrix}
        \end{align*}
        and $\v{x}$ is the number of servings of each type of food.

        The dual of the diet program can be written as follows
        \begin{align*}
            \max \v{b}^T \v{y} \\
            s.t. A^T \v{y} \le \v{c} \\
            \v{y} \ge \v{0}
        \end{align*}

    \item % #3
        A paper mill manufactures rolls of paper of a standard width, 3 meters.
        But customers want to buy rolls of shorter width, and the mill has to
        cut such rolls from the 3m rolls.
        Let us consider an order of
        \begin{itemize}
            \item 97 rolls of width 135 cm,
            \item 610 rolls of width 108 cm,
            \item 395 rolls of width 93 cm, and
            \item 211 rolls of width 42 cm.
        \end{itemize}
        What is the smallest number of 3m rolls that have to be cut in order to
        satisfy this order?

        First we must enumerate all of the different ways that a 3m roll can be
        cut up into these lengths.
        \begin{enumerate}
            \item $c_1 = 2 \times 135$
            \item $c_2 = 135 + 108 + 42$
            \item $c_3 = 135 + 93 + 42$
            \item $c_4 = 135 + 3 \times 42$
            \item $c_5 = 2 \times 108 + 2 \times 42$
            \item $c_6 = 108 + 2 \times 93$
            \item $c_7 = 108 + 93 + 2 \times 42$
            \item $c_8 = 108 + 4 \times 42$
            \item $c_9 = 3 \times 93$.
            \item $c_{10} = 2 \times 93 + 2 \times 42$
            \item $c_{11} = 93 + 4 \times 42$
            \item $c_{12} = 7 \times 42$
        \end{enumerate}
        Any other cuts leave excess paper that could be used for the order so
        won't be necessary.

        Then we can create the integer linear program.
        \begin{align*}
            \min* \sum{i = 1}{12}{c_i} \\
            s.t. 
            2c_1 + c_2 + c_3 + c_4 &\ge 97 \\
            c_2 + 2c_5 + c_6 + c_7 + c_8 &\ge 610 \\
            c_3 + 2 c_6 + c_7 + 3c_9 + 2c_{10} + c_{11} &\ge 395 \\
            c_2 + c_3 + 3c_4 + 2c_5 + 2c_7 + 4c_8 + 2c_{10} + 4c_{11} + 7c_{12} &\ge 211
        \end{align*}

\end{enumerate}
\end{document}
