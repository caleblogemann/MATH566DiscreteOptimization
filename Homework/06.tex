\documentclass[11pt, oneside]{article}
\usepackage[letterpaper, margin=2cm]{geometry}
\usepackage{MATH566}
%\usepackage{sagetex}

\begin{document}
\noindent \textbf{\Large{Caleb Logemann \\
MATH 566 Discrete Optimization\\
Homework 6
}}

%\lstinputlisting[language=Sage]{03_2.sage}
\begin{enumerate}
  \item % #1 Done
    Consider the graph below
    \begin{center}
      \tikzset{insep/.style={inner sep=2pt, outer sep=0pt, circle, fill},}
      \begin{tikzpicture}[scale=2]
        \draw
        (0,0) node[insep,label=left:$s$](s){}
        (1,1) node[insep,label=left:$ $](a){}
        (1,0) node[insep,label=left:$ $](b){}
        (1,-1) node[insep,label=left:$ $](c){}
        (2,1) node[insep,label=left:$ $](d){}
        (2,0) node[insep,label=left:$ $](e){}
        (2,-1) node[insep,label=left:$ $](f){}
        (3,0) node[insep,label=right:$t$](t){}
        ;

        \foreach \x/\y/\n in {s/a/3, a/d/7, d/t/2, s/b/4, b/e/2, e/t/5, b/f/1}{
          \draw[-triangle 45](\x) -- node[above]{\n} (\y);
        }
        \foreach \x/\y/\n in {s/c/8, c/f/2, f/t/10}{
          \draw[-triangle 45](\x) -- node[below]{\n} (\y);
        }
        \foreach \x/\y/\n in {c/b/2, b/a/3, d/e/5, e/f/1}{
          \draw[-triangle 45](\x) -- node[right]{\n} (\y);
        }
      \end{tikzpicture}
    \end{center}
    Find a shortest path and prove optimality using duality (find dual LP and
    its optimal solution)

    First let me redraw the graph with all of the vertices labeled.
    \begin{center}
      \tikzset{insep/.style={inner sep=2pt, outer sep=0pt, circle, fill},}
      \begin{tikzpicture}[scale=2]
        \draw
        (0,0) node[insep,label=left:$s$](s){}
        (1,1) node[insep,label=120:$a$](a){}
        (1,0) node[insep,label=45:$b$](b){}
        (1,-1) node[insep,label=210:$c$](c){}
        (2,1) node[insep,label=45:$d$](d){}
        (2,0) node[insep,label=45:$e$](e){}
        (2,-1) node[insep,label=-45:$f$](f){}
        (3,0) node[insep,label=right:$t$](t){}
        ;

        \foreach \x/\y/\n in {s/a/3, a/d/7, d/t/2, s/b/4, b/e/2, e/t/5, b/f/1}{
          \draw[-triangle 45](\x) -- node[above]{\n} (\y);
        }
        \foreach \x/\y/\n in {s/c/8, c/f/2, f/t/10}{
          \draw[-triangle 45](\x) -- node[below]{\n} (\y);
        }
        \foreach \x/\y/\n in {c/b/2, b/a/3, d/e/5, e/f/1}{
          \draw[-triangle 45](\x) -- node[right]{\n} (\y);
        }
      \end{tikzpicture}
    \end{center}

    From observation it is possible to see that the shortest path is
    $s \to b \to e \to t$ and it has weight 11.

    In order to verify this, the dual of the shortest path linear program can
    be solved.
    The dual of the shortest path linear program is shown below
    \[
      \begin{array}{ll@{}ll}
        \max* & y_t - y_s \\
        \text{s.t.} & y_v - y_u \le c(uv) \quad \forall (u, v) \in E \\
                    & y_s = 0
      \end{array}
    \]

    This linear program can be solved using the following sage script.
    \lstinputlisting[language=Sage]{06_1.sage}
    The output of this script is as follows
    \begin{verbatim}
      Objective Value: 11.0
      y[a] = 2.0
      y[b] = 4.0
      y[c] = 2.0
      y[d] = 9.0
      y[e] = 6.0
      y[f] = 1.0
      y[s] = 0.0
      y[t] = 11.0
    \end{verbatim}
    This shows that the linear program found a path of length 11 and the results
    show that $s \to b \to e \to t$ is adding up the weights on their
    respective edges.
    In other words $y[b] = 4$ and the weight from $s \to b$ is 4.
    The weight from $b \to e$ is 2, so $y[e] = 4 + 2 = 6$, and the
    weight from $e \to t$ is 5, so $y[t] = 6 + 5 = 11$.

  \item % #2 Done
    Consider the network below with given edge values, forming an integer feasible flow.
    Find a list of path and cycle flows whose sum is this flow.
    \begin{center}
      \tikzset{insep/.style={inner sep=2pt, outer sep=0pt, circle, fill},}
      \begin{tikzpicture}[scale=2]
        \draw
        (0,0) node[circle,draw,label=left:$ $](s){s}
        (0,1) node[insep,label=left:$ $](a){}
        (-1,0) node[insep,label=left:$ $](b){}
        (0,-1) node[insep,label=left:$ $](c){}
        (1,0) node[insep,label=left:$ $](d){}
        (3,0) node[draw,circle,label=right:$ $](t){t}
        (3,1) node[insep,label=left:$ $](e){}
        (2,0) node[insep,label=left:$ $](f){}
        (3,-1) node[insep,label=left:$ $](g){}
        (4,0) node[insep,label=left:$ $](h){}
        ;
        \foreach \x/\y/\n in {b/a/2, s/b/0, a/d/2, s/d/3, e/a/1, d/f/2, c/g/1, e/f/0, t/f/1, h/t/1, h/e/1}{
          \draw[-triangle 45](\x) -- node[above]{\n} (\y);
        }
        \foreach \x/\y/\n in {c/b/2, f/g/3, g/h/2}{
          \draw[-triangle 45](\x) -- node[below]{\n} (\y);
        }
        \foreach \x/\y/\n in {a/s/1, s/c/0, e/t/0, g/t/2, d/c/3}{
          \draw[-triangle 45](\x) -- node[right]{\n} (\y);
        }
      \end{tikzpicture}
    \end{center}

    Gallai; Ford and Fulkerson proved a theorem that stated that any flow can be
    decomposed into $s$-$t$ paths, $\mathcal{P}$ and circuits $\mathcal{C}$ with weight function
    $w:\mathcal{P} \cup \mathcal{C} \to \RR^{+}$, such that
    \begin{align*}
      f(e) &= \sum{e \in P \in \mathcal{P}}{}{w(P)} + \sum{e \in C \in \mathcal{C}}{}{w(C)} \\
      value(f) &= \sum{P \in \mathcal{P}}{}{w(P)} \\
      \abs{\mathcal{P} + \mathcal{C}} &\le \abs{E(G)}
    \end{align*}
    In order to find such a decomposition I will first relabel all of the
    vertices as follows.
    \begin{center}
      \tikzset{insep/.style={inner sep=2pt, outer sep=0pt, circle, fill},}
      \begin{tikzpicture}[scale=2]
        \draw
        (0,0) node[circle,draw,label=left:$ $](s){s}
        (0,1) node[insep,label=135:$a$](a){}
        (-1,0) node[insep,label=left:$b$](b){}
        (0,-1) node[insep,label=225:$c$](c){}
        (1,0) node[insep,label=45:$d$](d){}
        (3,0) node[draw,circle,label=right:$ $](t){t}
        (3,1) node[insep,label=45:$e$](e){}
        (2,0) node[insep,label=135:$f$](f){}
        (3,-1) node[insep,label=315:$g$](g){}
        (4,0) node[insep,label=right:$h$](h){}
        ;
        \foreach \x/\y/\n in {b/a/2, s/b/0, a/d/2, s/d/3, e/a/1, d/f/2, c/g/1, e/f/0, t/f/1, h/t/1, h/e/1}{
          \draw[-triangle 45](\x) -- node[above]{\n} (\y);
        }
        \foreach \x/\y/\n in {c/b/2, f/g/3, g/h/2}{
          \draw[-triangle 45](\x) -- node[below]{\n} (\y);
        }
        \foreach \x/\y/\n in {a/s/1, s/c/0, e/t/0, g/t/2, d/c/3}{
          \draw[-triangle 45](\x) -- node[right]{\n} (\y);
        }
      \end{tikzpicture}
    \end{center}

    Now one way this flow can be decomposed is by using 1 path and 3 circuits
    as follows.
    The one $s$-$t$ path is $P = s \to d \to f \to g \to t$ with weight $w(P) = 2$.
    \begin{center}
      \tikzset{insep/.style={inner sep=2pt, outer sep=0pt, circle, fill},}
      \begin{tikzpicture}[scale=2]
        \draw
        (0,0) node[circle,draw,label=left:$ $](s){s}
        (0,1) node[insep,label=135:$a$](a){}
        (-1,0) node[insep,label=left:$b$](b){}
        (0,-1) node[insep,label=225:$c$](c){}
        (1,0) node[insep,label=45:$d$](d){}
        (3,0) node[draw,circle,label=right:$ $](t){t}
        (3,1) node[insep,label=45:$e$](e){}
        (2,0) node[insep,label=135:$f$](f){}
        (3,-1) node[insep,label=315:$g$](g){}
        (4,0) node[insep,label=right:$h$](h){}
        ;
        \foreach \x/\y/\n in {b/a/, s/b/, a/d/, s/d/2, e/a/, d/f/2, c/g/, e/f/, t/f/, h/t/, h/e/}{
          \draw[-triangle 45](\x) -- node[above]{\n} (\y);
        }
        \foreach \x/\y/\n in {c/b/, f/g/2, g/h/}{
          \draw[-triangle 45](\x) -- node[below]{\n} (\y);
        }
        \foreach \x/\y/\n in {a/s/, s/c/, e/t/, g/t/2, d/c/}{
          \draw[-triangle 45](\x) -- node[right]{\n} (\y);
        }
      \end{tikzpicture}
    \end{center}

    The first circuit is $C_1 = d \to c \to b \to a \to d$ with weight $w(C_1) = 2$.
    \begin{center}
      \tikzset{insep/.style={inner sep=2pt, outer sep=0pt, circle, fill},}
      \begin{tikzpicture}[scale=2]
        \draw
        (0,0) node[circle,draw,label=left:$ $](s){s}
        (0,1) node[insep,label=135:$a$](a){}
        (-1,0) node[insep,label=left:$b$](b){}
        (0,-1) node[insep,label=225:$c$](c){}
        (1,0) node[insep,label=45:$d$](d){}
        (3,0) node[draw,circle,label=right:$ $](t){t}
        (3,1) node[insep,label=45:$e$](e){}
        (2,0) node[insep,label=135:$f$](f){}
        (3,-1) node[insep,label=315:$g$](g){}
        (4,0) node[insep,label=right:$h$](h){}
        ;
        \foreach \x/\y/\n in {b/a/2, s/b/, a/d/2, s/d/, e/a/, d/f/, c/g/, e/f/, t/f/, h/t/, h/e/}{
          \draw[-triangle 45](\x) -- node[above]{\n} (\y);
        }
        \foreach \x/\y/\n in {c/b/2, f/g/, g/h/}{
          \draw[-triangle 45](\x) -- node[below]{\n} (\y);
        }
        \foreach \x/\y/\n in {a/s/, s/c/, e/t/, g/t/, d/c/2}{
          \draw[-triangle 45](\x) -- node[right]{\n} (\y);
        }
      \end{tikzpicture}
    \end{center}

    The second circuit is $C_2 = a \to s \to d \to c \to g \to h \to e \to a$
    with weight $w(C_2) = 1$.
    \begin{center}
      \tikzset{insep/.style={inner sep=2pt, outer sep=0pt, circle, fill},}
      \begin{tikzpicture}[scale=2]
        \draw
        (0,0) node[circle,draw,label=left:$ $](s){s}
        (0,1) node[insep,label=135:$a$](a){}
        (-1,0) node[insep,label=left:$b$](b){}
        (0,-1) node[insep,label=225:$c$](c){}
        (1,0) node[insep,label=45:$d$](d){}
        (3,0) node[draw,circle,label=right:$ $](t){t}
        (3,1) node[insep,label=45:$e$](e){}
        (2,0) node[insep,label=135:$f$](f){}
        (3,-1) node[insep,label=315:$g$](g){}
        (4,0) node[insep,label=right:$h$](h){}
        ;
        \foreach \x/\y/\n in {b/a/, s/b/, a/d/, s/d/1, e/a/1, d/f/, c/g/1, e/f/, t/f/, h/t/, h/e/1}{
          \draw[-triangle 45](\x) -- node[above]{\n} (\y);
        }
        \foreach \x/\y/\n in {c/b/, f/g/, g/h/1}{
          \draw[-triangle 45](\x) -- node[below]{\n} (\y);
        }
        \foreach \x/\y/\n in {a/s/1, s/c/, e/t/, g/t/, d/c/1}{
          \draw[-triangle 45](\x) -- node[right]{\n} (\y);
        }
      \end{tikzpicture}
    \end{center}

    The third and final circuit is $C_3 = f \to g \to h \to t \to f$ with
    weight $w(C_3) = 1$.
    \begin{center}
      \tikzset{insep/.style={inner sep=2pt, outer sep=0pt, circle, fill},}
      \begin{tikzpicture}[scale=2]
        \draw
        (0,0) node[circle,draw,label=left:$ $](s){s}
        (0,1) node[insep,label=135:$a$](a){}
        (-1,0) node[insep,label=left:$b$](b){}
        (0,-1) node[insep,label=225:$c$](c){}
        (1,0) node[insep,label=45:$d$](d){}
        (3,0) node[draw,circle,label=right:$ $](t){t}
        (3,1) node[insep,label=45:$e$](e){}
        (2,0) node[insep,label=135:$f$](f){}
        (3,-1) node[insep,label=315:$g$](g){}
        (4,0) node[insep,label=right:$h$](h){}
        ;
        \foreach \x/\y/\n in {b/a/, s/b/, a/d/, s/d/, e/a/, d/f/, c/g/, e/f/, t/f/1, h/t/1, h/e/}{
          \draw[-triangle 45](\x) -- node[above]{\n} (\y);
        }
        \foreach \x/\y/\n in {c/b/, f/g/1, g/h/1}{
          \draw[-triangle 45](\x) -- node[below]{\n} (\y);
        }
        \foreach \x/\y/\n in {a/s/, s/c/, e/t/, g/t/, d/c/}{
          \draw[-triangle 45](\x) -- node[right]{\n} (\y);
        }
      \end{tikzpicture}
    \end{center}

    This set of paths and circuits with corresponding weight function satisfies
    all three of the properties stated in the Theorem.
    The sum of these paths and circuits results in the original flow.
    The sum of the weights on the paths is equal to the value of the flow.
    Also the number of paths and circuits is less than the number of edges of
    $G$.

  \item % #3 Done
    Consider the network below with given capacity and flow values.
    (The  edge label $f,u$ means flow-value $f$ and capacity $u$.) 
    Find augmenting paths and augment the flow to a maximum flow.
    Provide the list of residual graphs AND augmenting paths.
    In other words, run Ford-Fulkerson algorithm. 
    \begin{center}
      \tikzset{insep/.style={inner sep=2pt, outer sep=0pt, circle, fill},}
      \begin{tikzpicture}[scale=2]
        \draw
        (0,0) node[insep,label=left:$s$](s){}
        (1,1) node[insep,label=left:$ $](a){}
        (1,0) node[insep,label=left:$ $](b){}
        (1,-1) node[insep,label=left:$ $](c){}
        (2,1) node[insep,label=left:$ $](d){}
        (2,0) node[insep,label=left:$ $](e){}
        (2,-1) node[insep,label=left:$ $](f){}
        (3,0) node[insep,label=right:$t$](t){}
        ;
        \foreach \x/\y/\n in {s/a/{3,3}, a/d/{3,7}, d/t/{2,2}, s/b/{1,4}, b/e/{1,1}, e/t/{1,8}, b/f/{0,2}}{
          \draw[-triangle 45](\x) -- node[above]{\n} (\y);
        }
        \foreach \x/\y/\n in {s/c/{2,8}, c/f/{2,4}, f/t/{3,6}}{
          \draw[-triangle 45](\x) -- node[below]{\n} (\y);
        }
        \foreach \x/\y/\n in {c/b/{0,2}, b/a/{0,3}, d/e/{1,5}, e/f/{1,1}}{
          \draw[-triangle 45](\x) -- node[right]{\n} (\y);
        }
      \end{tikzpicture}
    \end{center}

    I will show the augmenting paths and the flow on the original graph
    however I have attached the residual graphs on a separate sheet of paper.
    First let me relabel the vertices.
    \begin{center}
      \tikzset{insep/.style={inner sep=2pt, outer sep=0pt, circle, fill},}
      \begin{tikzpicture}[scale=2]
        \draw
        (0,0) node[insep,label=left:$s$](s){}
        (1,1) node[insep,label=120:$a$](a){}
        (1,0) node[insep,label=45:$b$](b){}
        (1,-1) node[insep,label=210:$c$](c){}
        (2,1) node[insep,label=45:$d$](d){}
        (2,0) node[insep,label=45:$e$](e){}
        (2,-1) node[insep,label=-45:$f$](f){}
        (3,0) node[insep,label=right:$t$](t){}
        ;
        \foreach \x/\y/\n in {s/a/{3,3}, a/d/{3,7}, d/t/{2,2}, s/b/{1,4}, b/e/{1,1}, e/t/{1,8}, b/f/{0,2}}{
          \draw[-triangle 45](\x) -- node[above]{\n} (\y);
        }
        \foreach \x/\y/\n in {s/c/{2,8}, c/f/{2,4}, f/t/{3,6}}{
          \draw[-triangle 45](\x) -- node[below]{\n} (\y);
        }
        \foreach \x/\y/\n in {c/b/{0,2}, b/a/{0,3}, d/e/{1,5}, e/f/{1,1}}{
          \draw[-triangle 45](\x) -- node[right]{\n} (\y);
        }
      \end{tikzpicture}
    \end{center}
    The initial augmenting path will be $P = s \to b \to a \to d \to e \to t$.
    The minimum capacity over this path is $\gamma = 3$.
    Augmenting on this path gives the following flow.
    \begin{center}
      \tikzset{insep/.style={inner sep=2pt, outer sep=0pt, circle, fill},}
      \begin{tikzpicture}[scale=2]
        \draw
        (0,0) node[insep,label=left:$s$](s){}
        (1,1) node[insep,label=120:$a$](a){}
        (1,0) node[insep,label=45:$b$](b){}
        (1,-1) node[insep,label=210:$c$](c){}
        (2,1) node[insep,label=45:$d$](d){}
        (2,0) node[insep,label=45:$e$](e){}
        (2,-1) node[insep,label=-45:$f$](f){}
        (3,0) node[insep,label=right:$t$](t){}
        ;
        \foreach \x/\y/\n in {s/a/{3,3}, a/d/{6,7}, d/t/{2,2}, s/b/{4,4}, b/e/{1,1}, e/t/{4,8}, b/f/{0,2}}{
          \draw[-triangle 45](\x) -- node[above]{\n} (\y);
        }
        \foreach \x/\y/\n in {s/c/{2,8}, c/f/{2,4}, f/t/{3,6}}{
          \draw[-triangle 45](\x) -- node[below]{\n} (\y);
        }
        \foreach \x/\y/\n in {c/b/{0,2}, b/a/{3,3}, d/e/{4,5}, e/f/{1,1}}{
          \draw[-triangle 45](\x) -- node[right]{\n} (\y);
        }
      \end{tikzpicture}
    \end{center}

    The second augmenting path I will use is $P = s \to c \to f \to t$ with a
    minimum capacity of $\gamma = 2$.
    The new flow will be
    \begin{center}
      \tikzset{insep/.style={inner sep=2pt, outer sep=0pt, circle, fill},}
      \begin{tikzpicture}[scale=2]
        \draw
        (0,0) node[insep,label=left:$s$](s){}
        (1,1) node[insep,label=120:$a$](a){}
        (1,0) node[insep,label=45:$b$](b){}
        (1,-1) node[insep,label=210:$c$](c){}
        (2,1) node[insep,label=45:$d$](d){}
        (2,0) node[insep,label=45:$e$](e){}
        (2,-1) node[insep,label=-45:$f$](f){}
        (3,0) node[insep,label=right:$t$](t){}
        ;
        \foreach \x/\y/\n in {s/a/{3,3}, a/d/{6,7}, d/t/{2,2}, s/b/{4,4}, b/e/{1,1}, e/t/{4,8}, b/f/{0,2}}{
          \draw[-triangle 45](\x) -- node[above]{\n} (\y);
        }
        \foreach \x/\y/\n in {s/c/{4,8}, c/f/{4,4}, f/t/{5,6}}{
          \draw[-triangle 45](\x) -- node[below]{\n} (\y);
        }
        \foreach \x/\y/\n in {c/b/{0,2}, b/a/{3,3}, d/e/{4,5}, e/f/{1,1}}{
          \draw[-triangle 45](\x) -- node[right]{\n} (\y);
        }
      \end{tikzpicture}
    \end{center}

    The third augmenting path is $P = s \to c \to b \to f \to t$ with minimum
    capacity, $\gamma = 1$.
    Augmenting along this path gives
    \begin{center}
      \tikzset{insep/.style={inner sep=2pt, outer sep=0pt, circle, fill},}
      \begin{tikzpicture}[scale=2]
        \draw
        (0,0) node[insep,label=left:$s$](s){}
        (1,1) node[insep,label=120:$a$](a){}
        (1,0) node[insep,label=45:$b$](b){}
        (1,-1) node[insep,label=210:$c$](c){}
        (2,1) node[insep,label=45:$d$](d){}
        (2,0) node[insep,label=45:$e$](e){}
        (2,-1) node[insep,label=-45:$f$](f){}
        (3,0) node[insep,label=right:$t$](t){}
        ;
        \foreach \x/\y/\n in {s/a/{3,3}, a/d/{6,7}, d/t/{2,2}, s/b/{4,4}, b/e/{1,1}, e/t/{4,8}, b/f/{1,2}}{
          \draw[-triangle 45](\x) -- node[above]{\n} (\y);
        }
        \foreach \x/\y/\n in {s/c/{5,8}, c/f/{4,4}, f/t/{6,6}}{
          \draw[-triangle 45](\x) -- node[below]{\n} (\y);
        }
        \foreach \x/\y/\n in {c/b/{1,2}, b/a/{3,3}, d/e/{4,5}, e/f/{1,1}}{
          \draw[-triangle 45](\x) -- node[right]{\n} (\y);
        }
      \end{tikzpicture}
    \end{center}

    The fourth augmenting path is $P = s \to c \to b \to f \to e \to t$ with
    minimum capacity, $\gamma = 1$.
    Note that this is decreasing the flow on $f \to e$.
    \begin{center}
      \tikzset{insep/.style={inner sep=2pt, outer sep=0pt, circle, fill},}
      \begin{tikzpicture}[scale=2]
        \draw
        (0,0) node[insep,label=left:$s$](s){}
        (1,1) node[insep,label=120:$a$](a){}
        (1,0) node[insep,label=45:$b$](b){}
        (1,-1) node[insep,label=210:$c$](c){}
        (2,1) node[insep,label=45:$d$](d){}
        (2,0) node[insep,label=45:$e$](e){}
        (2,-1) node[insep,label=-45:$f$](f){}
        (3,0) node[insep,label=right:$t$](t){}
        ;
        \foreach \x/\y/\n in {s/a/{3,3}, a/d/{6,7}, d/t/{2,2}, s/b/{4,4}, b/e/{1,1}, e/t/{5,8}, b/f/{2,2}}{
          \draw[-triangle 45](\x) -- node[above]{\n} (\y);
        }
        \foreach \x/\y/\n in {s/c/{6,8}, c/f/{4,4}, f/t/{6,6}}{
          \draw[-triangle 45](\x) -- node[below]{\n} (\y);
        }
        \foreach \x/\y/\n in {c/b/{2,2}, b/a/{3,3}, d/e/{4,5}, e/f/{0,1}}{
          \draw[-triangle 45](\x) -- node[right]{\n} (\y);
        }
      \end{tikzpicture}
    \end{center}
    This is last augmenting path, and so this flow is optimal.

  \item % #4
    Let $(G, u, s, t)$ be a network, and let $\delta^+(X)$ and $\delta^+(Y)$ be
    minimum $s$-$t$-cuts in $(G,u)$.
    Show that $\delta^+(X\cap Y)$ and $\delta^+(X \cup Y)$ are also minimum
    $s$-$t$-cuts in $(G, u, s, t)$.

    \begin{proof}
      Let $(G, u, s, t)$ be a network with edges $E$ and vertices $V$.
      Let $X, Y \subset V$ such that $\delta^+(X)$ and $\delta^+(Y)$ are minimum
      $s$-$t$ cuts.
      Consider $X \cap Y$
    \end{proof}

  \item % #5
    Show that in case of irrational capacities, the Ford-Fulkerson algorithm may
    not terminate at all.
    Hint: See the Korte book (in particular exercises on page 199.).
    It contains the following network:
    \begin{center}
      \tikzset{insep/.style={inner sep=2pt, outer sep=0pt, circle, fill},}
      \begin{tikzpicture}[scale=2]
        \draw
        (0,-0.5) node[insep,label=left:$s$](s){}
        (1,1) node[insep,label=left:$ $](a1){}
        (1,0) node[insep,label=left:$ $](a2){}
        (1,-1) node[insep,label=left:$ $](a3){}
        (1,-2) node[insep,label=left:$ $](a4){}
        (2,1) node[insep,label=left:$ $](b1){}
        (2,0) node[insep,label=left:$ $](b2){}
        (2,-1) node[insep,label=left:$ $](b3){}
        (2,-2) node[insep,label=left:$ $](b4){}
        (3,-0.5) node[insep,label=right:$t$](t){}
        ;
        \foreach \x\n in {1,2,3,4}{
          \foreach \y\n in {1,2,3,4}{
            \draw (a\x) -- (b\y);
          }
        }
        \foreach \x\n in {1,2,3,4}{
          \draw (s) -- (a\x);
          \draw (b\x) -- (t);
        }
        \draw[-triangle 45, line width=1pt](a1) --node[above]{1} (b1);
        \draw[-triangle 45, line width=1pt](a2) --node[above]{$\sigma$} (b2);
        \draw[-triangle 45, line width=1pt](a3) --node[above]{$\sigma^2$} (b3);
      \end{tikzpicture}
    \end{center}
    Where $\sigma = \frac{\sqrt{5}-1}{2}$. Note that $\sigma$ satisfies
    $\sigma^n = \sigma^{n+1} + \sigma^{n+2}$. All other capacities are 1.

    In order to show that the Ford-Fulkerson algorithm may not terminate it must
    be shown that there is an infinite sequence of augmenting paths.

  \item % #6
    Red-Blue meta algorithm for MST.
    Let $G$ be a graph and $w$ be a weight assignment to $E(G)$.
    Assume that all weights are distinct.
    Start with all edges being uncolored.
    Apply the following rules as long as possible.
    \begin{itemize}
      \item if $e \in E$ is in a cycle $C$ where $e$ is the heaviest edge, color $e$ red
      \item if there is a cut where $e\in E$ is the lightest edge, color $e$ blue.
    \end{itemize}
    Claim is that blue edges form a minimum spanning tree.
    \begin{itemize}
      \item Show that red edge cannot be in MST.
      \item Show that blue edge must be in MST.
      \item Show that blue edges form a tree
      \item Show that every edge gets colored.
      \item Show that no edge satisfies both red and blue criteria. (i.e. every edge has one color).
    \end{itemize}

  \item % #7 Done
    Implement Edmonds-Karp algorithm and run it on the network from question three.
    Print the sequence of augmenting paths used by your implementation.
    Print the flow and its value.

    I implemented the Edmonds-Karp algorithm in the following function.
    \lstinputlisting[language=Sage]{edmondsKarp.sage}

    This algorithm using a breadth first search which is implemented in the
    following function.
    \lstinputlisting[language=Sage]{breadthFirstSearch.sage}

    In order to run this algorithm on the graph from problem 3, I first
    relabeled the vertices in this graph.
    The graph was relabeled as shown below.
    \begin{center}
      \tikzset{insep/.style={inner sep=2pt, outer sep=0pt, circle, fill},}
      \begin{tikzpicture}[scale=2]
        \draw
        (0,0) node[insep,label=left:$s$](s){}
        (1,1) node[insep,label=120:$a$](a){}
        (1,0) node[insep,label=45:$b$](b){}
        (1,-1) node[insep,label=210:$c$](c){}
        (2,1) node[insep,label=45:$d$](d){}
        (2,0) node[insep,label=45:$e$](e){}
        (2,-1) node[insep,label=-45:$f$](f){}
        (3,0) node[insep,label=right:$t$](t){}
        ;
        \foreach \x/\y in {s/a, a/d, d/t, s/b, b/e, e/t, b/f, s/c, c/f, f/t, c/b, b/a, d/e, e/f}{
          \draw[-triangle 45](\x) -- (\y);
        }
      \end{tikzpicture}
    \end{center}
    \lstinputlisting[language=Sage]{06_7.sage}
    This is the output of this script.
    Each list is the augmenting path.
    Each tuple is an edge in the augmenting path, with first entry the starting
    vertex, the second entry the ending vertex, and the third entry the
    available flow.
    The dictionary shows the flow on each edge in the form edge:flow.
    \begin{verbatim}
[('s', 'a', 3), ('a', 'd', 7), ('d', 't', 2)]
[('s', 'b', 4), ('b', 'e', 1), ('e', 't', 8)]
[('s', 'b', 3), ('b', 'f', 2), ('f', 't', 6)]
[('s', 'c', 8), ('c', 'f', 4), ('f', 't', 4)]
[('s', 'a', 1), ('a', 'd', 5), ('d', 'e', 5), ('e', 't', 7)]
[('s', 'b', 1), ('b', 'a', 3), ('a', 'd', 4), ('d', 'e', 4), ('e', 't', 6)]
[('s', 'c', 4), ('c', 'b', 2), ('b', 'a', 2), ('a', 'd', 3),
  ('d', 'e', 3), ('e', 't', 5)]
{
  ('b', 'f', 2): 2,
  ('c', 'b', 2): 2,
  ('b', 'a', 3): 3,
  ('f', 't', 6): 6,
  ('s', 'b', 4): 4,
  ('e', 'f', 1): 0,
  ('a', 'd', 7): 6,
  ('s', 'c', 8): 6,
  ('d', 'e', 5): 4,
  ('s', 'a', 3): 3,
  ('b', 'e', 1): 1,
  ('c', 'f', 4): 4,
  ('d', 't', 2): 2,
  ('e', 't', 8): 5
  }
}
    \end{verbatim}
    This flow can also be shown on the graph as follows.
    \begin{center}
      \tikzset{insep/.style={inner sep=2pt, outer sep=0pt, circle, fill},}
      \begin{tikzpicture}[scale=2]
        \draw
        (0,0) node[insep,label=left:$s$](s){}
        (1,1) node[insep,label=120:$a$](a){}
        (1,0) node[insep,label=45:$b$](b){}
        (1,-1) node[insep,label=210:$c$](c){}
        (2,1) node[insep,label=45:$d$](d){}
        (2,0) node[insep,label=45:$e$](e){}
        (2,-1) node[insep,label=-45:$f$](f){}
        (3,0) node[insep,label=right:$t$](t){}
        ;
        \foreach \x/\y/\n in {s/a/{3,3}, a/d/{6,7}, d/t/{2,2}, s/b/{4,4}, b/e/{1,1}, e/t/{5,8}, b/f/{2,2}}{
          \draw[-triangle 45](\x) -- node[above]{\n} (\y);
        }
        \foreach \x/\y/\n in {s/c/{6,8}, c/f/{4,4}, f/t/{6,6}}{
          \draw[-triangle 45](\x) -- node[below]{\n} (\y);
        }
        \foreach \x/\y/\n in {c/b/{2,2}, b/a/{3,3}, d/e/{4,5}, e/f/{0,1}}{
          \draw[-triangle 45](\x) -- node[right]{\n} (\y);
        }
      \end{tikzpicture}
    \end{center}
\end{enumerate}
\end{document}
