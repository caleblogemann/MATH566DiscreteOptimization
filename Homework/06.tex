\documentclass[11pt, oneside]{article}
\usepackage[letterpaper, margin=2cm]{geometry}
\usepackage{MATH566}
%\usepackage{sagetex}

\begin{document}
\noindent \textbf{\Large{Caleb Logemann \\
MATH 566 Discrete Optimization\\
Homework 6
}}

%\lstinputlisting[language=Sage]{03_2.sage}
\begin{enumerate}
  \item % #1
    Consider the graph below
    \begin{center}
      \tikzset{insep/.style={inner sep=2pt, outer sep=0pt, circle, fill},}
      \begin{tikzpicture}[scale=2]
        \draw
        (0,0) node[insep,label=left:$s$](s){}
        (1,1) node[insep,label=left:$ $](a){}
        (1,0) node[insep,label=left:$ $](b){}
        (1,-1) node[insep,label=left:$ $](c){}
        (2,1) node[insep,label=left:$ $](d){}
        (2,0) node[insep,label=left:$ $](e){}
        (2,-1) node[insep,label=left:$ $](f){}
        (3,0) node[insep,label=right:$t$](t){}
        ;

        \foreach \x/\y/\n in {s/a/3, a/d/7, d/t/2, s/b/4, b/e/2, e/t/5, b/f/1}{
          \draw[-triangle 45](\x) -- node[above]{\n} (\y);
        }
        \foreach \x/\y/\n in {s/c/8, c/f/2, f/t/10}{
          \draw[-triangle 45](\x) -- node[below]{\n} (\y);
        }
        \foreach \x/\y/\n in {c/b/2, b/a/3, d/e/5, e/f/1}{
          \draw[-triangle 45](\x) -- node[right]{\n} (\y);
        }
      \end{tikzpicture}
    \end{center}
    Find a shortest path and prove optimality using duality (find dual LP and
    its optimal solution)

  \item % #2
    Consider the network below with given edge values, forming an integer feasible flow.
    Find a list of path and cycle flows whose sum is this flow.
    \begin{center}
      \tikzset{insep/.style={inner sep=2pt, outer sep=0pt, circle, fill},}
      \begin{tikzpicture}[scale=2]
        \draw
        (0,0) node[circle,draw,label=left:$ $](s){s}
        (0,1) node[insep,label=left:$ $](a){}
        (-1,0) node[insep,label=left:$ $](b){}
        (0,-1) node[insep,label=left:$ $](c){}
        (1,0) node[insep,label=left:$ $](d){}
        (3,0) node[draw,circle,label=right:$ $](t){t}
        (3,1) node[insep,label=left:$ $](e){}
        (2,0) node[insep,label=left:$ $](f){}
        (3,-1) node[insep,label=left:$ $](g){}
        (4,0) node[insep,label=left:$ $](h){}
        ;
        \foreach \x/\y/\n in {b/a/2, s/b/0, a/d/2, s/d/3, e/a/1, d/f/2, c/g/1, e/f/0, t/f/1, h/t/1, h/e/1}{
          \draw[-triangle 45](\x) -- node[above]{\n} (\y);
        }
        \foreach \x/\y/\n in {c/b/2, f/g/3, g/h/2}{
          \draw[-triangle 45](\x) -- node[below]{\n} (\y);
        }
        \foreach \x/\y/\n in {a/s/1, s/c/0, e/t/0, g/t/2, d/c/3}{
          \draw[-triangle 45](\x) -- node[right]{\n} (\y);
        }
      \end{tikzpicture}
    \end{center}

  \item % #3
    Consider the network below with given capacity and flow values.
    (The  edge label $f,u$ means flow-value $f$ and capacity $u$.) 
    Find augmenting paths and augment the flow to a maximum flow.
    Provide the list of residual graphs AND augmenting paths.
    It other words, run Ford-Fulkerson algorithm. 
    \begin{center}
      \tikzset{insep/.style={inner sep=2pt, outer sep=0pt, circle, fill},}
      \begin{tikzpicture}[scale=2]
        \draw
        (0,0) node[insep,label=left:$s$](s){}
        (1,1) node[insep,label=left:$ $](a){}
        (1,0) node[insep,label=left:$ $](b){}
        (1,-1) node[insep,label=left:$ $](c){}
        (2,1) node[insep,label=left:$ $](d){}
        (2,0) node[insep,label=left:$ $](e){}
        (2,-1) node[insep,label=left:$ $](f){}
        (3,0) node[insep,label=right:$t$](t){}
        ;
        \foreach \x/\y/\n in {s/a/{3,3}, a/d/{3,7}, d/t/{2,2}, s/b/{1,4}, b/e/{1,1}, e/t/{1,8}, b/f/{0,2}}{
          \draw[-triangle 45](\x) -- node[above]{\n} (\y);
        }
        \foreach \x/\y/\n in {s/c/{2,8}, c/f/{2,4}, f/t/{3,6}}{
          \draw[-triangle 45](\x) -- node[below]{\n} (\y);
        }
        \foreach \x/\y/\n in {c/b/{0,2}, b/a/{0,3}, d/e/{1,5}, e/f/{1,1}}{
          \draw[-triangle 45](\x) -- node[right]{\n} (\y);
        }
      \end{tikzpicture}
    \end{center}



  \item % #4
    Let $(G,u, s, t)$ be a network, and let $\delta^+(X)$ and $\delta^+(Y)$ be
    minimum $s$-$t$-cuts in $(G,u)$.
    Show that $\delta^+(X\cap Y)$ and $\delta^+(X \cup Y)$ are also minimum
    $s$-$t$-cuts in $(G,u)$.

  \item % #5
    Show that in case of irrational capacities, the Ford-Fulkerson algorithm may
    not terminate at all.
    Hint: See the Korte book (in particular exercises on page 199.).
    It contains the following network:
    \begin{center}
      \tikzset{insep/.style={inner sep=2pt, outer sep=0pt, circle, fill},}
      \begin{tikzpicture}[scale=2]
        \draw
        (0,-0.5) node[insep,label=left:$s$](s){}
        (1,1) node[insep,label=left:$ $](a1){}
        (1,0) node[insep,label=left:$ $](a2){}
        (1,-1) node[insep,label=left:$ $](a3){}
        (1,-2) node[insep,label=left:$ $](a4){}
        (2,1) node[insep,label=left:$ $](b1){}
        (2,0) node[insep,label=left:$ $](b2){}
        (2,-1) node[insep,label=left:$ $](b3){}
        (2,-2) node[insep,label=left:$ $](b4){}
        (3,-0.5) node[insep,label=right:$t$](t){}
        ;
        \foreach \x\n in {1,2,3,4}{
          \foreach \y\n in {1,2,3,4}{
            \draw (a\x) -- (b\y);
          }
        }
        \foreach \x\n in {1,2,3,4}{
          \draw (s) -- (a\x);
          \draw (b\x) -- (t);
        }
        \draw[-triangle 45, line width=1pt](a1) --node[above]{1} (b1);
        \draw[-triangle 45, line width=1pt](a2) --node[above]{$\sigma$} (b2);
        \draw[-triangle 45, line width=1pt](a3) --node[above]{$\sigma^2$} (b3);
      \end{tikzpicture}
    \end{center}
    Where $\sigma = \frac{\sqrt{5}-1}{2}$. Note that $\sigma$ satisfies
    $\sigma^n = \sigma^{n+1} + \sigma^{n+2}$. All other capacities are 1.

  \item % #6
    Red-Blue meta algorithm for MST.
    Let $G$ be a graph and $w$ be a weight assignment to $E(G)$.
    Assume that all weights are distinct.
    Start with all edges being uncolored.
    Apply the following rules as long as possible.
    \begin{itemize}
      \item if $e \in E$ is in a cycle $C$ where $e$ is the heaviest edge, color $e$ red
      \item if there is a cut where $e\in E$ is the lightest edge, color $e$ blue.
    \end{itemize}
    Claim is that blue edges form a minimum spanning tree.
    \begin{itemize}
      \item Show that red edge cannot be in MST.
      \item Show that blue edge must be in MST.
      \item Show that blue edges form a tree
      \item Show that every edge gets colored.
      \item Show that no edge satisfies both red and blue criteria. (i.e. every edge has one color).
    \end{itemize}

  \item % #7
    Implement Edmonds-Karp algorithm and run it on the network from question three.
    Print the sequence of augmenting paths used by your implementation.
    Print the flow and its value.

    I implemented the Edmonds-Karp algorithm in the following function.
    \lstinputlisting[language=Sage]{edmondsKarp.sage}

    This algorithm using a breadth first search which is implemented in the
    following function.
    \lstinputlisting[language=Sage]{breadthFirstSearch.sage}

    In order to run this algorithm on the graph from problem 3, I first
    relabeled the vertices in this graph.
    The graph was relabeled as shown below.
    \begin{center}
      \tikzset{insep/.style={inner sep=2pt, outer sep=0pt, circle, fill},}
      \begin{tikzpicture}[scale=2]
        \draw
        (0,0) node[insep,label=left:$s$](s){}
        (1,1) node[insep,label=120:$a$](a){}
        (1,0) node[insep,label=45:$b$](b){}
        (1,-1) node[insep,label=210:$c$](c){}
        (2,1) node[insep,label=45:$d$](d){}
        (2,0) node[insep,label=45:$e$](e){}
        (2,-1) node[insep,label=-45:$f$](f){}
        (3,0) node[insep,label=right:$t$](t){}
        ;
        \foreach \x/\y in {s/a, a/d, d/t, s/b, b/e, e/t, b/f, s/c, c/f, f/t, c/b, b/a, d/e, e/f}{
          \draw[-triangle 45](\x) -- (\y);
        }
      \end{tikzpicture}
    \end{center}
    \lstinputlisting[language=Sage]{06_7.sage}
    This is the output of this script.
    Each list is the augmenting path.
    Each tuple is an edge in the augmenting path, with first entry the starting
    vertex, the second entry the ending vertex, and the third entry the
    available flow.
    The dictionary shows the flow on each edge in the form edge:flow.
    \begin{verbatim}
[('s', 'a', 3), ('a', 'd', 7), ('d', 't', 2)]
[('s', 'b', 4), ('b', 'e', 1), ('e', 't', 8)]
[('s', 'b', 3), ('b', 'f', 2), ('f', 't', 6)]
[('s', 'c', 8), ('c', 'f', 4), ('f', 't', 4)]
[('s', 'a', 1), ('a', 'd', 5), ('d', 'e', 5), ('e', 't', 7)]
[('s', 'b', 1), ('b', 'a', 3), ('a', 'd', 4), ('d', 'e', 4), ('e', 't', 6)]
[('s', 'c', 4), ('c', 'b', 2), ('b', 'a', 2), ('a', 'd', 3),
  ('d', 'e', 3), ('e', 't', 5)]
{
  ('b', 'f', 2): 2,
  ('c', 'b', 2): 2,
  ('b', 'a', 3): 3,
  ('f', 't', 6): 6,
  ('s', 'b', 4): 4,
  ('e', 'f', 1): 0,
  ('a', 'd', 7): 6,
  ('s', 'c', 8): 6,
  ('d', 'e', 5): 4,
  ('s', 'a', 3): 3,
  ('b', 'e', 1): 1,
  ('c', 'f', 4): 4,
  ('d', 't', 2): 2,
  ('e', 't', 8): 5
  }
}
    \end{verbatim}
    This flow can also be shown on the graph as follows.
    \begin{center}
      \tikzset{insep/.style={inner sep=2pt, outer sep=0pt, circle, fill},}
      \begin{tikzpicture}[scale=2]
        \draw
        (0,0) node[insep,label=left:$s$](s){}
        (1,1) node[insep,label=120:$a$](a){}
        (1,0) node[insep,label=45:$b$](b){}
        (1,-1) node[insep,label=210:$c$](c){}
        (2,1) node[insep,label=45:$d$](d){}
        (2,0) node[insep,label=45:$e$](e){}
        (2,-1) node[insep,label=-45:$f$](f){}
        (3,0) node[insep,label=right:$t$](t){}
        ;
        \foreach \x/\y/\n in {s/a/{3,3}, a/d/{6,7}, d/t/{2,2}, s/b/{4,4}, b/e/{1,1}, e/t/{5,8}, b/f/{2,2}}{
          \draw[-triangle 45](\x) -- node[above]{\n} (\y);
        }
        \foreach \x/\y/\n in {s/c/{6,8}, c/f/{4,4}, f/t/{6,6}}{
          \draw[-triangle 45](\x) -- node[below]{\n} (\y);
        }
        \foreach \x/\y/\n in {c/b/{2,2}, b/a/{3,3}, d/e/{4,5}, e/f/{0,1}}{
          \draw[-triangle 45](\x) -- node[right]{\n} (\y);
        }
      \end{tikzpicture}
    \end{center}
\end{enumerate}
\end{document}
