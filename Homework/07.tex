\documentclass[11pt, oneside]{article}
\usepackage[letterpaper, margin=2cm]{geometry}
\usepackage{MATH566}
%\usepackage{sagetex}

\begin{document}
\noindent \textbf{\Large{Caleb Logemann \\
MATH 566 Discrete Optimization\\
Homework 7
}}

%\lstinputlisting[language=Sage]{03_2.sage}
\begin{enumerate}
  \item % #1 Done
    A cut of $G$ is \emph{minimal} if there is no cut of $G$ properly contained
    in it.
    Prove that the random contraction algorithm returns only minimal cuts.

    \begin{proof}
      Let $G$ be a connected graph and let the set of edges
      $E = \set{e_1, \ldots, e_k}$ be a cut returned by the random contraction
      algorithm.
      Now suppose $E$ is not a minimal cut, that is there exists a proper subset
      $F \subset E$ that is also a cut of $G$.
      Thus there is an edge $e$, such that $e \in E$ and $e \not\in F$.
      Let $E$ be the cut between the set of vertices $X$ and $\bar{X}$ of $G$.
      Since $e \in E$, that implies that $e = \set{a, b}$ where $a \in X$ and
      $b \in \bar{X}$.
      Since $E$ was the cut returned by the random contraction algorithm, this
      implies that at the last step $X$ and $\bar{X}$ were both contracted to a
      single vertex.
      If a set of vertices is contracted to a single vertex they must have been
      connected in the original graph, because the algorithm only contracts
      along edges.
      Thus the $X$ and $\bar{X}$ when considered as subgraphs must be connected.
      Now consider the cut $F$, since the cut is a subset of $E$, the sets $X$
      and $\bar{X}$ are still connected with edges in $F$ removed.
      Now consider $u \in X$ and $v \in \bar{X}$, since $X$ is connected there
      exists a path in $G/F$ from $u$ to $a$, and since $\bar{X}$ is connected
      there is a path from $b$ to $v$.
      Now since $e \not\in F$, there is a path from $u$ tp $v$.
      This shows that there is a path from any two vertices in $G/F$, and thus
      $F$ is not a cut at all.
      Since $F$ is not a cut, $E$ must be a minimal cut, showing that the
      random contraction algorithm does return minimal cuts.
    \end{proof}

  \item % #2 Done
    Implement Node Identification Minimum Cut Algorithm.
    Try it on the following graph:
    \begin{center}
      \tikzset{insep/.style={inner sep=2pt, outer sep=0pt, circle, fill},}
      \begin{tikzpicture}[scale=1.3]
        \draw
          (0,0) node[insep,label=below:$g$](g){}
          (2,0) node[insep,label=below:$f$](f){}
          (2,1) node[insep,label=right:$e$](e){}
          (2,2) node[insep,label=right:$a$](a){}
          (1,3) node[insep,label=above:$b$](b){}
          (0,1) node[insep,label=below left:$d$](d){}
          (0,2) node[insep,label=above:$c$](c){}
          (-1,1) node[insep,label=left:$h$](h){}
        ;
        \draw
          \foreach \x/\y/\t in {g/f/6, d/e/3, c/a/2, c/b/3, b/a/5,h/c/2, h/d/1}{
            (\x)--node[pos=0.5,label=above:$\t$]{} (\y)
          }
          (h)--node[pos=0.5,label=below:$5$]{} (g)
          \foreach \x/\y/\t in {g/d/2, d/c/4, f/e/2, e/a/3}{
            (\x)--node[pos=0.5,label=right:$\t$]{} (\y)
          }
        ;
      \end{tikzpicture}
    \end{center}

    I implemented the Node Identification algorithm with the following
    functions.
    \lstinputlisting[language=Sage]{nodeIdentification.sage}
    \lstinputlisting[language=Sage]{contractVertices.sage}

    The following script uses these function to run the Node Identification
    algorithm on the graph given for this problem.
    \lstinputlisting[language=Sage]{07_2.sage}

    The output of this script is
    \begin{verbatim}
      minimum cut: set(['h', 'g', 'f'])
      minimum cut value: 7
    \end{verbatim}

  \item % #3 Done
    Implement Random Contraction Algorithm.
    Try it on the following graph. 
    Run it many times and based on your experiment conclude what is the probability that that your algorithm succeeds on this graph.
    \begin{center}
      \tikzset{insep/.style={inner sep=2pt, outer sep=0pt, circle, fill},}
      \begin{tikzpicture}[scale=1.3]
        \draw
          (0,0) node[insep,label=below:$g$](g){}
          (2,0) node[insep,label=below:$f$](f){}
          (2,1) node[insep,label=right:$e$](e){}
          (2,2) node[insep,label=right:$a$](a){}
          (1,3) node[insep,label=above:$b$](b){}
          (0,1) node[insep,label=below left:$d$](d){}
          (0,2) node[insep,label=above:$c$](c){}
          (-1,1) node[insep,label=left:$h$](h){}
        ;
        \draw
          \foreach \x/\y/\t in {g/f/6, d/e/3, c/a/2, c/b/3, b/a/5,h/c/2, h/d/1}{
            (\x)--node[pos=0.5,label=above:$\t$]{} (\y)
          }
          (h)--node[pos=0.5,label=below:$5$]{} (g)
          \foreach \x/\y/\t in {g/d/2, d/c/4, f/e/2, e/a/3}{
            (\x)--node[pos=0.5,label=right:$\t$]{} (\y)
          }
          ;
      \end{tikzpicture}
    \end{center}

    I implemented the random contraction algorithm with the following function.
    This function makes use of the contractVertices function, which was shown
    in the previous exercise, but is ommited here.
    \lstinputlisting[language=Sage]{randomContraction.sage}

    The following script uses these function to run the randomContraction
    algorithm on the graph given for this problem.
    \lstinputlisting[language=Sage]{07_3.sage}

    The output of this script is shown below.
    As it can be seen with a hundred trials the algorithm finds the minimum
    cut.
    \begin{verbatim}
      minimum cut: set(['a', 'c', 'b', 'e', 'd'])
      minimum cut value: 7
    \end{verbatim}

    Also the script computes an experimental probability that the algorithm
    finds the correct minimum cut, based on the number of trials that are run.
    Note the interesting fact that at just 20 trials the probability of finding
    the right cut is over 90\%.
    This is much higher than the lower bound given in the notes.
    \begin{center}
    \begin{tabular}{cc}
      \toprule
      number of trials & success percentage \\
      \midrule
       1 & 0.126 \\
       5 & 0.438 \\
      10 & 0.668 \\
      15 & 0.834 \\
      20 & 0.918 \\
      25 & 0.956 \\
      30 & 0.972 \\
      35 & 0.980 \\
      40 & 0.992 \\
      45 & 0.992 \\
      50 & 0.998 \\
      \bottomrule
    \end{tabular}
    \end{center}

  \item % #4 Done
    Construct Gomory-Hu Tree for the following graph using the algorithm from the class. 
    Numbers on edges correspond to capacities.
    Show Show steps after every new cut.
    \begin{center}
      \tikzset{insep/.style={inner sep=2pt, outer sep=0pt, circle, fill},}
      \begin{tikzpicture}
        \draw
        \foreach \x/\a in {0/a,60/b,120/c,180/d,240/e,300/f}{
          (\x:2.5) node[insep](\a){}
        }
        (a) -- node[right]{1} (b) -- node[above]{3} (c) -- node[left]{2}
        (d) -- node[below]{1}(e) -- node[below]{3}(f) -- node[right]{1}(a)
        (b) -- node[below]{2} (d) -- node[below]{1} (f) -- node[right]{2} (b)
        ;
      \end{tikzpicture}
    \end{center}
    Check your answer with Sage using method \verb|Graph.gomory_hu_tree()|.
    Do not implement it yourself, just run it.

    First in order to run the Gomory Hu algorithm, I will label the vertices.
    \tikzstyle{vertex_style}=[circle, draw, inner sep=2pt]
    \tikzstyle{dot_style}=[circle, draw,fill=black,radius=0.002pt,inner sep=1pt]
    \begin{center}
      \begin{tikzpicture}[scale=1]
        \node[vertex_style](1) at (1.5,2) {$a$};
        \node[vertex_style](2) at (3.5,2) {$b$};
        \node[vertex_style](3) at (5,0) {$c$};
        \node[vertex_style](4) at (3.5,-2) {$d$};
        \node[vertex_style](5) at (1.5,-2) {$e$};
        \node[vertex_style](6) at (0,0) {$f$};

        \draw (1) --node[above]{$3$} (2);
        \draw (1) --node[above]{$2$} (6);
        \draw (2) --node[right]{$1$} (3) ;
        \draw (2) --node[right]{$2$} (4) ;
        \draw (2) --node[below]{$2$} (6) ;
        \draw (3) --node[below]{$1$} (4);
        \draw (4) --node[below]{$3$} (5);
        \draw (4) --node[above]{$1$} (6);
        \draw (5) --node[below]{$1$} (6);
      \end{tikzpicture}
    \end{center}

    First start with a tree that is $T = (\set{V}, \emptyset)$.
    This tree looks like.
    \begin{center}
      \begin{tikzpicture}
        \node[vertex_style] at (0, 0) {
          \begin{tikzpicture}
            \node[vertex_style](1) at (1.5,2) {$a$};
            \node[vertex_style](2) at (3.5,2) {$b$};
            \node[vertex_style](3) at (5,0) {$c$};
            \node[vertex_style](4) at (3.5,-2) {$d$};
            \node[vertex_style](5) at (1.5,-2) {$e$};
            \node[vertex_style](6) at (0,0) {$f$};

            \draw (1) --node[above]{$3$} (2);
            \draw (1) --node[above]{$2$} (6);
            \draw (2) --node[right]{$1$} (3) ;
            \draw (2) --node[right]{$2$} (4) ;
            \draw (2) --node[below]{$2$} (6) ;
            \draw (3) --node[below]{$1$} (4);
            \draw (4) --node[below]{$3$} (5);
            \draw (4) --node[above]{$1$} (6);
            \draw (5) --node[below]{$1$} (6);
          \end{tikzpicture}
        };
      \end{tikzpicture}
    \end{center}
    Now I will chose to find the minimum $a$-$b$ cut, which corresponds to
    finding the minimum $a$-$b$ flow in the following graph.
    The minimum cut found is shown in red.
    \begin{center}
      \begin{tikzpicture}[scale=1]
        \node[vertex_style](1) at (1.5,2) {$a$};
        \node[vertex_style](2) at (3.5,2) {$b$};
        \node[vertex_style](3) at (5,0) {$c$};
        \node[vertex_style](4) at (3.5,-2) {$d$};
        \node[vertex_style](5) at (1.5,-2) {$e$};
        \node[vertex_style](6) at (0,0) {$f$};

        \draw (1) --node[above]{$3$} (2);
        \draw (1) --node[above]{$2$} (6);
        \draw (2) --node[right]{$1$} (3) ;
        \draw (2) --node[right]{$2$} (4) ;
        \draw (2) --node[below]{$2$} (6) ;
        \draw (3) --node[below]{$1$} (4);
        \draw (4) --node[below]{$3$} (5);
        \draw (4) --node[above]{$1$} (6);
        \draw (5) --node[below]{$1$} (6);
        \draw[color=red,line width=1pt] (0,0.5)--(3.5,2.5);
      \end{tikzpicture}
    \end{center}
    Using this cut the new Gomory Hu tree is
    \begin{center}
      \begin{tikzpicture}
        \node[vertex_style](1) at (0, 0) {$a$};
        \node[vertex_style](X) at (5, 0) {
          \begin{tikzpicture}
            \node[vertex_style](2) at (3.5,2) {$b$};
            \node[vertex_style](3) at (5,0) {$c$};
            \node[vertex_style](4) at (3.5,-2) {$d$};
            \node[vertex_style](5) at (1.5,-2) {$e$};
            \node[vertex_style](6) at (0,0) {$f$};

            \draw (2) --node[right]{$1$} (3) ;
            \draw (2) --node[right]{$2$} (4) ;
            \draw (2) --node[below]{$2$} (6) ;
            \draw (3) --node[below]{$1$} (4);
            \draw (4) --node[below]{$3$} (5);
            \draw (4) --node[above]{$1$} (6);
            \draw (5) --node[below]{$1$} (6);
          \end{tikzpicture}
        };
        \draw (1)--node[above]{5}(X);
      \end{tikzpicture}
    \end{center}

    Now I will find the minimum $b$-$c$ cut in the following graph.
    The minimum cut found is shown in red.
    \begin{center}
      \begin{tikzpicture}[scale=1]
        \node[vertex_style](1) at (1.5,2) {$a$};
        \node[vertex_style](2) at (3.5,2) {$b$};
        \node[vertex_style](3) at (5,0) {$c$};
        \node[vertex_style](4) at (3.5,-2) {$d$};
        \node[vertex_style](5) at (1.5,-2) {$e$};
        \node[vertex_style](6) at (0,0) {$f$};

        \draw (1) --node[above]{$3$} (2);
        \draw (1) --node[above]{$2$} (6);
        \draw (2) --node[right]{$1$} (3) ;
        \draw (2) --node[right]{$2$} (4) ;
        \draw (2) --node[below]{$2$} (6) ;
        \draw (3) --node[below]{$1$} (4);
        \draw (4) --node[below]{$3$} (5);
        \draw (4) --node[above]{$1$} (6);
        \draw (5) --node[below]{$1$} (6);
        \draw[color=red,line width=1pt] (4,-2)--(4,2);
      \end{tikzpicture}
    \end{center}
    Using this cut the new Gomory Hu tree is
    \begin{center}
      \begin{tikzpicture}
        \node[vertex_style](a) at (0, 0) {$a$};
        \node[vertex_style](X) at (5, 0) {
          \begin{tikzpicture}
            \node[vertex_style](2) at (3.5,2) {$b$};
            \node[vertex_style](4) at (3.5,-2) {$d$};
            \node[vertex_style](5) at (1.5,-2) {$e$};
            \node[vertex_style](6) at (0,0) {$f$};

            \draw (2) --node[right]{$2$} (4) ;
            \draw (2) --node[below]{$2$} (6) ;
            \draw (4) --node[below]{$3$} (5);
            \draw (4) --node[above]{$1$} (6);
            \draw (5) --node[below]{$1$} (6);
          \end{tikzpicture}
        };
        \node[vertex_style](c) at (10, 0){$c$};
        \draw (a)--node[above]{5}(X);
        \draw (X)--node[above]{2}(c);
      \end{tikzpicture}
    \end{center}

    Now I will find the minimum $b$-$d$ cut in the following graph.
    The minimum cut found is shown in red.
    \begin{center}
      \begin{tikzpicture}[scale=1]
        \node[vertex_style](1) at (1.5,2) {$a$};
        \node[vertex_style](2) at (3.5,2) {$b$};
        \node[vertex_style](3) at (5,0) {$c$};
        \node[vertex_style](4) at (3.5,-2) {$d$};
        \node[vertex_style](5) at (1.5,-2) {$e$};
        \node[vertex_style](6) at (0,0) {$f$};

        \draw (1) --node[above]{$3$} (2);
        \draw (1) --node[above]{$2$} (6);
        \draw (2) --node[right]{$1$} (3) ;
        \draw (2) --node[right]{$2$} (4) ;
        \draw (2) --node[below]{$2$} (6) ;
        \draw (3) --node[below]{$1$} (4);
        \draw (4) --node[below]{$3$} (5);
        \draw (4) --node[above]{$1$} (6);
        \draw (5) --node[below]{$1$} (6);
        \draw[color=red,line width=1pt] (0,-1)--(5,-1);
      \end{tikzpicture}
    \end{center}
    Using this cut the new Gomory Hu tree is shown below.
    \begin{center}
      \begin{tikzpicture}
        \node[vertex_style](a) at (0, 0) {$a$};
        \node[vertex_style](X) at (5, 0) {
          \begin{tikzpicture}
            \node[vertex_style](2) at (1.75,1) {$b$};
            \node[vertex_style](6) at (0,0) {$f$};
            \draw (2) --node[below]{$2$} (6) ;
          \end{tikzpicture}
        };
        \node[vertex_style](Y) at (5, -5) {
          \begin{tikzpicture}
            \node[vertex_style](4) at (3.5,-2) {$d$};
            \node[vertex_style](5) at (1.5,-2) {$e$};
            \draw (4) --node[below]{$3$} (5);
          \end{tikzpicture}
        };
        \node[vertex_style](c) at (10, 0){$c$};
        \draw (a)--node[above]{5}(X);
        \draw (X)--node[above]{2}(c);
        \draw (X)--node[right]{5}(Y);
      \end{tikzpicture}
    \end{center}

    Now I will find the minimum $b$-$f$ cut in the following graph, where $e$
    and $d$ are contracted.
    \begin{center}
      \begin{tikzpicture}[scale=1]
        \node[vertex_style](1) at (1.5,2) {$a$};
        \node[vertex_style](2) at (3.5,2) {$b$};
        \node[vertex_style](3) at (5,0) {$c$};
        \node[vertex_style](4) at (2.5,-2) {$d,e$};
        \node[vertex_style](6) at (0,0) {$f$};

        \draw (1) --node[above]{$3$} (2);
        \draw (1) --node[above]{$2$} (6);
        \draw (2) --node[right]{$1$} (3) ;
        \draw (2) --node[right]{$2$} (4) ;
        \draw (2) --node[below]{$2$} (6) ;
        \draw (3) --node[below]{$1$} (4);
        \draw (4) --node[above]{$2$} (6);
        \draw[color=red, line width=1pt](1,2)--(1,-2);
      \end{tikzpicture}
    \end{center}
    Using this cut the new Gomory Hu graph is 
    \begin{center}
      \begin{tikzpicture}
        \node[vertex_style](a) at (1.5,2) {$a$};
        \node[vertex_style](b) at (3.5,2) {$b$};
        \node[vertex_style](c) at (5,0) {$c$};
        \node[vertex_style](f) at (0,0) {$f$};
        \node[vertex_style](X) at (2.5, -2) {
          \begin{tikzpicture}
            \node[vertex_style](4) at (3.5,-2) {$d$};
            \node[vertex_style](5) at (1.5,-2) {$e$};
            \draw (4) --node[below]{$3$} (5);
          \end{tikzpicture}
        };
        \draw (a)--node[above]{5}(b);
        \draw (b)--node[above]{2}(c);
        \draw (b)--node[above]{6}(f);
        \draw (b)--node[right]{5}(X);
      \end{tikzpicture}
    \end{center}

    Lastly I will find the minimum $d$-$e$ cut in the following contracted
    graph.
    \begin{center}
      \begin{tikzpicture}[scale=1]
        \node[vertex_style](1) at (2.5,0) {$a, b, c, f$};
        \node[vertex_style](4) at (3.5,-2) {$d$};
        \node[vertex_style](5) at (1.5,-2) {$e$};

        \draw (1) --node[right]{$4$} (4);
        \draw (1) --node[left]{$1$} (5);
        \draw (4) --node[below]{$3$} (5);
        \draw[color=red, line width=1pt] (1.3, -.5) -- (3.3, -2.5);
      \end{tikzpicture}
    \end{center}
    Thus the full Gomory Hu tree is
    \begin{center}
      \begin{tikzpicture}
        \node[vertex_style](a) at (1.5,2) {$a$};
        \node[vertex_style](b) at (3.5,2) {$b$};
        \node[vertex_style](c) at (5,0) {$c$};
        \node[vertex_style](d) at (3.5,-2) {$d$};
        \node[vertex_style](e) at (1.5,-2) {$e$};
        \node[vertex_style](f) at (0,0) {$f$};
        \draw (a)--node[above]{5}(b);
        \draw (b)--node[above]{2}(c);
        \draw (b)--node[above]{6}(f);
        \draw (b)--node[right]{5}(d);
        \draw (d)--node[below]{4}(e);
      \end{tikzpicture}
    \end{center}

    I checked this in Sage with the following script.
    \lstinputlisting[language=Sage]{07_4.sage}
    The output of this script is the following image of the Gomory Hu tree.
    \begin{center}
      \includegraphics[scale=.8]{Figures/07_1.png}
    \end{center}
    This graph has a different structure but it has the same weights
    As the Gomory Hu tree is not unique I believe this confirms the
    tree that I arrived at.

\end{enumerate}
\end{document}
